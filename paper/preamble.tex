\documentclass[a5paper, twoside]{article}
\usepackage[top=2.0cm,bottom=2.0cm,left=2.0cm,right=2.0cm]{geometry}

\usepackage{setspace, multirow, xltxtra, longtable, ltxtable, amsmath, fancybox, marginnote, rotating, needspace, titletoc, enumerate}
\usepackage[authorformat=year, authorformat=smallcaps, oxford=true, ibidem=nostrict,
bibformat=numbered,
bibformat=ibidem]{jurabib}
%\jurabibsetup{%
%	bibformat={ibidem, numbered}
%}
\setlength{\jbbibhang}{4em}


\renewcommand{\textsc}[1]{{\fontspec{TeX Gyre Pagella}\scshape#1}}
\newcommand{\textscbf}[1]{\textsc{\textbf{#1}}}

\usepackage{comment, framed}
\specialcomment{paper}{\begin{leftbar}}{\end{leftbar}}
%\excludecomment{paper}
\specialcomment{extra}{\fontsize{8}{8}\fontspec{Gentium Basic}}{\fontsize{10}{10}\fontspec{Gentium Basic}}

%\hyphenpenalty=5000
\tolerance=99999
\widowpenalty=1000
\clubpenalty=1000

\setlength\parskip{\medskipamount}
\setlength\parindent{0pt}

\IfFileExists{url.sty}{\usepackage{url}}
{\newcommand{\url}{\texttt}}

\newcommand{\zero}{{\fontspec{Apple Symbols}∅}}

\newcommand{\nosic}{}

%\newcommand{\C}[1]{{\fontsize{11}{11}\fontspec[Language=Welsh]{Junicode}{#1}}}
%\newcommand{\C}[1]{{\fontspec[Language=Welsh,SmallCapsFont={LMRoman10 Caps}, HyphenChar=None]{Optima}{#1}}}
%\newcommand{\C}[1]{{\fontspec[Language=Welsh,SmallCapsFont={LMRoman10 Caps}, HyphenChar=None]{Gentium Basic}\textit{#1}}}
\newcommand{\C}[1]{{\fontspec{Palemonas MUFI}\textit{#1}}}
\newcommand{\Cpar}[1]{{\fontspec[Language=Welsh,SmallCapsFont={LMRoman10 Caps}, HyphenChar=None]{Gentium Basic}#1}}
\newcommand{\BHtr}[1]{{\fontspec[HyphenChar=None]{Gentium}\textit{#1}}}
\newcommand{\slot}[1]{{{\textsc{#1}}}}
\newcommand{\trans}[1]{{\LR{\fontspec{TeX Gyre Pagella}\textit{#1}}}}

\usepackage[nooverlap, latin]{ruby}
\renewcommand{\rubysize}{0.75}
%\renewcommand{\rubysep}{-1cm}
%\renewcommand{\rubysep}{-0.9cm}
\renewcommand{\ruby}[2]{#1}

%\newcommand{\gl}[2]{\ruby{#1}{\fontspec{Gill Sans Light}\mdseries\upshape #2\,}}
\newcommand{\gl}[2]{#1}
\newcommand{\Hebrew}[1]{{\RL{\fontspec[Script=Hebrew]{Guttman Hodes}#1}}}
\newcommand{\gram}[1]{{\fontspec{TeX Gyre Pagella}\textsc{#1}}}

%\newcommand{\appmargin}[1]{\marginpar{\vskip-0.5em\R{\begin{flushright}
%
%	{\scriptsize #1}\end{flushright}}}}
\newcommand{\appmargin}[1]{}

\newcommand{\Chl}[1]{\highlight{#1}}
\newcommand{\BHhl}[1]{\BHhighlight{#1}}
\newcommand{\highlight}[1]{{{\underline{#1}}}}
\newcommand{\BHhighlight}[1]{{{\underline{#1}}}}
\newcommand{\hl}[1]{\textbf{#1}}
%\newcommand{\quo}[1]{{\fontsize{8}{8}\textit{#1}}}
\newcommand{\quo}[1]{{\textit{#1}}}
\newcommand{\secu}[1]{\textbf{#1}}
%\newcommand{\highlight}[1]{{\addfontfeature{Color=600000FF}{#1}}}
%\renewcommand{\emph}[1]{\textbf{#1}}
\newcommand{\IPA}[1]{{\fontspec{Gentium}#1}}

%\clubpenalty=300
%\widowpenalty=300
%\widowpenalty=10000
%\clubpenalty=10000
%\raggedbottom

%\newcommand{\patternlimit}{{\fontspec{Apple Symbols}◇}}
\newcommand{\patternlimit}{{\fontspec{Apple Symbols}}}
\newcommand{\pattern}[1]{\patternlimit~{#1}~\patternlimit}

\newcommand{\pA}{A}
\newcommand{\pB}{B}
\newcommand{\pC}{C}
\newcommand{\Aplus}{\mbox{\pA{$+$}}}
\newcommand{\Aminus}{\mbox{\pA{$-$}}}



%%% Paper-specific %%%

\newcommand{\llyfr}[1] {%
	\textsc{
	\ifthenelse{\equal{#1}{1}}{Gen}{%
	\ifthenelse{\equal{#1}{2}}{Ex}{%
	\ifthenelse{\equal{#1}{3}}{Lev}{%
	\ifthenelse{\equal{#1}{4}}{Num}{%
	\ifthenelse{\equal{#1}{5}}{Deu}{#1}}}}}}.%
}
%\newcommand{\vref}[4]{\ref{verse:#1:#2:#3:#4}\marginnote{\setstretch{0.75}\begin{tabular}{c}\llyfr{1}~#2:#3\\עמ'~\pageref{verse:#1:#2:#3:#4}\end{tabular}}}
\newcommand{\vref}[4]{\ref{verse:#1:#2:#3:#4}}
\newcommand{\xvref}[4]{***} % references for yet to be written exx
\newcommand{\secref}[1]{\mbox{\LR{{\fontspec{SBL Hebrew}§}\LR{\ref{#1}}}}}
\newcommand{\sic}{\ {(\textit{sic})}}
\newcommand{\bcite}[3]{\llyfr{#1}~#2:#3}

%%% Examples %%%
%\edef\marginnotetextwidth{1cm}
\newcounter{exampleno}
\setcounter{exampleno}{0}
\newenvironment{bareexample}[1] {
	%\begin{changemargin}{3.0cm}{0cm}
	\needspace{12\baselineskip}\refstepcounter{exampleno}
		%\ifthenelse{\isodd{\thepage}}
		%	{\marginnote{\begin{tabular}{ll}\framebox{\arabic{exampleno}}&#1\end{tabular}}[-0.6em]}
		%	{\marginnote{\begin{tabular}{ll}#1&\framebox{\arabic{exampleno}}\end{tabular}}[-0.6em]}
		{\marginnote{\small\centering\framebox{\arabic{exampleno}}\\[0.20cm]#1}[0.6em]}
	}
	{
	%\end{changemargin}
	}

\newenvironment{example}[5] % #1=book, #2=chapter, #3=verse, #4=copy (of the same verse; different copies have different hightlighting), #5=shift (of verse number)
	{
	\setstretch{0.51}
	\vspace{1em}
	\begin{bareexample}
		{{
			{
			\setstretch{0.75}
			{\llyfr{#1}}\\
			\LR{{#2}{:}%
			{#3}}%
%			\ifthenelse{\not\equal{#4}{}}{{\small .#4}}{}%
			\ifthenelse{\not\equal{#5}{}}{*}{}
			}
			}}\label{verse:#1:#2:#3:#4}
		\begin{longtable}{p{0.46\linewidth}p{0.46\linewidth}}
	}
	{
		\end{longtable}
		\end{bareexample}
		\vspace{-1em}
	}
\newcommand{\explain}{\noindent$\rightarrow$\ }

\newenvironment{bilingquote}[0]
	{\begin{longtable}{p{0.46\linewidth}p{0.46\linewidth}}}
	{\end{longtable}}

	\newcommand{\Lcol}[1]{{\setLR{\vskip-2em}\C{%
	%\addfontfeature{HyphenChar="2E17}%ADDME WHEN PRINTING***
	#1}}}
\newcommand{\Rcol}[1]{{\setRL{\vskip-2em}\BH{#1}}}
\newcommand{\transline}[2]
%	{{\Lcol{#1}\vspace{0.0em}} &
%	{{\Rcol{#2}}\vspace{0.0em}}\\}
	{#1 & #2\\}
\newcommand{\cotransline}[2]{\transline{\scriptsize#1}{\scriptsize#2}}
\newcommand{\quoling}[4]% Hebrew (Tiberean), Cymraeg, Hebrew (Latin transcription), KJV
{
%\begin{bilingquote}
	\transline{{\fontspec[HyphenChar=None]{Junicode}#2}}{\fontspec[HyphenChar=None]{Gentium}#3}\\
	\transline{{\small\fontspec{Gentium}#4}}{\vspace{-0.9em}\begin{RTL}{\BH{\small#1}}\end{RTL}}
%\end{bilingquote}
}


\newcounter{pointno}
\setcounter{pointno}{1}
%\reversemarginpar
\renewcommand*{\raggedleftmarginnote}{\raggedleft}
\newcommand{\hopoint}{\marginnote{\ovalbox{\Roman{pointno}}}\stepcounter{pointno}}

\newenvironment{changemargin}[2]{%
 \begin{list}{}{%
  \setlength{\topsep}{0pt}%
  \setlength{\leftmargin}{#1}%
  \setlength{\rightmargin}{#2}%
  \setlength{\listparindent}{\parindent}%
  \setlength{\itemindent}{\parindent}%
  \setlength{\parsep}{\parskip}%
 }%
\item[]}{\end{list}}

\makeatletter
\renewcommand{\section}{\@startsection{section}{1}{0mm}
	{3ex plus1ex minus1ex}%
	{\smallskipamount}{\fontspec{Gentium}\Large}}%
\renewcommand{\subsection}{\@startsection{subsection}{1}{0mm}
	{3ex plus1ex minus1ex}%
	{\smallskipamount}{\fontspec{Gentium}\large}}%
\renewcommand{\subsubsection}{\@startsection{subsubsection}{1}{0mm}
	{3ex plus1ex minus1ex}%
	{\smallskipamount}{\fontspec{Gentium}}}%
\makeatother

\usepackage{tikz}
\usetikzlibrary{arrows, decorations.pathmorphing, backgrounds, positioning, fit, petri}

\newcommand{\paradigma}[1]{\begin{tabular}{|c|}#1\end{tabular}}

\newcommand{\shama}{{\fontspec[HyphenChar=None]{Gentium}\textit{{šåmaʿ}}}}
\newcommand{\YHWH}{{\fontspec[HyphenChar=None]{Gentium}\textsc{{yhwh}}}}
\newcommand{\LORD}{{\textsc{\mbox{lord}}}}
\newcommand{\click}{{\fontspec{FreeSerif}✴}}
\newcommand{\point}[1]{[\textit{#1}]}
\newcommand{\optout}[1]{{\small [{#1}]}}


\usepackage{fontspec, bidi}
\defaultfontfeatures{Mapping=tex-text}
\setmainfont[Language=English]{Gentium Book Basic}
\setmonofont{LMTypewriter10 Regular}
\newcommand{\BH}[1]{\RL{\fontspec[SmallCapsFont=SBL Hebrew, Script=Hebrew]{SBL Hebrew}#1}}

\newcommand{\tounfold}[1]{{\addfontfeature{Color=0000FFFF}\symbolglyph{\addfontfeature{Color=0000FFFF}⛭}~\BH{#1}~\symbolglyph{\addfontfeature{Color=0000FFFF}⛭}}}

\newcommand{\symbolglyph}[1]{{\fontspec{Symbola}#1}}
\newcommand{\symbolglyphalt}[1]{{\fontspec{DejaVu Serif}#1}}

\usepackage{newunicodechar}
\newunicodechar{ſ}{s}
\newunicodechar{ꝛ}{r}
