\section{Introduction}

\begin{paper}
	{\click} Hi, thank you all for coming here,

	{\click} In 1588 William Morgan translated the Bible into Welsh from the original Hebrew and Greek. What I’ll do in the next twenty minutes is to present you some of my findings regarding the means by which Morgan \hl{bridged the linguistic gap} between two languages which are quite different, Biblical Hebrew and Welsh, thus producing a natural, independent Welsh text.

	The domain of our examination here is the \hl{senses} (sight, hearing, touch, smell and taste), but the methodology can be applied to other domains as well. Due to fundamental structural differences in lexicon and idiomatics, it was the translator’s obligation to mediate and choose the most appropriate equivalents in the target language, on the basis of his reading and understanding of the original text.

	\begin{compactitem}
		\item {\click} \point{A} In some cases the choice is between, say, two main options and has to be made on every instance of the original phrase,
		\item \point{B and C} in others a single original phrase is always rendered by a certain phrase in the target language
			\begin{compactitem}
				\item (if in A we see a split, here we can see a merge of B and C, thus flattening lexical information rather than diversifying it in A),
			\end{compactitem}
		\item and in \point{D} some cases the division is between many, finely tuned options.
	\end{compactitem}
\end{paper}

\begin{hopoint}
	A → \{X, Y\}\hfill
	B → \{Z\}\hfill
	C → \{Z\}\hfill
	D → \{X, Z, U, V, W\}
\end{hopoint}

\begin{paper}
	{\click} The \hl{methodology} is quite straightforward: \point{1} using a Biblical concordance I collected verses containing occurances of relevant Hebrew lexemes and collocations and their Welsh translation. \point{2} Then I grouped these verses according to Morgan’s translation choices, \point{3} looking for patterns and generalisations, the overview of which is presented here.

	\begin{leftbar}
		{\click} The edition I’ve based upon as my \hl{corpus} is the original 1588 edition, from a contemporary facsimile, not the 1620 posthumous revision by Parry and Davies, which is misleadingly titled \C{Beibl William Morgan} nowadays and is the one most familiar to Welsh speakers. The reason is that the former shows a uniform, unedited idiolect and it’s relation to the Hebrew original is clearer and simpler.
	\end{leftbar}

	\begin{leftbar}
		{\click} A note concerning the extent of my examination: for the extremely common verbs \bh{rå̄ʾå̄} ‘to see’ and \bh{šå̄maʿ} ‘hear’, which have more than a thousand occurances each, I \hl{limited} myself to the Pentateuch for practical reasons; for all of the other lexemes and collocations I’ve used the whole Hebrew Bible.
	\end{leftbar}

	{\click} Today’s paper is an extension of the one I delivered at the \hl{previous Congress}, four years ago, which was focused on hearing. In the meantime between congresses I broadened the scope of research, including other senses. Today’s paper will give less focus on hearing and more on the other major sense, sight.
\end{paper}
