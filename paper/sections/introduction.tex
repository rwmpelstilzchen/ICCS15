\begin{paper}
	\section{Introduction}

	{\click} Hi,

	{\click} In the next twenty minutes I’ll present you some of my findings regarding the means by which William Morgan, who translated the Bible into Welsh, \hl{bridged the linguistic gap} between two languages which are quite different, both grammatically and lexically: Biblical Hebrew and Welsh. Thus he produced a natural Welsh text on the basis of his reading of the Hebrew original text.

	The domain of our examination here is the \hl{senses} (sight, hearing, touch, smell and taste), but the methodology can be applied to other semantic domains and, as we will see, there seem to be some general principles guiding Morgan practice of translation. As Biblical Hebrew and Welsh are structured differently, it was the translator’s obligation to mediate and choose the most appropriate equivalent in the target language, on the basis of his reading and understanding of the original text. So, in order for the translation to make \emph{sense}, the translator had to differentiate the \emph{senses} of the original phrases (in our case, the ones concerning the \emph{senses}) through the linguistic means and inventory of the target language… For more about translation as a decision process, see \cite{levy.j:1967:translation}.

	{\click} The edition I’ve based upon as my \hl{corpus} is the original 1588 edition, from a contamporary facsimile, not the 1620 posthumous revision by Parry and Davies, which is commonly titled William Morgan Bible nowadays and is the one most familiar to Welsh speakers. The reason is that the former shows a uniform, unedited idiolect and it’s relation to the Hebrew original is clearer and simpler.

	{\click} The \hl{methodology} is straightforward: using a Biblical concordance I collected the verses containing occurances of relevant Hebrew words and collocations and their Welsh translation. Then I grouped these verses according to Morgan’s translation choices, looking for patterns and generalisations, the overview of which is presented here.
	
	{\click} Note: for the extremely common verbs \bh{råʾå} ‘see’ and \bh{šåmaʿ} ‘hear’ I \hl{limited} myself to the Pentateuch for practical reasons; for all of the other word and collocations I’ve used the whole Hebrew Bible.

	{\click} Today’s paper is an extension of the one I delivered at the \hl{previous Congress}, four years ago, which was focused on hearing. In the meantime between congresses I broadened the scope of research, including other senses.
\end{paper}
