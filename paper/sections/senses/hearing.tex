\subsection{Hearing}

\subsubsectionoccurances{\bh{šåmaʿ}}{295}

\begin{paper}
	\tounfold{***}
\end{paper}



\paragraph{\C{clywed}}

\subparagraphoccurances{Content}{14}

\begin{paper}
	As with \C{gweled} translating \bh{råʾå} with a content complement, only \C{clywed} is selectable.
\end{paper}

\begin{example}{Gen.}{42}{2}{}{}
	\quoling
	{וַיֹּ֕אמֶר הִנֵּ֣ה \lhl{שָׁמַ֔עְתִּי} \hlA{כִּ֥י} יֶשׁ־שֶׁ֖בֶר בְּמִצְרָ֑יִם רְדוּ־שָׁ֙מָּה֙ וְשִׁבְרוּ־לָ֣נוּ מִשָּׁ֔ם וְנִחְיֶ֖ה וְלֹ֥א נָמֽוּת׃}
	{Dywedodd hefyd, wele \lhl{clywais} \hlA{fod} ŷd yn yꝛ Aipht, ewch i wared yno, a phꝛynnwch i ni oddi yno, fel y bôm fyw, ac na byddom feirw.}
	{way·yōmɛr hinnē \lhl{šåmaʿtī} \hlA{kī} yɛš·šɛḇɛr bə·miṣråyim rəḏū·šåmm·å wə·šiḇrū·l·ånū miš·šåm wə·niḥyɛ wə·lō nåmūṯ}
	{And he said, Behold, I have \lhl{heard} \hlA{that} there is corn in Egypt: get you down thither, and buy for us from thence; that we may live, and not die.}
\end{example}
\begin{paper}
	\explain In \exvref{Gen.}{42}{2}{} there is a \BHtr{kī} ‘that’ phrase.
\end{paper}

\begin{example}{Gen.}{41}{15}{1}{}
	\quoling
	{וַיֹּ֤אמֶר פַּרְעֹה֙ אֶל־יֹוסֵ֔ף חֲלֹ֣ום חָלַ֔מְתִּי וּפֹתֵ֖ר אֵ֣ין אֹתֹ֑ו וַאֲנִ֗י \lhl{שָׁמַ֤עְתִּי} עָלֶ֙יךָ֙ \hlA{לֵאמֹ֔ר} תִּשְׁמַ֥ע חֲלֹ֖ום לִפְתֹּ֥ר אֹתֹֽו׃}
	{A Pharao a ddywedodd wꝛth Joſeph, bꝛeuddwydiais freuddwyd, ac nid [oes] ai deonglo ef: ond myfi a \lhl{glywais} \hlA{ddywedyd} am danat ti, y gwꝛandewi freuddwyd iw ddeonglu.}
	{way·yōmɛr parʿō ʾɛl·yōsēp̄ ḥălōm ḥålamtī ū·p̄ōṯēr ʾēn ʾōṯ·ō wa·ʾănī \lhl{šåmaʿtī} ʿålɛḵå \hlA{lē·mōr} tišmaʿ ḥălōm li·p̄tōr ʾōṯ·ō}
	{And Pharaoh said unto Joseph, I have dreamed a dream, and \emph{there is} none that can interpret it: and I have \lhl{heard} say of thee, \hlA{\emph{that}} thou canst understand a dream to interpret it.}
\end{example}
\begin{paper}
	\explain And in \exvref{Gen.}{41}{15}{1} another form of content complement occurs.
\end{paper}



\paragraph{\C{gwrando}}

\subparagraphoccurances{\bh{šåmaʿ}~+ \textsc{preposition}}{84}

\begin{paper}
	When \bh{šåmaʿ} is followed by a prepositional complement, it has the sense of \hl{obeying} or \hl{accepting}, and as such is translated only by \C{gwrando} with \C{ar} ‘on’ complement.
\end{paper}

\begin{example}{Ex.}{4}{1}{}{}
	\quoling
	{וַיַּ֤עַן מֹשֶׁה֙ וַיֹּ֔אמֶר וְהֵן֙ לֹֽא־יַאֲמִ֣ינוּ לִ֔י וְלֹ֥א \lhl{יִשְׁמְע֖וּ} \hlA{בְּ}קֹלִ֑י כִּ֣י יֹֽאמְר֔וּ לֹֽא־נִרְאָ֥ה אֵלֶ֖יךָ יְהוָֽה׃}
	{Yna Moſes a attebodd, ac a ddywedodd, etto wele ni chꝛedant i mi ac ni \lhl{wꝛandawant} \hlA{ar} fy llais: onid dywedant nid ymgdangoſodd yꝛ Arglwydd i ti.}
	{way·yaʿan mōšɛ way·yōmɛr wə·hēn lō·yaʾămīnū l·ī wə·lō \lhl{yišməʿū} \hlA{bə·}qōlī kī yōmrū lō·nirʾå ʾēlɛḵå {\YHWH}}
	{And Moses answered and said, But, behold, they will not believe me, nor \lhl{hearken} \hlA{unto} my voice: for they will say, The {\LORD} hath not appeared unto thee.}
\end{example}
\begin{paper}
	\explain So in \exvref{2}{4}{1}{} Moses doesn’t fear the children of Israel will not be able to hear him, but he fears they will not \hl{follow}, will not \hl{accept} his orders.
\end{paper}



\subsubsection{\bh{hɛʾɛ̆zīn}}

\subsubsectionoccurances{\bh{hiṭṭå}~+ \bh{ʾōzɛn}}{26}

\begin{paper}
	There is an idiomatic collocation of \bh{hiṭṭå} ‘to incline’ and \bh{ʾōzɛn} ‘ear’. This collocation is translated literally mostly by \C{gogwydd} or \C{gostwng} with \C{clust} ‘ear’ as an object. These two lexemes have about the equal number of occurances, and I couldn’t find any clear-cut distinction in their use. \bh{hiṭṭå} is translated with \C{***gŵyro} once.

	Almost all occurances of \bh{hiṭṭå ʾōzɛn}, with a few exceptions from Psalms, collocate also with verbs of hearing and accepting: ‘incline your ear and hear (that is, accept)’. \bh{šåmaʿ} in this position is expectably translated by \C{gwrando}.
\end{paper}

\begin{example}{Isa.}{55}{3}{}{}
	\quoling
	{\lhl{הַטּ֤וּ} \hlA{אָזְנְכֶם֙} וּלְכ֣וּ אֵלַ֔י \hlB{שִׁמְע֖וּ} וּתְחִ֣י נַפְשְׁכֶ֑ם וְאֶכְרְתָ֤ה לָכֶם֙ בְּרִ֣ית עוֹלָ֔ם חַֽסְדֵ֥י דָוִ֖ד הַנֶּאֱמָנִֽים׃}
	{\lhl{Gogwyddwch} eich \hlA{cluſtiau}, a deuwch attaf, \hlB{gwꝛandewch} fel y byddo byw eich enaid, ac mi a wnâf gyfammod tragywyddawl â chwi [ſef] ſiccr dꝛugareddau Dafydd.}
	{\lhl{haṭṭū} \hlA{ʾåznəḵɛm} ū·lḵū ʾēla·y \hlB{šimʿū} ū·ṯḥī nap̄šəḵɛm wə·ʾɛḵrəṯå låḵɛm bərīṯ ʿōlåm ḥasḏē ḏåwīḏ han·nɛʾɛ̆månīm}
	{\lhl{Incline} your \hlA{ear}, and come unto me: \hlB{hear}, and your soul shall live; and I will make an everlasting covenant with you, \textit{even} the sure mercies of David.}
\end{example}

\begin{example}{Prov.}{5}{13}{}{}
	\quoling
	{וְֽלֹא־\hlB{שָׁ֭מַעְתִּי} בְּק֣וֹל מוֹרָ֑י וְ֝לִֽמְלַמְּדַ֗י לֹא־\lhl{הִטִּ֥יתִי} \hlA{אָזְנִֽי}׃}
	{Ac ni \hlB{wꝛandewais} ar lef fy athꝛawon, ac ni \lhl{oſtyngais} fyng-\hlA{hluſt} i’m diſcawdwŷꝛ}
	{wə·lō·\hlB{šåmaʿtī} bə·qōl mōråy wə·li·mlamməḏay lō·\lhl{hiṭṭīṯī} \hlA{ʾåznī}}
	{And have not \hlB{obeyed} the voice of my teachers, nor \lhl{inclined} mine \hlA{ear} to them that instructed me!}
\end{example}



\subsubsection{Minor verbs}

%\paragraph{Påṯaḥ}


\tounfold{מההרצאה הקודמת; לקצר מאוד}

% \subsection{Neutralising environments}
%
% \begin{paper}
% 	{\click} Now let's proceed to isolating \hl{neutralising syntactic environments}, where only one of the two is selectable. \hl{Three} evident environments are the following:
% 	\begin{itemize}
% 		\item {\click} ‘\shama~+ \textsc{preposition}’, which is translated by \C{gwꝛando}~+ the Welsh preposition \C{ar} ‘on’. \optout{84 examples.}
% 		\item {\click} ‘\shama~+ \textsc{content}’, which is translated with \C{clywed}. \optout{14 examples.}
% 		\item and {\click} ‘\shama~+ \textsc{object}+\textsc{participle}’, which is translated with \C{clywed} as well. \optout{5 examples.}
% 	\end{itemize}
% \end{paper}
%
% \subsubsection{\shama~+ \textsc{preposition}}
%
% \begin{paper}
% 	{\click} {\click} {\shama} can be followed by three Hebrew prepositions \point{use fingers}: \BHtr{bə-}, \BHtr{ʾɛl} and \BHtr{lə-}. The difference between these three after \shama, whatever it is, is \hl{flattened} in the Welsh translation: they are all converted to a single Welsh preposition \C{ar} ‘on’. The opposite direction generally holds true as well, meaning that ‘\C{gwꝛando}~+ \C{ar}’ translates ‘\shama~+ \textsc{preposition}’ (other cases cases of \C{gwꝛando} have \hl{direct} objects).
%
% %~% There are some generalisations regarding the complements of the prepositions, which is an internal issue of Hebrew syntax. This issue is transferred to the Welsh quite literally, for example: all 36 cases of ‘\shama~+ \BHtr{bə-}’ are followed by a grammaticalised intermediating element \BHtr{qōl} ‘voice’, which is translated by \C{llais} with several semantically motivated exceptions (\C{llef} ‘cry’, \C{ſwn} ‘sound’), and almost none of the 38 cases of ‘\shama~+ \BHtr{ʾɛl}’ is followed by \BHtr{qōl}, and thus translated. This is an example for how translated texts~— especially of sacred texts, which tend to be literal~— have different syntax from native texts. I don't try to offer generalisations for ‘the Welsh Syntax (or Lexicon)’; each (kind of) text has its own system.
% As for the \hl{complements} of the prepositions, the Welsh text \hl{follows} the Hebrew one quite \hl{literally}. For example, it translates the grammaticalised intermediating element \BHtr{qōl} ‘voice’ by \C{llais}, with several semantically motivated exceptions (like \C{llef} ‘cry’, and \C{ſwn} ‘sound’). This is an example for how translated texts~— especially sacred texts, which tend to be translated literally~— have \hl{different syntax} from native texts. I don't try to offer generalisations for ‘the Welsh Syntax (or Lexicon)’; each kind of text, \hl{each text}, has its \hl{own system}.
%
% A \hl{common property} of the examples of this kind \point{point at the highlighted text} is that their semantics is of hearing as \hl{obeying} or \hl{accepting}. We will return to this later.
%
% Let's have a look at \hl{three examples}: for \point{use fingers} \BHtr{bə-} (\vref{2}{4}{1}{}), \BHtr{ʾɛl-} (\vref{2}{6}{9}{}) and \BHtr{lə-} (\vref{1}{3}{17}{}).
%
% (The KJV text is, of course, given only as an aid)
% \end{paper}
%
% \begin{example}{2}{4}{1}{}{}
% 	\quoling
% 	{Yna Moſes a attebodd, ac a ddywedodd, etto wele ni chꝛedant i mi ac ni \highlight{wꝛandawant} \highlight{ar} fy llais: onid dywedant nid ymgdangoſodd yꝛ Arglwydd i ti.}
% 	{וַיַּ֤עַן מֹשֶׁה֙ וַיֹּ֔אמֶר וְהֵן֙ לֹֽא־יַאֲמִ֣ינוּ לִ֔י וְלֹ֥א \BHhighlight{יִשְׁמְע֖וּ} \BHhighlight{בְּ}קֹלִ֑י כִּ֣י יֹֽאמְר֔וּ לֹֽא־נִרְאָ֥ה אֵלֶ֖יךָ יְהוָֽה׃}
% 	{way-yaʿan mōšɛ way-yōmɛr wə-hēn lō-yaʾămīnū l-ī wə-lō \highlight{yišməʿū} \highlight{bə-}qōlī kī yōmrū lō-nirʾå ʾēlɛḵå {\YHWH}}
% 	{And Moses answered and said, But, behold, they will not believe me, nor \highlight{hearken} {unto} my voice: for they will say, The {\LORD} hath not appeared unto thee.}
% \end{example}
% \begin{paper}
% 	\explain Moses says the children of Israel will not believe him when he will tell them God will bring them forth out of Egypt and will not \hl{follow}, \hl{accept}, his orders (lit.\ in Heb.~‘\textit{they will not hear}’) \point{point accordingly}.
% \end{paper}
%
% \begin{paper}
% 	And…
% \end{paper}
% \begin{example}{2}{6}{9}{}{}
% 	\quoling
% 	{A Moſes a lefarodd felly wꝛth feibion Iſrael: ond ni \highlight{wꝛandawſant} \highlight{ar} Moſes, gan gyfyngdꝛa yſpꝛyd, a chan y gaethiwed galed.}
% 	{וַיְדַבֵּ֥ר מֹשֶׁ֛ה כֵּ֖ן אֶל־בְּנֵ֣י יִשְׂרָאֵ֑ל וְלֹ֤א \BHhighlight{שָֽׁמְעוּ֙} \BHhighlight{אֶל־}מֹשֶׁ֔ה מִקֹּ֣צֶר ר֔וּחַ וּמֵעֲבֹדָ֖ה קָשָֽׁה׃}
% 	{way-ḏabbēr mōšɛ kēn ʾɛl-bənē yiśråʾēl wə-lō \highlight{šåmʿū} \highlight{ʾɛl-}mōšɛ miq-qōṣɛr rūaḥ ū-mē-ʿăḇōḏå qåšå}
% 	{And Moses spake so unto the children of Israel: but they \highlight{hearkened} not {unto} Moses for anguish of spirit, and for cruel bondage.}
% \end{example}
% \begin{paper}
% 	\explain They indeed do not \hl{follow} him. Notice that we are not talking here about actual, sensory, hearing: they did perceive his voice, the acoustic waves, they just didn't follow.
% \end{paper}
%
% \begin{example}{1}{3}{17}{}{}
% 	\quoling
% 	{Hefyd wꝛth Adda y dywedodd, am \highlight{wꝛando} o honot \highlight{ar} lais dy wꝛaig, a bwytta o'ꝛ pꝛenn am yꝛ hwn y goꝛchymynnaſwn i ti gan ddywedyd, na fwytta o honaw: melldigedic [fydd] y ddaiar o'th achos di, a thꝛwy lafur y bwyttei o honi holl ddyddiau dy enioes.}
% 	{וּלְאָדָ֣ם אָמַ֗ר כִּֽי־\BHhighlight{שָׁמַעְתָּ֮} \BHhighlight{לְ}קֹ֣ול אִשְׁתֶּךָ֒ וַתֹּ֙אכַל֙ מִן־הָעֵ֔ץ אֲשֶׁ֤ר צִוִּיתִ֙יךָ֙ לֵאמֹ֔ר לֹ֥א תֹאכַ֖ל מִמֶּ֑נּוּ אֲרוּרָ֤ה הָֽאֲדָמָה֙ בַּֽעֲבוּרֶ֔ךָ בְּעִצָּבֹון֙ תֹּֽאכֲלֶ֔נָּה כֹּ֖ל יְמֵ֥י חַיֶּֽיךָ׃}
% 	{ū-l-ʾåḏåm ʾåmar kī-\highlight{šåmaʿtå} \highlight{lə-}qōl ʾištɛḵå wat-tōḵal min-hå-ʿēṣ ʾăšɛr ṣiwwīṯīḵå lē-mōr lō ṯōḵal mimm-ɛnnū ʾărūrå hå-ʾăḏåmå ba-ʿăḇūrɛḵå bə-ʿiṣṣåḇōn tōḵălɛnnå kōl yəmē ḥayyɛḵå}
% 	{And unto Adam he said, Because thou hast \highlight{hearkened} {unto} the voice of thy wife, and hast eaten of the tree, of which I commanded thee, saying, Thou shalt not eat of it: cursed is the ground for thy sake; in sorrow shalt thou eat of it all the days of thy life;}
% \end{example}
% \begin{paper}
% 	\explain Adam not only sensorially heard Eve, but also \hl{followed} her and ate of the tree.
% \end{paper}
%
%
%
% \subsubsection{\shama~+ \textsc{content}}
%
% \begin{paper}
% 	{\click} {\click} \hl{Two} Hebrew syntactical patterns are used for denoting \hl{hearing of content} in the corpus: {\shama} with a \BHtr{kī} ‘that’ phrase, which introduces indirect speech, and {\shama} with a form of \BHtr{ʾåmar} ‘say’. All these cases, of hearing content, are translated by \C{clywed}, followed by a variety of Welsh structures.
%
% 	In
% \end{paper}
%
% \begin{example}{1}{42}{2}{}{}
% 	\quoling
% 	{Dywedodd hefyd, wele \highlight{clywais} fod ŷd yn yꝛ Aipht, ewch i wared yno, a phꝛynnwch i ni oddi yno, fel y bôm fyw, ac na byddom feirw.}
% 	{וַיֹּ֕אמֶר הִנֵּ֣ה \BHhighlight{שָׁמַ֔עְתִּי} \BHhighlight{כִּ֥י} יֶשׁ־שֶׁ֖בֶר בְּמִצְרָ֑יִם רְדוּ־שָׁ֙מָּה֙ וְשִׁבְרוּ־לָ֣נוּ מִשָּׁ֔ם וְנִחְיֶ֖ה וְלֹ֥א נָמֽוּת׃}
% 	{way-yōmɛr hinnē \highlight{šåmaʿtī} \highlight{kī} yɛš-šɛḇɛr bə-miṣråyim rəḏū-šåmm-å wə-šiḇrū-l-ånū miš-šåm wə-niḥyɛ wə-lō nåmūṯ}
% 	{And he said, Behold, I have \highlight{heard} that there is corn in Egypt: get you down thither, and buy for us from thence; that we may live, and not die.}
% \end{example}
% \begin{paper}
% 	\explain there is a \BHtr{kī} phrase.
% \end{paper}
%
% \begin{paper}
% 	And in
% \end{paper}
% \begin{example}{1}{41}{15}{1}{}
% 	\quoling
% 	{A Pharao a ddywedodd wꝛth Joſeph, bꝛeuddwydiais freuddwyd, ac nid [oes] ai deonglo ef: ond myfi a \highlight{glywais} \highlight{ddywedyd} am danat ti, y gwꝛandewi freuddwyd iw ddeonglu.}
% 	{וַיֹּ֤אמֶר פַּרְעֹה֙ אֶל־יֹוסֵ֔ף חֲלֹ֣ום חָלַ֔מְתִּי וּפֹתֵ֖ר אֵ֣ין אֹתֹ֑ו וַאֲנִ֗י \BHhighlight{שָׁמַ֤עְתִּי} עָלֶ֙יךָ֙ \BHhighlight{לֵאמֹ֔ר} תִּשְׁמַ֥ע חֲלֹ֖ום לִפְתֹּ֥ר אֹתֹֽו׃}
% 	{way-yōmɛr parʿō ʾɛl-yōsēp̄ ḥălōm ḥålamtī ū-p̄ōṯēr ʾēn ʾōṯ-ō wa-ʾănī \highlight{šåmaʿtī} ʿålɛḵå \highlight{lē-mōr} tišmaʿ ḥălōm li-p̄tōr ʾōṯ-ō}
% 	{And Pharaoh said unto Joseph, I have dreamed a dream, and \textit{there is} none that can interpret it: and I have \highlight{heard} say of thee, \textit{that} thou canst understand a dream to interpret it.}
% \end{example}
% \begin{paper}
% 	\explain a form of \BHtr{ʾåmar} occurs.
% \end{paper}
%
% \begin{paper}
% 	The way Morgan translated these complex syntactic complements is fascinating, but out of our scope here. The relevant fact is that hearing of \hl{content} is translated using \C{clywed}.
% \end{paper}
%
%
%
% \subsubsection{\shama~+ \textsc{object}+\textsc{participle}}
%
% \begin{paper}
% 	{\click} The Hebrew structure of ‘\shama~+ \textsc{object}+\textsc{participle}’ is our third neutralising environment. {\click} It is translated by its \hl{closest structural (and semantic) equivalent}, ‘\C{clywed}~+ \C{\textup{\textsc{object}}}+[\C{yn}+\C{\textup{\textsc{infinitive}}}]’, with the Welsh ‘\C{yn}+\textsc{infinitive}’ paralleled to the Hebrew participle.
% \end{paper}
%
% \begin{paper}
% 	Here it is with the co-text of this beautiful linguistic parallelism (they all fit!).
% \end{paper}
% \begin{example}{4}{11}{10}{}{}
% 	\quoling
% 	{\highlight{A chlybu} Moſes y bobl yn wylo trwy eu tylwythau, pob vn yn nrws ei babell: ac enynnodd dig yꝛ Arglwydd yn fawꝛ, a dꝛwg oedd gan Moſes.}
% 	{\BHhighlight{וַיִּשְׁמַ֨ע} מֹשֶׁ֜ה אֶת־הָעָ֗ם בֹּכֶה֙ לְמִשְׁפְּחֹתָ֔יו אִ֖ישׁ לְפֶ֣תַח אָהֳל֑וֹ וַיִּֽחַר־אַ֤ף יְהוָה֙ מְאֹ֔ד וּבְעֵינֵ֥י מֹשֶׁ֖ה רָֽע׃}
% 	{\highlight{way-yišmaʿ} mōšɛ ʾɛṯ-hå-ʿåm bōḵɛ lə-mišpəḥōṯåw ʾīš lə-p̄ɛṯaḥ ʾåhålō way-yiḥar-ʾap̄ {\YHWH} məʾōḏ ū-ḇ-ʿēnē mōšɛ råʿ}
% 	{Then Moses \highlight{heard} the people weep throughout their families, every man in the door of his tent: and the anger of the {\LORD} was kindled greatly; Moses also was displeased.}
% \end{example}
%
%
%
% \subsection{\C{clywed}:\C{gwꝛando}}
%
% \begin{paper}
% 	{\click} Now, after we've \ruby{sifted}{\IPA{/sɪft/}} the cases where a true opposition exists from those in which only one of the two is selectable, we are getting to the core of the paper: the opposition between \C{clywed} and \C{gwꝛando}.
%
% 	{\click} In the nutshell, the opposition is this \point{point while speaking}:
% 	\begin{itemize}
% 		\item \C{clywed} is ‘hearing’ in the simple, semantically unmarked, sense of \hl{sensory perception of sound}.
% 		\item \C{gwꝛando}, on the other hand, is ‘hearing’ in \hl{any other sense}, involving actual sensory hearing or not. These include obeying, accepting, following, judging, interpreting, etc.
% 	\end{itemize}
%
% 	Keep this table in mind.
% \end{paper}
%
% \begin{tabular}{l|ll}
% 	& {sensory} & {\ruby{additional}{(non-sensory)} meaning}\\ %& \small{markedness}\\
% 	\hline
% 	\hl{\C{clywed}}  & $+$   & $-$\\% & $-$\\
% 	\hl{\C{gwꝛando}} & $\pm$ & $+$% & $+$
% \end{tabular}
%
% \begin{paper}
% 	This is no surprise for any speaker of Welsh, but what I find intriguing is those many cases in which Morgan actually had to add \hl{new information}, \hl{new interpretation}, by choosing one over the other. I hope to show this in the next few minutes.
% \end{paper}
%
%
%
% \subsubsection{Lists of senses and abilities}
% \label{sec-senses}
%
% \begin{paper}
% 	We will begin with the obvious and proceed to the more complex cases.
%
% 	{\click} Lists of senses are \ruby{\emph{par excellence}}{\IPA{/ˌpɑːr ˌɛksəˈlɑːns/}} examples for the use of {\shama} in the pure \hl{sensory} meaning, by definition. {\shama} in these examples is, as expected, translated by \C{clywed}. There are several such lists, of senses and abilities, in the Hebrew Bible.
% \end{paper}
%
% \begin{example}{5}{4}{28}{}{}
% 	\quoling
% 	{Ac yno y gwaſanaethwch dduwiau [o] waith dwylo dŷn, [ſef] pꝛen, a maen, y rhai ni welant, ac ni \highlight{chlywant}, ac ni fwyttânt, ac ni aroglant.}
% 	{וַעֲבַדְתֶּם־שָׁ֣ם אֱלֹהִ֔ים מַעֲשֵׂ֖ה יְדֵ֣י אָדָ֑ם עֵ֣ץ וָאֶ֔בֶן אֲשֶׁ֤ר לֹֽא־יִרְאוּן֙ וְלֹ֣א \BHhighlight{יִשְׁמְע֔וּן} וְלֹ֥א יֹֽאכְל֖וּן וְלֹ֥א יְרִיחֻֽן׃}
% 	{wa-ʿăḇaḏtɛm-šåm ʾɛ̆lōhīm maʿăśē yəḏē ʾåḏåm ʿēṣ wå-ʾɛḇɛn ʾăšɛr lō-yirʾūn wə-lō \highlight{yišməʿūn} wə-lō yōḵlūn wə-lō yərīḥun}
% 	{And there ye shall serve gods, the work of men's hands, wood and stone, which neither see, nor \highlight{hear}, nor eat, nor smell.}
% \end{example}
% \begin{paper}
% 	\point{point in terzas:} they neither see, nor hear, nor eat, nor smell.
% \end{paper}
%
% \begin{example}{Ps}{115}{4–7}{}{}
% 	\quoling
% 	{\textsuperscript{4}~Eu delwau hwy [ydynt] o aur, ac arian, [ſef] o waith dynnion.
% 	\textsuperscript{5}~Genau [ſydd] iddynt, ac ni lefarant, llygaid [ſydd] ganddynt, ac ni welant.
% 	\textsuperscript{6}~[Y mae] cluſtiau iddynt, ac ni \highlight{chlywant}, ffroenau [ſydd] ganddynt, ac ni aroglant.
% 	\textsuperscript{7}~Dwylo [ſydd] iddynt, ac ni theimlant: traed [ſy] iddynt, ac ni cherddant: ni leiſiant [ychwaith] ai gwddf.}
% 	{\textsuperscript{4}~עֲצַבֵּיהֶם כֶּ֣סֶף וְזָהָ֑ב מַ֝עֲשֵׂ֗ה יְדֵ֣י אָדָֽם׃
% 	\textsuperscript{5}~פֶּֽה־לָ֭הֶם וְלֹ֣א יְדַבֵּ֑רוּ עֵינַ֥יִם לָ֝הֶ֗ם וְלֹ֣א יִרְאֽוּ׃
% 	\textsuperscript{6}~אָזְנַ֣יִם לָ֭הֶם וְלֹ֣א \BHhighlight{יִשְׁמָ֑עוּ} אַ֥ף לָ֝הֶ֗ם וְלֹ֣א יְרִיחֽוּן׃
% 	\textsuperscript{7}~יְדֵיהֶ֤ם ׀ וְלֹ֬א יְמִישׁ֗וּן רַ֭גְלֵיהֶם וְלֹ֣א יְהַלֵּ֑כוּ לֹֽא־יֶ֝הְגּ֗וּ בִּגְרוֹנָֽם׃}
% 	{\textsuperscript{4}~ʿăṣabbēhɛm kɛsɛp̄ wə-zåhåḇ maʿăśē yəḏē ʾåḏåm
% 	\textsuperscript{5}~pɛ-l-åhɛm wə-lō yəḏabbērū ʿēnayim l-åhɛm wə-lō yirʾū
% 	\textsuperscript{6}~ʾåznayim l-åhɛm wə-lō \highlight{yišmåʿū} ʾap̄ l-åhɛm wə-lō yərīḥūn
% 	\textsuperscript{7}~yəḏēhɛm wə-lō yəmīšūn raḡlēhɛm wə-lō yəhallēḵū lō-yɛhgū bi-ḡrōnåm}
% 	{\textsuperscript{4}~Their idols \emph{are} silver and gold, the work of men's hands.
% 	\textsuperscript{5}~They have mouths, but they speak not: eyes have they, but they see not:
% 	\textsuperscript{6}~They have ears, but they \highlight{hear} not: noses have they, but they smell not:
% 	\textsuperscript{7}~They have hands, but they handle not: feet have they, but they walk not: neither speak they through their throat.}
% \end{example}
% \begin{paper}
% 	This is a beautiful example for mental and other abilities.
% \end{paper}
%
% \subsubsection{Judgement}
%
% \begin{paper}
% 	{\click} {\shama} in the sense of \hl{judging}, of deciding one way or the other, or of asking God for judgement, is translated by \C{gwꝛando}. Ex.~\vref{5}{1}{16}{} is such an example, out of five.
% \end{paper}
%
% \begin{example}{5}{1}{16}{}{}
% 	\quoling
% 	{A'r amſer hwnnw y goꝛchymynnais i'ch barnwŷꝛ chwi gan ddywedyd: \highlight{gwꝛandewch} [ddadleuon] rhwng eich bꝛodyꝛ, a bernwch gyfiawnder rhwng gŵꝛ ai frawd, ac ai eſtron hefyd.}
% 	{וָאֲצַוֶּה֙ אֶת־שֹׁ֣פְטֵיכֶ֔ם בָּעֵ֥ת הַהִ֖וא לֵאמֹ֑ר \BHhighlight{שָׁמֹ֤עַ} בֵּין־אֲחֵיכֶם֙ וּשְׁפַטְתֶּ֣ם צֶ֔דֶק בֵּֽין־אִ֥ישׁ וּבֵין־אָחִ֖יו וּבֵ֥ין גֵּרֽוֹ׃}
% 	{wå-ʾăṣawwɛ ʾɛṯ-šōp̄ṭēḵɛm b-å-ʿēṯ ha-hi lē-mōr \highlight{šåmōaʿ} bēn-ʾăḥēḵɛm ū-šp̄aṭtɛm ṣɛḏɛq bēn-ʾīš ū-ḇēn-ʾåḥīw ū-ḇēn gērō}
% 	{And I charged your judges at that time, saying, \highlight{Hear} \emph{the causes} between your brethren, and judge righteously between \emph{every} man and his brother, and the stranger \emph{that is} with him.}
% \end{example}
%
%
%
% \subsubsection{Vows}
%
% \begin{paper}
% 	{\click} Chapter 30 of the book of Numbers deals with making women's \hl{vows} void. We will not enter the religious legal details here, but the relevant fact is that whether her father or husband actually \hl{hears} the vow being pronounced is crucial to the validity of making it void. \C{clywed} is being used. There are 9 examples in that chapter.
% \end{paper}
%
% \begin{example}{4}{30}{13}{}{W12}
% 	\quoling
% 	{Ond os ei gŵꝛ gan ddiddymmu ai diddymma hwynt y dydd y \highlight{clywo}, ni ſaif dim a ddaeth allan oi gwefuſau, oi haddunedau, ac o rwymedigaeth ei henaid, ei gŵꝛ ai diddymmodd hwynt, ar Arglwydd a fadde iddi.}
% 	{וְאִם־הָפֵר֩ יָפֵ֨ר אֹתָ֥ם ׀ אִישָׁהּ֮ בְּי֣וֹם \BHhighlight{שָׁמְעוֹ֒} כָּל־מוֹצָ֨א שְׂפָתֶ֧יהָ לִנְדָרֶ֛יהָ וּלְאִסַּ֥ר נַפְשָׁ֖הּ לֹ֣א יָק֑וּם אִישָׁ֣הּ הֲפֵרָ֔ם וַיהוָ֖ה יִֽסְלַֽח־לָֽהּ׃}
% 	{wə-ʾim-håp̄ēr yåp̄ēr ʾōṯ-åm ʾīšåh bə-yōm \highlight{šåmʿō} kål-mōṣå śəp̄åṯɛhå li-nḏårɛhå ū-l-ʾissar nap̄šåh lō yåqūm ʾīšåh hăp̄ēråm wa-{\YHWH} yislaḥ-l-åh}
% 	{But if her husband hath utterly made them void on the day he \highlight{heard} \textit{them}; \textit{then} whatsoever proceeded out of her lips concerning her vows, or concerning the bond of her soul, shall not stand: her husband hath made them void; and the {\LORD} shall forgive her.}
% \end{example}
%
%
%
% \subsubsection{Interpretation of dreams}
%
% \begin{paper}
% 	{\click} Hearing others' dreams in the sense of interpreting them is translated by \C{gwꝛando}. As you can see in ex.~\vref{1}{41}{15}{2}:
% \end{paper}
%
% \begin{example}{1}{41}{15}{2}{}
% 	\quoling
% 	{A Pharao a ddywedodd wꝛth Joſeph, bꝛeuddwydiais freuddwyd, ac nid [oes] ai deonglo ef: ond myfi a glywais ddywedyd am danat ti, y \Chl{gwꝛandewi} freuddwyd iw ddeonglu.}
% 	{וַיֹּ֤אמֶר פַּרְעֹה֙ אֶל־יֹוסֵ֔ף חֲלֹ֣ום חָלַ֔מְתִּי וּפֹתֵ֖ר אֵ֣ין אֹתֹ֑ו וַאֲנִ֗י שָׁמַ֤עְתִּי עָלֶ֙יךָ֙ לֵאמֹ֔ר \BHhl{תִּשְׁמַ֥ע} חֲלֹ֖ום לִפְתֹּ֥ר אֹתֹֽו׃}
% 	{way-yōmɛr parʿō ʾɛl-yōsēp̄ ḥălōm ḥålamtī ū-p̄ōṯēr ʾēn ʾōṯ-ō wa-ʾănī šåmaʿtī ʿålɛḵå lē-mōr \highlight{tišmaʿ} ḥălōm li-p̄tōr ʾōṯ-ō}
% 	{And Pharaoh said unto Joseph, I have dreamed a dream, and \textit{there is} none that can interpret it: and I have heard say of thee, \textit{that} thou canst \highlight{understand} [lit.\ in Heb.~\textit{hear}] a dream to interpret it.}
% \end{example}
%
% \begin{paper}
% 	{\click} Now Morgan reads the Hebrew text in the same manner when Joseph asks his brothers to \hl{hear} his dream (according to Morgan, to \hl{interpret} his dream):
% \end{paper}
%
% \begin{example}{1}{37}{6}{}{}
% 	\quoling
% 	{O blegit dywedaſe wꝛthynt, \Chl{gwꝛandewch} atolwg y bꝛeuddwyd hwn, yꝛ hwn a frenddwydiais.}
% 	{וַיֹּ֖אמֶר אֲלֵיהֶ֑ם \BHhl{שִׁמְעוּ}־נָ֕א הַחֲלֹ֥ום הַזֶּ֖ה אֲשֶׁ֥ר חָלָֽמְתִּי׃}
% 	{way-yōmɛr ʾălēhɛm \highlight{šimʿū}-nå ha-ḥălōm haz-zɛ ʾăšɛr ḥålåmtī}
% 	{And he said unto them, \highlight{Hear}, I pray you, this dream which I have dreamed:}
% \end{example}
%
% \begin{paper}
% 	{\click} In comparison, when what's being heard is the dream's \hl{content} (the \hl{narrative}) or an \hl{interpretation} thereof, Morgan uses \C{clywed}:
% \end{paper}
%
% \begin{example}{Judges}{7}{15}{}{}
% 	\quoling
% 	{Pan \Chl{glybu} Gedeon adꝛoddiad y bꝛeuddwyd ai ddirnad, yna efe a addolodd: ac a ddychwelodd i werſſyll Iſrael, ac a ddywedodd, cyfodwch, canys rhoddodd yꝛ Arglwydd werſſyll y Madianiaid yn eich llaw chwi.}
% 	{וַיְהִי֩ \BHhl{כִשְׁמֹ֨עַ} גִּדְע֜וֹן אֶת־מִסְפַּ֧ר הַחֲל֛וֹם וְאֶת־שִׁבְר֖וֹ וַיִּשְׁתָּ֑חוּ וַיָּ֙שָׁב֙ אֶל־מַחֲנֵ֣ה יִשְׂרָאֵ֔ל וַיֹּ֣אמֶר ק֔וּמוּ כִּֽי־נָתַ֧ן יְהוָ֛ה בְּיֶדְכֶ֖ם אֶת־מַחֲנֵ֥ה מִדְיָֽן׃}
% 	{way-hī \highlight{ḵi-šmōaʿ} giḏʿōn ʾɛṯ-mispar ha-ḥălōm wə-ʾɛṯ-šiḇrō way-yištåḥū way-yåšåḇ ʾɛl-maḥănē yiśråʾēl way-yōmɛr qūmū kī-nåṯan {\YHWH} bə-yɛḏḵɛm ʾɛṯ-maḥănē miḏyån}
% 	{And it was \textit{so}, when Gideon \highlight{heard} the telling of the dream, and the interpretation thereof, that he worshipped, and returned into the host of Israel, and said, Arise; for the {\LORD} hath delivered into your hand the host of Midian.}
% \end{example}
%
% \begin{paper}
% 	So, hearing as \hl{interpreting} is translated with \C{gwꝛando}; hearing one telling you the content or an interpretaion of a dream is translated with \C{clywed}.
% \end{paper}
%
% \subsubsection{\C{Ni a wnawn, ac a wꝛandawn}}
%
% \begin{paper}
% 	{\click} Exodus~24:7 is one of the most well-known verses in the Jewish tradition. This verse has been given many \ruby{exegeses}{\IPA{/ɛksɪˈdʒiːsEs/}} and interpretations, most of them claiming that the fact that \BHtr{naʿăśɛ} (‘we will do’, \C{gwnawn}) precedes \BHtr{nišmåʿ} (‘we will hear’, \C{gwꝛandawn}) is a token for the people of Israel's willingness to accept the Torah: they say they will \hl{do}, even before \hl{hearing} what to do.
%
% 	Now, I do not know whether Morgan was familiar with the Jewish Biblical \ruby{hermeneutics}{\IPA{/hɜːmɪˈnjuːtɪks/}}, I guess not, but anyway his interpretation of this verse is different. He does not use \C{clywed} here, but \C{gwꝛando}. This means he considers \BHtr{naʿăśɛ} and \BHtr{nišmåʿ} as parallel, that is, being two members of a Biblical paralellism, as two ways for saying basically the same thing: we will obey.
%
% 	(The \ruby{Vulgate}{\IPA{/ˈvʌlgət/}}, by the way, translates ‘\textit{faciemus et erimus oboedientes}’ and the English translations follow.)
% \end{paper}
%
% \begin{example}{2}{24}{7}{}{}
% 	\quoling
% 	{Ac efe a gymmerth lyfr y cyfammod, ac ai darllenodd lle y clywe y bobl, a dywedaſant ni a wnawn, ac a \Chl{wꝛandawn} yꝛ hyn oll a lefarodd yꝛ Arglwydd.}
% 	{וַיִּקַּח֙ סֵ֣פֶר הַבְּרִ֔ית וַיִּקְרָ֖א בְּאָזְנֵ֣י הָעָ֑ם וַיֹּ֣אמְר֔וּ כֹּ֛ל אֲשֶׁר־דִּבֶּ֥ר יְהוָ֖ה נַעֲשֶׂ֥ה \BHhl{וְנִשְׁמָֽע}׃}
% 	{way-yiqqaḥ sēp̄ɛr hab-bərīṯ way-yiqrå bə-ʾåznē hå-ʿåm way-yōmrū kōl ʾăšɛr-dibbɛr {\YHWH} naʿăśɛ \highlight{wə-nišmåʿ}}
% 	{And he took the book of the covenant, and read in the audience of the people: and they said, All that the {\LORD} hath said will we do, and be \highlight{obedient} [lit.\ in Heb.~\textit{we will hear}].}
% \end{example}
%
% \begin{paper}
% 	This kind of matching \hl{hearing} with \hl{doing} (and \hl{keeping}) in the way of parallelism is quite common in the Pentateuch: there are over 20 examples of this kind, matching \C{gwꝛando} with \C{gwneuthur} (‘to do’) and with, commonly, with \C{cadw} (‘to keep’).
%
% 	{\click} One such example, which exemplifies the opposition between \C{clywed} and \C{gwꝛando} very clearly, and with which we will finish, is \vref{5}{5}{24}{1}:
% \end{paper}
%
% \begin{example}{5}{5}{24}{1}{H27 W27}
% 	\quoling
% 	{Neſſa di a \Chl{chlyw} 'r hyn oll a ddywed yꝛ Arglwydd ein Duw, a llefara di wꝛthym ni yꝛ hyn oll a lefaro 'r Arglwydd ein Duw wꝛthit ti: ac nyni a \highlight{wꝛandawn}, ac a wnawn [hynny.]}
% 	{קְרַ֤ב אַתָּה֙ \BHhl{וּֽשֲׁמָ֔ע} אֵ֛ת כָּל־אֲשֶׁ֥ר יֹאמַ֖ר יְהוָ֣ה אֱלֹהֵ֑ינוּ וְאַ֣תְּ ׀ תְּדַבֵּ֣ר אֵלֵ֗ינוּ אֵת֩ כָּל־אֲשֶׁ֨ר יְדַבֵּ֜ר יְהוָ֧ה אֱלֹהֵ֛ינוּ אֵלֶ֖יךָ \BHhighlight{וְשָׁמַ֥עְנוּ} וְעָשִֽׂינוּ׃}
% 	{qəraḇ ʾattå \highlight{ū-šămåʿ} ʾēṯ kål-ʾăšɛr yōmar {\YHWH} ʾɛ̆lōhēnū wə-ʾat təḏabbēr ʾēlēnū ʾēṯ kål-ʾăšɛr yəḏabbēr {\YHWH} ʾɛ̆lōhēnū ʾēlɛḵå \highlight{wə-šåmaʿnū} wə-ʿåśīnū}
% 	{Go thou near, and \highlight{hear} all that the {\LORD} our God shall say: and speak thou unto us all that the {\LORD} our God shall speak unto thee; and we will \highlight{hear} \textit{it}, and do \textit{it}.}
% \end{example}
% \begin{paper}
% 	\explain In this example there are two occurrences of {\shama} which are translated \hl{differentially} into Welsh: the first one with \C{clywed} and the second one with \C{gwꝛando}. The context of this verse is that the people are afraid of God's voice at Mount Sinai, and ask Moses to hear God's words and transmit them to them. Moses will \hl{hear} \point{point at \C{chlyw} and \BHtr{ū-šămåʿ}}, will receive, God's words and \hl{speak} \point{point at \C{llefara} and \BHtr{təḏabbēr}} them unto the people. This already means they will hear Moses' voice sensorially; in saying ‘we will hear’ \point{point at \C{wꝛandawn} and \BHtr{wə-šåmaʿnū}} they do \hl{not} mean they will sensorially hear Moses, but that they will \hl{accept} God's commandments, in other words: ‘we will do’ \point{point at \C{wnawn} and \BHtr{wə-ʿåśīnū}}.
% \end{paper}
%
% \begin{paper}
% 	There are many more interesting examples and many more relevant facts and findings, but, well, \hl{time is short}.
% \end{paper}
