\subsection{Hearing}

\subsubsectiontext{\bh{šå̄maʿ}}{295 occurances in the Pentateuch}

\begin{paper}
	{\click} \bh{šå̄maʿ} ‘to hear’ is the auditory equivalent of the visual \bh{rå̄ʾå̄}, serving as the principal verb of hearing.

	As I’ve said in the overview, \C{clywed} is used for simple sensory perception and content, while \C{gwrando} is used in the sense of obeying, accepting, following, judging, interpreting, etc.
\end{paper}



\paragraph{\C{clywed}}

\subparagraphoccurances{Content}{14}

\begin{paper}
	As with \C{gweled} translating \bh{rå̄ʾå̄} with a content complement, only \C{clywed} is selectable.
\end{paper}

\begin{example}{Gen.}{42}{2}{}{}
	\quoling
	{וַיֹּ֕אמֶר הִנֵּ֣ה \lhl{שָׁמַ֔עְתִּי} \hlA{כִּ֥י} יֶשׁ־שֶׁ֖בֶר בְּמִצְרָ֑יִם רְדוּ־שָׁ֙מָּה֙ וְשִׁבְרוּ־לָ֣נוּ מִשָּׁ֔ם וְנִחְיֶ֖ה וְלֹ֥א נָמֽוּת׃}
	{Dywedodd hefyd, wele \lhl{clywais} \hlA{fod} ŷd yn yꝛ Aipht, ewch i wared yno, a phꝛynnwch i ni oddi yno, fel y bôm fyw, ac na byddom feirw.}
	{way·yōmɛr hinnē \lhl{šå̄maʿtī} \hlA{kī} yɛš·šɛḇɛr bə·miṣrå̄yim rəḏū·šå̄mm·å̄ wə·šiḇrū·l·å̄nū miš·šå̄m wə·niḥyɛ wə·lō nå̄mūṯ}
	{And he said, Behold, I have \lhl{heard} \hlA{that} there is corn in Egypt: get you down thither, and buy for us from thence; that we may live, and not die.}
\end{example}
\begin{paper}
	\explain In \exvref{Gen.}{42}{2}{} there is a \bh{kī} ‘that’ phrase.
\end{paper}

\begin{example}{Gen.}{41}{15}{1}{}
	\quoling
	{וַיֹּ֤אמֶר פַּרְעֹה֙ אֶל־יֹוסֵ֔ף חֲלֹ֣ום חָלַ֔מְתִּי וּפֹתֵ֖ר אֵ֣ין אֹתֹ֑ו וַאֲנִ֗י \lhl{שָׁמַ֤עְתִּי} עָלֶ֙יךָ֙ \hlA{לֵאמֹ֔ר} תִּשְׁמַ֥ע חֲלֹ֖ום לִפְתֹּ֥ר אֹתֹֽו׃}
	{A Pharao a ddywedodd wꝛth Joſeph, bꝛeuddwydiais freuddwyd, ac nid [oes] ai deonglo ef: ond myfi a \lhl{glywais} \hlA{ddywedyd} am danat ti, y gwꝛandewi freuddwyd iw ddeonglu.}
	{way·yōmɛr parʿō ʾɛl·yōsēp̄ ḥălōm ḥå̄lamtī ū·p̄ōṯēr ʾēn ʾōṯ·ō wa·ʾănī \lhl{šå̄maʿtī} ʿå̄lɛḵå̄ \hlA{lē·mōr} tišmaʿ ḥălōm li·p̄tōr ʾōṯ·ō}
	{And Pharaoh said unto Joseph, I have dreamed a dream, and \kjvit{there is} none that can interpret it: and I have \lhl{heard} say of thee, \hlA{\kjvit{that}} thou canst understand a dream to interpret it.}
\end{example}
\begin{paper}
	\explain And in \exvref{Gen.}{41}{15}{1} another form of content complement occurs.
\end{paper}



\paragraph{\C{gwrando}}

\subparagraphoccurances{\bh{šå̄maʿ}~+ \textsc{preposition}}{84}

\begin{paper}
	When \bh{šå̄maʿ} is followed by a prepositional complement, it has the sense of \hl{obeying} or \hl{accepting}, and as such is translated only by \C{gwrando} with \C{ar} ‘on’ complement.
\end{paper}

\begin{example}{Ex.}{4}{1}{}{}
	\quoling
	{וַיַּ֤עַן מֹשֶׁה֙ וַיֹּ֔אמֶר וְהֵן֙ לֹֽא־יַאֲמִ֣ינוּ לִ֔י וְלֹ֥א \lhl{יִשְׁמְע֖וּ} \hlA{בְּ}קֹלִ֑י כִּ֣י יֹֽאמְר֔וּ לֹֽא־נִרְאָ֥ה אֵלֶ֖יךָ יְהוָֽה׃}
	{Yna Moſes a attebodd, ac a ddywedodd, etto wele ni chꝛedant i mi ac ni \lhl{wꝛandawant} \hlA{ar} fy llais: onid dywedant nid ymgdangoſodd yꝛ Arglwydd i ti.}
	{way·yaʿan mōšɛ way·yōmɛr wə·hēn lō·yaʾămīnū l·ī wə·lō \lhl{yišməʿū} \hlA{bə·}qōlī kī yōmrū lō·nirʾå̄ ʾēlɛḵå̄ {\YHWH}}
	{And Moses answered and said, But, behold, they will not believe me, nor \lhl{hearken} \hlA{unto} my voice: for they will say, The {\LORD} hath not appeared unto thee.}
\end{example}
\begin{paper}
	\explain So in \exvref{Ex.}{4}{1}{} Moses doesn’t fear the children of Israel will not be able to hear him, but he fears they will not \hl{follow}, will not \hl{accept} his orders.
\end{paper}



\subparagraphoccurances{Judgement}{5}

\begin{paper}
	{\click} \bh{šå̄maʿ} in the sense of \hl{judging}, of deciding one way or the other, or of asking God for judgement, is translated by \C{gwꝛando}.
\end{paper}

\begin{example}{Deut.}{1}{16}{}{}
	\quoling
	{וָאֲצַוֶּה֙ אֶת־שֹׁ֣פְטֵיכֶ֔ם בָּעֵ֥ת הַהִ֖וא לֵאמֹ֑ר \lhl{שָׁמֹ֤עַ} בֵּין־אֲחֵיכֶם֙ וּשְׁפַטְתֶּ֣ם צֶ֔דֶק בֵּֽין־אִ֥ישׁ וּבֵין־אָחִ֖יו וּבֵ֥ין גֵּרֽוֹ׃}
	{A’r amſer hwnnw y goꝛchymynnais i’ch barnwŷꝛ chwi gan ddywedyd: \lhl{gwꝛandewch} [ddadleuon] rhwng eich bꝛodyꝛ, a bernwch gyfiawnder rhwng gŵꝛ ai frawd, ac ai eſtron hefyd.}
	{wå̄·ʾăṣawwɛ ʾɛṯ·šōp̄ṭēḵɛm b·å̄·ʿēṯ ha·hi lē·mōr \lhl{šå̄mōaʿ} bēn·ʾăḥēḵɛm ū·šp̄aṭtɛm ṣɛḏɛq bēn·ʾīš ū·ḇēn·ʾå̄ḥīw ū·ḇēn gērō}
	{And I charged your judges at that time, saying, \lhl{Hear} \kjvit{the causes} between your brethren, and judge righteously between \kjvit{every} man and his brother, and the stranger \kjvit{that is} with him.}
\end{example}



\subparagraph{Interpretation of dreams}

\begin{paper}
	{\click} Hearing others’ dreams in the sense of interpreting them is translated by \C{gwꝛando}. As you can see in \exvref{Gen.}{41}{15}{2}:
\end{paper}

\begin{example}{Gen.}{41}{15}{2}{}
	\quoling
	{וַיֹּ֤אמֶר פַּרְעֹה֙ אֶל־יֹוסֵ֔ף חֲלֹ֣ום חָלַ֔מְתִּי וּפֹתֵ֖ר אֵ֣ין אֹתֹ֑ו וַאֲנִ֗י שָׁמַ֤עְתִּי עָלֶ֙יךָ֙ לֵאמֹ֔ר \lhl{תִּשְׁמַ֥ע} \hlA{חֲלֹ֖ום} לִפְתֹּ֥ר אֹתֹֽו׃}
	{A Pharao a ddywedodd wꝛth Joſeph, bꝛeuddwydiais freuddwyd, ac nid [oes] ai deonglo ef: ond myfi a glywais ddywedyd am danat ti, y \lhl{gwꝛandewi} \hlA{freuddwyd} iw ddeonglu.}
	{way·yōmɛr parʿō ʾɛl·yōsēp̄ ḥălōm ḥå̄lamtī ū·p̄ōṯēr ʾēn ʾōṯ·ō wa·ʾănī šå̄maʿtī ʿå̄lɛḵå̄ lē·mōr \lhl{tišmaʿ} \hlA{ḥălōm} li·p̄tōr ʾōṯ·ō}
	{And Pharaoh said unto Joseph, I have dreamed a dream, and \kjvit{there is} none that can interpret it: and I have heard say of thee, \kjvit{that} thou canst \lhl{understand} a \hlA{dream} to interpret it.}
\end{example}

\begin{paper}
	{\click} Morgan reads the Hebrew text in the same manner when Joseph asks his brothers to \hl{hear} his dream (that is, according to Morgan, to \hl{interpret} his dream):
\end{paper}

\begin{example}{Gen.}{37}{6}{}{}
	\quoling
	{וַיֹּ֖אמֶר אֲלֵיהֶ֑ם \lhl{שִׁמְעוּ}־נָ֕א \hlA{הַחֲלֹ֥ום} הַזֶּ֖ה אֲשֶׁ֥ר חָלָֽמְתִּי׃}
	{O blegit dywedaſe wꝛthynt, \lhl{gwꝛandewch} atolwg y \hlA{bꝛeuddwyd} hwn, yꝛ hwn a frenddwydiais.}
	{way·yōmɛr ʾălēhɛm \lhl{šimʿū}·nå̄ ha·\hlA{ḥălōm} haz·zɛ ʾăšɛr ḥå̄lå̄mtī}
	{And he said unto them, \lhl{Hear}, I pray you, this \hlA{dream} which I have dreamed:}
\end{example}

\begin{paper}
	{\click} In comparison, when what’s being heard is the dream’s \hl{content} (the \hl{narrative}) or someone else’s \hl{interpretation} thereof, Morgan uses \C{clywed}:
\end{paper}

\begin{example}{Judg.}{7}{15}{}{}
	\quoling
	{וַיְהִי֩ \lhl{כִשְׁמֹ֨עַ} גִּדְע֜וֹן אֶת־\hlA{מִסְפַּ֧ר הַחֲל֛וֹם} וְאֶת־\hlA{שִׁבְר֖וֹ} וַיִּשְׁתָּ֑חוּ […]}% וַיָּ֙שָׁב֙ אֶל־מַחֲנֵ֣ה יִשְׂרָאֵ֔ל וַיֹּ֣אמֶר ק֔וּמוּ כִּֽי־נָתַ֧ן יְהוָ֛ה בְּיֶדְכֶ֖ם אֶת־מַחֲנֵ֥ה מִדְיָֽן׃}
	{Pan \lhl{glybu} Gedeon \hlA{adꝛoddiad y bꝛeuddwyd} ai \hlA{ddirnad}, yna efe a addolodd: […]} %ac a ddychwelodd i werſſyll Iſrael, ac a ddywedodd, cyfodwch, canys rhoddodd yꝛ Arglwydd werſſyll y Madianiaid yn eich llaw chwi.}
	{way·hī \lhl{ḵi·šmōaʿ} giḏʿōn ʾɛṯ·\hlA{mispar ha·ḥălōm} wə·ʾɛṯ·\hlA{šiḇrō} way·yištå̄ḥū […]}% way·yå̄šå̄ḇ ʾɛl·maḥănē yiśrå̄ʾēl way·yōmɛr qūmū kī·nå̄ṯan {\YHWH} bə·yɛḏḵɛm ʾɛṯ·maḥănē miḏyå̄n}
	{And it was \textit{so}, when Gideon \lhl{heard} the \hlA{telling of the dream}, and the \hlA{interpretation} thereof, that he worshipped, […]}% and returned into the host of Israel, and said, Arise; for the {\LORD} hath delivered into your hand the host of Midian.}
\end{example}

\begin{paper}
	So, hearing as \hl{interpreting} is translated with \C{gwꝛando}; hearing one telling you the content or an interpretaion of a dream is translated with \C{clywed}.
\end{paper}



\paragraphoccurances{\C{deall}}{2}

\begin{paper}
	The Hebrew idiomatic phrase of ‘hearing a language’ in the sense of knowing it has no direct Welsh equivalent using either \C{clywed} or \C{gwrando}; Morgan’s solution was to use \C{deall} ‘to understand’ instead.
\end{paper}

\begin{example}{Gen.}{11}{7}{}{}
	\quoling
	{הָ֚בָה נֵֽרְדָ֔ה וְנָבְלָ֥ה שָׁ֖ם שְׂפָתָ֑ם אֲשֶׁר֙ לֹ֣א \lhl{יִשְׁמְע֔וּ} אִ֖ישׁ \hlA{שְׂפַ֥ת} רֵעֵֽהוּ׃}
	{Deuwch deſcynnwn, a chymmyſcwn yno eu hiaith hwynt fel na \lhl{ddeallo} vn \hlA{iaith} ei gilydd}
	{hå̄ḇå̄ nērḏå̄ wə·nå̄ḇlå̄ šå̄m śəp̄å̄ṯå̄m ʾăšɛr lō \lhl{yišməʿū} ʾīš \hlA{śəp̄aṯ} rēʿēhū}
	{Go to, let us go down, and there confound their language, that they may not \lhl{understand} one another’s \hlA{speech}.}
\end{example}

\begin{example}{Deut.}{28}{49}{}{}
	\quoling
	{יִשָּׂ֣א יְהוָה֩ עָלֶ֨יךָ גּ֤וֹי מֵרָחוֹק֙ מִקְצֵ֣ה הָאָ֔רֶץ כַּאֲשֶׁ֥ר יִדְאֶ֖ה הַנָּ֑שֶׁר גּ֕וֹי אֲשֶׁ֥ר לֹא־\lhl{תִשְׁמַ֖ע} \hlA{לְשֹׁנֽוֹ}׃}
	{Yꝛ Arglwydd a ddwg i’th erbyn genedl o bell [ſef] o eithaf y tir, megis ac yꝛ eheda yꝛ eryꝛ: cenedl yꝛ hon ni \lhl{ddealli} ei \hlA{hiaith}.}
	{yiśśå̄ {\YHWH} ʿå̄lɛḵå̄ gōy mē·rå̄ḥōq mi·qṣē hå̄·ʾå̄rɛṣ ka·ʾăšɛr yiḏʾɛ han·nå̄šɛr gōy ʾăšɛr lō·\lhl{ṯišmaʿ} \hlA{ləšōnō}}
	{The {\LORD} shall bring a nation against thee from far, from the end of the earth, \kjvit{as swift} as the eagle flieth; a nation whose tongue thou shalt not understand;}
\end{example}



\subsubsection{\bh{hɛʾɛ̆zīn}}

\subsubsectionoccurances{\bh{hiṭṭå̄}~+ \bh{ʾōzɛn}}{26}

\begin{paper}
	There is an idiomatic collocation of \bh{hiṭṭå̄} ‘to incline’ and \bh{ʾōzɛn} ‘ear’. This collocation is translated literally mostly by \C{gogwydd} or \C{gostwng} with \C{clust} ‘ear’ as an object. These two lexemes have about the equal number of occurances, and I couldn’t find any clear-cut distinction in their use. \bh{hiṭṭå̄} is translated with \C{***gŵyro} once.

	Almost all occurances of \bh{hiṭṭå̄ ʾōzɛn}, with a few exceptions from Psalms, collocate also with verbs of hearing and accepting: ‘incline your ear and hear (that is, accept)’. \bh{šå̄maʿ} in this position is expectably translated by \C{gwrando}.
\end{paper}

\begin{example}{Isa.}{55}{3}{}{}
	\quoling
	{\lhl{הַטּ֤וּ} \hlA{אָזְנְכֶם֙} וּלְכ֣וּ אֵלַ֔י \hlB{שִׁמְע֖וּ} וּתְחִ֣י נַפְשְׁכֶ֑ם וְאֶכְרְתָ֤ה לָכֶם֙ בְּרִ֣ית עוֹלָ֔ם חַֽסְדֵ֥י דָוִ֖ד הַנֶּאֱמָנִֽים׃}
	{\lhl{Gogwyddwch} eich \hlA{cluſtiau}, a deuwch attaf, \hlB{gwꝛandewch} fel y byddo byw eich enaid, ac mi a wnâf gyfammod tragywyddawl â chwi [ſef] ſiccr dꝛugareddau Dafydd.}
	{\lhl{haṭṭū} \hlA{ʾå̄znəḵɛm} ū·lḵū ʾēla·y \hlB{šimʿū} ū·ṯḥī nap̄šəḵɛm wə·ʾɛḵrəṯå̄ lå̄ḵɛm bərīṯ ʿōlå̄m ḥasḏē ḏå̄wīḏ han·nɛʾɛ̆må̄nīm}
	{\lhl{Incline} your \hlA{ear}, and come unto me: \hlB{hear}, and your soul shall live; and I will make an everlasting covenant with you, \kjvit{even} the sure mercies of David.}
\end{example}

\begin{example}{Prov.}{5}{13}{}{}
	\quoling
	{וְֽלֹא־\hlB{שָׁ֭מַעְתִּי} בְּק֣וֹל מוֹרָ֑י וְ֝לִֽמְלַמְּדַ֗י לֹא־\lhl{הִטִּ֥יתִי} \hlA{אָזְנִֽי}׃}
	{Ac ni \hlB{wꝛandewais} ar lef fy athꝛawon, ac ni \lhl{oſtyngais} fyng-\hlA{hluſt} i’m diſcawdwŷꝛ}
	{wə·lō·\hlB{šå̄maʿtī} bə·qōl mōrå̄y wə·li·mlamməḏay lō·\lhl{hiṭṭīṯī} \hlA{ʾå̄znī}}
	{And have not \hlB{obeyed} the voice of my teachers, nor \lhl{inclined} mine \hlA{ear} to them that instructed me!}
\end{example}



%-%\paragraph{På̄ṯaḥ}
