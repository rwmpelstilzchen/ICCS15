\subsection{Taste}

\begin{paper}
	So our final sense is \hl{taste}.
\end{paper}

\subsubsectionoccurances[\bh{ṭå̄ʿam}]{\bh{ṭå̄ʿam} ‘to taste’}{10}

\begin{paper}
	The main verb translating \bh{ṭå̄ʿam} ‘to taste’ is \C{archwaithu}, and twice it is translated by \C{profi}. In one case, when \bh{ṭå̄ʿam} is used figuratively with a content complement, it is rendered as \C{gweled}, which is probably more suitable for the Welsh idiomatics.
\end{paper}

\begin{example}{1 Sam.}{14}{29}{}{}
	\quoling
	{[…] רְאוּ־נָא֙ כִּֽי־אֹ֣רוּ עֵינַ֔י כִּ֣י \lhl{טָעַ֔מְתִּי} מְעַ֖ט דְּבַ֥שׁ הַזֶּֽה׃}
	{[…], gwelwch yn awꝛ fel y goleuodd fy llygaid o herwydd i mi \lhl{archwaithu} y chydig o’ꝛ mêl hwn.}
	{[…] rəʾū-nå̄ kī-ʾōrū ʿēnay kī \lhl{ṭå̄ʿamtī} məʿaṭ dəḇaš haz-zɛ}
	{[…]: see, I pray you, how mine eyes have been enlightened, because I \lhl{tasted} a little of this honey.}
\end{example}

\begin{example}{Jon.}{3}{7}{}{}
	\quoling
	{[…] הָאָדָ֨ם וְהַבְּהֵמָ֜ה הַבָּקָ֣ר וְהַצֹּ֗אן אַֽל־\lhl{יִטְעֲמוּ֙} מְא֔וּמָה אַ֨ל־יִרְע֔וּ וּמַ֖יִם אַל־יִשְׁתּֽוּ׃}
	{[…]: dŷn, ac anifail, eidion, a dafad, na \lhl{phꝛofant} ddim, na phoꝛant, ac nac yſant ddwfr.}
	{[…] hå̄-ʾå̄ḏå̄m wə-hab-bəhēmå̄ hab-bå̄qå̄r wə-haṣ-ṣōn ʾal-\lhl{yiṭʿămū} məʾūmå̄ ʾal-yirʿū ū-mayim ʾal-yištū}
	{[…], Let neither man nor beast, herd nor flock, \lhl{taste} any thing: let them not feed, nor drink water:}
\end{example}

\begin{example}{Prov.}{31}{18}{}{}
	\quoling
	{\lhl{טָ֭עֲמָה} \hlA{כִּי}־ט֣וֹב סַחְרָ֑הּ לֹֽא־יִכְבֶּ֖ה *בליל **בַלַּ֣יְלָה נֵרָֽהּ׃}
	{Os hi a \lhl{wêl} \hlA{fôd} ei marſiandiaeth yn fuddiol, ni ddiffydd ei chanwyll ar hŷd y nôs.}
	{\lhl{ṭå̄ʿămå̄} \hlA{kī}·ṭōḇ saḥrå̄h lō·yiḵbɛ ḇ·al·laylå̄ nērå̄h}
	{She \lhl{perceiveth} \hlA{that} her merchandise \kjvit{is} good: her candle goeth not out by night.}
\end{example}
\begin{compactdesc}\small
	\item [Vulgate:] \lhl{gustavit} \hlA{quia} bona est negotiatio eius non extinguetur in nocte lucerna illius
	\item [Geneva:] She \lhl{feeleth} \hlA{that} her marchandise is good: her candle is not put out by night.
	\item [Bishops’:] And yf she \lhl{perceaue} \hlA{that} her huswiferie doth good, her candell goeth not out by nyght.
\end{compactdesc}



\subsubsectionoccurances[\bh{ṭaʿam}]{\bh{ṭaʿam} ‘taste’}{12}

\begin{paper}
	The noun \bh{ṭaʿam} can have either a literal meaning, sensory taste, or a figurative one, ‘reason’, ‘advice’, ‘understanding’, etc. Morgan reflects that: the literal meaning is translated by \C{blâs} ‘taste’, and the figurative ones by a variety of fine-tuned words \point{point on slide}.
\end{paper}

\begin{example}{Ex.}{16}{31}{}{}
	\quoling
	{וַיִּקְרְא֧וּ בֵֽית־יִשְׂרָאֵ֛ל אֶת־שְׁמ֖וֹ מָ֑ן וְה֗וּא כְּזֶ֤רַע גַּד֙ לָבָ֔ן וְ\lhl{טַעְמ֖וֹ} כְּצַפִּיחִ֥ת בִּדְבָֽשׁ׃}
	{A thŷ Iſrael a alwaſant ei henw ef Manna: ac efe oedd fel hâd Coꝛiander, yn wynn, ai \lhl{flâs} fel afrllad o fêl.}
	{way·yiqrəʾū ḇēṯ·yiśrå̄ʾēl ʾɛṯ·šəmō må̄n wə·hū kə·zɛraʿ gaḏ lå̄ḇå̄n wə·\lhl{ṭaʿmō} kə·ṣappīḥīṯ bi·ḏḇå̄š}
	{And the house of Israel called the name thereof Manna: and it was like coriander seed, white; and the \lhl{taste} of it was like wafers made with honey.}
\end{example}

\begin{example}{Ps.}{119}{66}{}{}
	\quoling
	{ט֤וּב \lhl{טַ֣עַם} וָדַ֣עַת לַמְּדֵ֑נִי כִּ֖י בְמִצְוֺתֶ֣יךָ הֶאֱמָֽנְתִּי׃}
	{Dyſc i mi iawn \lhl{ddeall}, a gŵybodaeth, o herwydd yn dy oꝛchymynnion di y credais.}
	{ṭūḇ \lhl{ṭaʿam} wå̄·ḏaʿaṯ lamməḏēnī kī ḇə·miṣwōṯɛḵå̄ hɛʾɛ̆må̄ntī}
	{Teach me good \lhl{judgment} and knowledge: for I have believed thy commandments.}
\end{example}

%deall (\bcite{Ps}{119}{66}, \bcite{Prov}{11}{22}),
%rhesum (\bcite{Prov}{26}{16}),
%synnwyr (\bcite{Job}{12}{20}),
%cyngor (\bcite{1Sam}{25}{33}),
%ymadrodd (\bcite{1Sam}{21}{14}),
%agwedd (\bcite{Ps}{34}{1}).

\begin{example}{Jon.}{3}{7}{}{}
	\quoling
	{וַיַּזְעֵ֗ק וַיֹּ֙אמֶר֙ בְּנִֽינְוֵ֔ה \lhl{מִטַּ֧עַם} הַמֶּ֛לֶךְ וּגְדֹלָ֖יו לֵאמֹ֑ר […]}
	{Yna y cyhoeddwyd, ac y dywedwyd yn Ninife \lhl{dꝛwy oꝛchymyn} y bꝛenin ai bendefigion, gan ddywedyd: […]}
	{way·yazʿēq way·yōmɛr bə·nīnwē \lhl{miṭ·ṭaʿam} ham·mɛlɛḵ ū·ḡḏōlå̄w lē·mōr […]}
	{And he caused \kjvit{it} to be proclaimed and published through Nineveh \lhl{by the decree} of the king and his nobles, saying, […]}
\end{example}



\subsubsectionoccurances[\bh{maṭʿammīm}]{\bh{maṭʿammīm} ‘delicacies’}{8}

\begin{paper}
	Another word of the same root is the \textit{pluralis tantum} noun \bh{maṭʿammīm} ‘delicacies’, which is translated by \C{dainteithion}.
\end{paper}

\begin{example}{Prov.}{23}{3}{}{}
	\quoling
	{אַל־תִּ֭תְאָו‪[Q]‬‪[q]‬ \lhl{לְמַטְעַמּוֹתָ֑יו} וְ֝ה֗וּא לֶ֣חֶם כְּזָבִֽים׃}
	{Na ddeiſyf ei \lhl{ddanteithion} ef: canys twyllodꝛus yw bwyd.}
	{ʾal·tiṯʾå̄w lə·\lhl{maṭʿammōṯå̄w} wə·hū lɛḥɛm kəzå̄ḇīm}
	{Be not desirous of his \lhl{dainties}: for they are deceitful meat.}
\end{example}
