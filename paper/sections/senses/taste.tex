\subsection{Taste}

\subsubsectionoccurances[\bh{ṭå̄ʿam}]{\bh{ṭå̄ʿam} ‘to taste’}{11}

\subsubsectionoccurances[\bh{ṭaʿam}]{\bh{ṭaʿam} ‘taste’}{12}

\begin{paper}
	The noun \bh{ṭaʿam} can have either a literal meaning, sensory taste, or a figurative one, ‘reason’, ‘advice’, ‘understanding’, etc. Morgan reflects that: the literal meaning is translated by \C{blâs} ‘taste’, and the figurative ones by a variety of fine-tuned words \point{point on slide} with the \textit{hápax legómenon} adverbial phrase \bh{miṭ-ṭaʿam} translated as \C{dꝛwy oꝛchymyn}.
\end{paper}

\begin{example}{Ex.}{16}{31}{}{}
	\quoling
	{וַיִּקְרְא֧וּ בֵֽית־יִשְׂרָאֵ֛ל אֶת־שְׁמ֖וֹ מָ֑ן וְה֗וּא כְּזֶ֤רַע גַּד֙ לָבָ֔ן וְ\lhl{טַעְמ֖וֹ} כְּצַפִּיחִ֥ת בִּדְבָֽשׁ׃}
	{A thŷ Iſrael a alwaſant ei henw ef Manna: ac efe oedd fel hâd Coꝛiander, yn wynn, ai \lhl{flâs} fel afrllad o fêl.}
	{way·yiqrəʾū ḇēṯ·yiśrå̄ʾēl ʾɛṯ·šəmō må̄n wə·hū kə·zɛraʿ gaḏ lå̄ḇå̄n wə·\lhl{ṭaʿmō} kə·ṣappīḥīṯ bi·ḏḇå̄š}
	{And the house of Israel called the name thereof Manna: and it was like coriander seed, white; and the \lhl{taste} of it was like wafers made with honey.}
\end{example}

\begin{example}{Ps.}{119}{66}{}{}
	\quoling
	{ט֤וּב \lhl{טַ֣עַם} וָדַ֣עַת לַמְּדֵ֑נִי כִּ֖י בְמִצְוֺתֶ֣יךָ הֶאֱמָֽנְתִּי׃}
	{Dyſc i mi iawn \lhl{ddeall}, a gŵybodaeth, o herwydd yn dy oꝛchymynnion di y credais.}
	{ṭūḇ \lhl{ṭaʿam} wå̄·ḏaʿaṯ lamməḏēnī kī ḇə·miṣwōṯɛḵå̄ hɛʾɛ̆må̄ntī}
	{Teach me good \lhl{judgment} and knowledge: for I have believed thy commandments.}
\end{example}

%deall (\bcite{Ps}{119}{66}, \bcite{Prov}{11}{22}),
%rhesum (\bcite{Prov}{26}{16}),
%synnwyr (\bcite{Job}{12}{20}),
%cyngor (\bcite{1Sam}{25}{33}),
%ymadrodd (\bcite{1Sam}{21}{14}),
%agwedd (\bcite{Ps}{34}{1}).

\begin{example}{Jon.}{3}{7}{}{}
	\quoling
	{וַיַּזְעֵ֗ק וַיֹּ֙אמֶר֙ בְּנִֽינְוֵ֔ה \lhl{מִטַּ֧עַם} הַמֶּ֛לֶךְ וּגְדֹלָ֖יו לֵאמֹ֑ר הָאָדָ֨ם וְהַבְּהֵמָ֜ה הַבָּקָ֣ר וְהַצֹּ֗אן אַֽל־יִטְעֲמוּ֙ מְא֔וּמָה אַ֨ל־יִרְע֔וּ וּמַ֖יִם אַל־יִשְׁתּֽוּ׃}
	{Yna y cyhoeddwyd, ac y dywedwyd yn Ninife \lhl{dꝛwy oꝛchymyn} y bꝛenin ai bendefigion, gan ddywedyd: dŷn, ac anifail, eidion, a dafad, na phꝛofant ddim, na phoꝛant, ac nac yſant ddwfr.}
	{way·yazʿēq way·yōmɛr bə·nīnwē \lhl{miṭ·ṭaʿam} ham·mɛlɛḵ ū·ḡḏōlå̄w lē·mōr hå̄·ʾå̄ḏå̄m wə·hab·bəhēmå̄ hab·bå̄qå̄r wə·haṣ·ṣōn ʾal·yiṭʿămū məʾūmå̄ ʾal·yirʿū ū·mayim ʾal·yištū}
	{And he caused \kjvit{it} to be proclaimed and published through Nineveh \lhl{by the decree} of the king and his nobles, saying, Let neither man nor beast, herd nor flock, taste any thing: let them not feed, nor drink water:}
\end{example}



\subsubsectionoccurances[\bh{maṭʿammīm}]{\bh{maṭʿammīm} ‘delicacies’}{8}

\begin{paper}
	Another word of the same root is the pluralis tantum verb \bh{maṭʿammīm} ‘delicacies’, which is only by \C{dainteithion}.
\end{paper}

\begin{example}{Prov.}{23}{3}{}{}
	\quoling
	{אַל־תִּ֭תְאָו‪[Q]‬‪[q]‬ \lhl{לְמַטְעַמּוֹתָ֑יו} וְ֝ה֗וּא לֶ֣חֶם כְּזָבִֽים׃}
	{Na ddeiſyf ei \lhl{ddanteithion} ef: canys twyllodꝛus yw bwyd.}
	{ʾal·tiṯʾå̄w lə·\lhl{maṭʿammōṯå̄w} wə·hū lɛḥɛm kəzå̄ḇīm}
	{Be not desirous of his \lhl{dainties}: for they are deceitful meat.}
\end{example}
