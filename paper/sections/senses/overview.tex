\setcounter{subsection}{-1}
\subsection{Overiew}

\begin{paper}
	{\click} The first question one has to ask oneself when approaching our topic is ‘\hl{what are the senses in question?}’. To try and answer this question with respect to the Bible we will have to deviate to antropology, theology, literary theory, cognitive science and other fields. Unfortunately, we don’t have to do this…

	In a recent publication, \cite{avrahami.y:2012:senses} came to the conclusion of a \hl{septasensory} model for biblical epistemology: sight, hearing, kinaesthesia, speech, taste/eating, smell and touch. This model is not uncontroversial, {\click} and for our current purposes we will use the (Western) \hl{pentasensory} model, which is more familiar to us.
\end{paper}

\begin{hopoint}
	\begin{tabular}{ll@{\quad→\quad}l}
		\textbf{sight}   & \bh{råʾå}   & \C{gweled, edrych, …}\\
		\textbf{hearing} & \bh{šåmaʿ}  & \C{clywed, gwrando, …}\\
		\textbf{touch}   & \bh{måšaš}  & \C{teimlo, …}\\
		\textbf{smell}   & \bh{hērīaḥ} & \C{arogli, …}\\
		\textbf{taste}   & \bh{ṭåʿam}  & \C{archwaithu, …}
	\end{tabular}
\end{hopoint}

\begin{paper}
\point{point} These are the most common \hl{Hebrew verbs} for these modalities and their most common \hl{translations}.

\tounfold{לפרט על השלושנקודות ומה אני עושה כאן}
\end{paper}



\subsubsection{Lists of modalities}

\tounfold{לכלול את:
	דניאל 5:23
	שאר הדוגמאות שמצאתי בעבודה הסמינריונית
	החלק הרלוונטי ב„חושים” במסד הנתונים}

\begin{example}{5}{4}{28}{}{}
	\quoling
	{וַעֲבַדְתֶּם־שָׁ֣ם אֱלֹהִ֔ים מַעֲשֵׂ֖ה יְדֵ֣י אָדָ֑ם עֵ֣ץ וָאֶ֔בֶן אֲשֶׁ֤ר לֹֽא־\hlA{יִרְאוּן֙} וְלֹ֣א \hlA{יִשְׁמְע֔וּן} וְלֹ֥א \hlA{יֹֽאכְל֖וּן} וְלֹ֥א \hlA{יְרִיחֻֽן}׃}
	{Ac yno y gwaſanaethwch dduwiau [o] waith dwylo dŷn, [ſef] pꝛen, a maen, y rhai ni \hlA{welant}, ac ni \hlA{chlywant}, ac ni \hlA{fwyttânt}, ac ni \hlA{aroglant}.}
	{wa·ʿăḇaḏtɛm·šåm ʾɛ̆lōhīm maʿăśē yəḏē ʾåḏåm ʿēṣ wå·ʾɛḇɛn ʾăšɛr lō·\hlA{yirʾūn} wə·lō \hlA{yišməʿūn} wə·lō \hlA{yōḵlūn} wə·lō \hlA{yərīḥun}}
	{And there ye shall serve gods, the work of men’s hands, wood and stone, which neither \hlA{see}, nor \hlA{hear}, nor \hlA{eat}, nor \hlA{smell}.}
\end{example}

\begin{example}{Ps}{115}{4–7}{}{}
	\quoling
	{%
		\textsuperscript{4}~עֲצַבֵּיהֶם כֶּ֣סֶף וְזָהָ֑ב מַ֝עֲשֵׂ֗ה יְדֵ֣י אָדָֽם׃
		\textsuperscript{5}~פֶּֽה־לָ֭הֶם וְלֹ֣א \hlA{יְדַבֵּ֑רוּ} עֵינַ֥יִם לָ֝הֶ֗ם וְלֹ֣א \hlA{יִרְאֽוּ}׃
		\textsuperscript{6}~אָזְנַ֣יִם לָ֭הֶם וְלֹ֣א \hlA{יִשְׁמָ֑עוּ} אַ֥ף לָ֝הֶ֗ם וְלֹ֣א \hlA{יְרִיחֽוּן}׃
		\textsuperscript{7}~יְדֵיהֶ֤ם ׀ וְלֹ֬א \hlA{יְמִישׁ֗וּן} רַ֭גְלֵיהֶם וְלֹ֣א \hlA{יְהַלֵּ֑כוּ} לֹֽא־\hlA{יֶ֝הְגּ֗וּ} בִּגְרוֹנָֽם׃
	}
	{%
		\textsuperscript{4}~Eu delwau hwy [ydynt] o aur, ac arian, [ſef] o waith dynnion.
		\textsuperscript{5}~Genau [ſydd] iddynt, ac ni \hlA{lefarant}, llygaid [ſydd] ganddynt, ac ni \hlA{welant}.
		\textsuperscript{6}~[Y mae] cluſtiau iddynt, ac ni \hlA{chlywant}, ffroenau [ſydd] ganddynt, ac ni \hlA{aroglant}.
		\textsuperscript{7}~Dwylo [ſydd] iddynt, ac ni \hlA{theimlant}: traed [ſy] iddynt, ac ni \hlA{cherddant}: ni \hlA{leiſiant} [ychwaith] ai gwddf.
	}
	{%
		\textsuperscript{4}~ʿăṣabbēhɛm kɛsɛp̄ wə·zåhåḇ maʿăśē yəḏē ʾåḏåm
		\textsuperscript{5}~pɛ·l·åhɛm wə·lō \hlA{yəḏabbērū} ʿēnayim l·åhɛm wə·lō \hlA{yirʾū}
		\textsuperscript{6}~ʾåznayim l·åhɛm wə·lō \hlA{yišmåʿū} ʾap̄ l·åhɛm wə·lō \hlA{yərīḥūn}
		\textsuperscript{7}~yəḏēhɛm wə·lō \hlA{yəmīšūn} raḡlēhɛm wə·lō \hlA{yəhallēḵū }lō·\hlA{yɛhgū} bi·ḡrōnåm
	}
	{%
		\textsuperscript{4}~Their idols \textit{are} silver and gold, the work of men’s hands.
		\textsuperscript{5}~They have mouths, but they \hlA{speak} not: eyes have they, but they \hlA{see} not:
		\textsuperscript{6}~They have ears, but they \hlA{hear} not: noses have they, but they \hlA{smell} not:
		\textsuperscript{7}~They have hands, but they \hlA{handle} not: feet have they, but they \hlA{walk} not: neither \hlA{speak} they through their throat.
	}
\end{example}



\subsubsection{\C{gweled}:\C{edrych} and \C{clywed}:\C{gwrando}}

\begin{paper}
	{\click} The \hl{key lexical distinctions} Morgan had to make when translating instances of the common verbs \bh{råʾå} ‘to see’ and \bh{šåmaʿ} ‘to hear’ are between \C{gweled} and \C{edrych}, and \C{clywed} and \C{gwrando}, respectively.

	{\click} The \hl{value} of these choices is, in the nutshell, that:
	\begin{compactitem}
		\item \C{gweled} and \C{clywed} designate simple, semantically unmarked \hl{sensory perception}, as well as perception of \hl{content} (seeing or hearing that something is such and such).
		\item \C{edrych} and \C{gwrando}, on the other hand, are marked with \hl{non-sensory meaning}, whether involving actual sensory perception or not.
			\begin{compactitem}
				\item For \C{edrych} this includes facing (with indication of direction or without it), looking at or for, accepting, examining, inspecting, etc.
				\item For \C{gwrando} this includes obeying, accepting, following, judging, interpreting, etc.
			\end{compactitem}
	\end{compactitem}

	So, as one can expect from this generalization, in the lists of modalities we’ve talked about it’s \C{gweled} and \C{clywed}, not \C{edrych} and \C{gwrando}, that are found.
\end{paper}

%*** \begin{hopoint}
%*** 	\begin{tabular}{l|cc}
%*** 		& {sensory} & {\ruby{additional}{(non-sensory)} meaning}\\
%*** 		\hline
%*** 		\C{gweled}, \C{clywed}  & + & -\\
%*** 		\C{edrych}, \C{gwrando} & ± & +
%*** 	\end{tabular}
%*** \end{hopoint}

\tounfold{דוגמה להכניס איפשהו:}

\begin{example}{Job}{35}{13}{}{}
	\quoling
	{אַךְ־שָׁ֭וְא לֹא־\highlight{יִשְׁמַ֥ע} ׀ אֵ֑ל וְ֝שַׁדַּ֗י לֹ֣א \highlight{יְשׁוּרֶֽנָּה}׃}
	{Diau na \highlight{wꝛendu} Duw oferedd: ac nad \highlight{edꝛych} yꝛ Holl-alluoc arno.}
	{ʾaḵ·šåw lō·\highlight{yišmaʿ} ʾēl wə·šadday lō \highlight{yəšūrɛnnå}}
	{Surely God will not \highlight{hear} vanity, neither will the Almighty \highlight{regard} it.}
\end{example}

\tounfold{***יכולת כללית תמיד הפשוט; פעולה מורכבת מסויימת תמיד המורכב}
