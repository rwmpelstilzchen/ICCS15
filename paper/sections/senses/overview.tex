\setcounter{subsection}{-1}
\subsection{Overiew}

\begin{paper}
	{\click} The first question one has to ask oneself when approaching our topic is ‘\hl{what are the senses in question?}’. To try and answer this question with respect to the Bible we will have to deviate to antropology, theology, literary theory, cognitive science and other fields. Unfortunately, we don’t have to do this…

	In a recent publication, \cite{avrahami.y:2012:senses} came to the conclusion of a \hl{septasensory} model for biblical epistemology: sight, hearing, kinaesthesia, speech, taste/eating, smell and touch. This model is not uncontroversial, {\click} and for our current purposes we will use the (Western) \hl{pentasensory} model, which is more familiar to us.
\end{paper}

\begin{hopoint}
	\begin{tabular}{ll@{\quad→\quad}l}
		\textbf{sight}   & \bh{rå̄ʾå̄}   & \C{gweled, edrych, …}\\
		\textbf{hearing} & \bh{šå̄maʿ}  & \C{clywed, gwrando, …}\\
		\textbf{touch}   & \bh{må̄šaš}  & \C{teimlo, …}\\
		\textbf{smell}   & \bh{hērīaḥ} & \C{arogli, …}\\
		\textbf{taste}   & \bh{ṭå̄ʿam}  & \C{archwaithu, …}
	\end{tabular}
\end{hopoint}

\begin{paper}
\point{point} These are the most common \hl{Hebrew verbs} for these modalities and their most common \hl{translations}.

\tounfold{לפרט על השלושנקודות ומה אני עושה כאן}
\end{paper}



\subsubsection{Listing modalities}

\tounfold{לכלול את:
	החלק הרלוונטי ב„חושים” במסד הנתונים}

\paragraph{Inability of idols}

\begin{example}{Deut.}{4}{28}{}{}
	\quoling
	{וַעֲבַדְתֶּם־שָׁ֣ם אֱלֹהִ֔ים מַעֲשֵׂ֖ה יְדֵ֣י אָדָ֑ם עֵ֣ץ וָאֶ֔בֶן אֲשֶׁ֤ר לֹֽא־\hlA{יִרְאוּן֙} וְלֹ֣א \hlA{יִשְׁמְע֔וּן} וְלֹ֥א \hlA{יֹֽאכְל֖וּן} וְלֹ֥א \hlA{יְרִיחֻֽן}׃}
	{Ac yno y gwaſanaethwch dduwiau [o] waith dwylo dŷn, [ſef] pꝛen, a maen, y rhai ni \hlA{welant}, ac ni \hlA{chlywant}, ac ni \hlA{fwyttânt}, ac ni \hlA{aroglant}.}
	{wa·ʿăḇaḏtɛm·šå̄m ʾɛ̆lōhīm maʿăśē yəḏē ʾå̄ḏå̄m ʿēṣ wå̄·ʾɛḇɛn ʾăšɛr lō·\hlA{yirʾūn} wə·lō \hlA{yišməʿūn} wə·lō \hlA{yōḵlūn} wə·lō \hlA{yərīḥun}}
	{And there ye shall serve gods, the work of men’s hands, wood and stone, which neither \hlA{see}, nor \hlA{hear}, nor \hlA{eat}, nor \hlA{smell}.}
\end{example}

\begin{example}{Ps.}{115}{4–7}{}{}
	\quoling
	{%
		\verseno{4}~עֲצַבֵּיהֶם כֶּ֣סֶף וְזָהָ֑ב מַ֝עֲשֵׂ֗ה יְדֵ֣י אָדָֽם׃\ %
		\verseno{5}~פֶּֽה־לָ֭הֶם וְלֹ֣א \hlA{יְדַבֵּ֑רוּ} עֵינַ֥יִם לָ֝הֶ֗ם וְלֹ֣א \hlA{יִרְאֽוּ}׃\ %
		\verseno{6}~אָזְנַ֣יִם לָ֭הֶם וְלֹ֣א \hlA{יִשְׁמָ֑עוּ} אַ֥ף לָ֝הֶ֗ם וְלֹ֣א \hlA{יְרִיחֽוּן}׃\ %
		\verseno{7}~יְדֵיהֶ֤ם ׀ וְלֹ֬א \hlA{יְמִישׁ֗וּן} רַ֭גְלֵיהֶם וְלֹ֣א \hlA{יְהַלֵּ֑כוּ} לֹֽא־\hlA{יֶ֝הְגּ֗וּ} בִּגְרוֹנָֽם׃
	}
	{%
		\verseno{4}~Eu delwau hwy [ydynt] o aur, ac arian, [ſef] o waith dynnion.\ %
		\verseno{5}~Genau [ſydd] iddynt, ac ni \hlA{lefarant}, llygaid [ſydd] ganddynt, ac ni \hlA{welant}.\ %
		\verseno{6}~[Y mae] cluſtiau iddynt, ac ni \hlA{chlywant}, ffroenau [ſydd] ganddynt, ac ni \hlA{aroglant}.\ %
		\verseno{7}~Dwylo [ſydd] iddynt, ac ni \hlA{theimlant}: traed [ſy] iddynt, ac ni \hlA{cherddant}: ni \hlA{leiſiant} [ychwaith] ai gwddf.
	}
	{%
		\verseno{4}~ʿăṣabbēhɛm kɛsɛp̄ wə·zå̄hå̄ḇ maʿăśē yəḏē ʾå̄ḏå̄m\ %
		\verseno{5}~pɛ·l·å̄hɛm wə·lō \hlA{yəḏabbērū} ʿēnayim l·å̄hɛm wə·lō \hlA{yirʾū}\ %
		\verseno{6}~ʾå̄znayim l·å̄hɛm wə·lō \hlA{yišmå̄ʿū} ʾap̄ l·å̄hɛm wə·lō \hlA{yərīḥūn}\ %
		\verseno{7}~yəḏēhɛm wə·lō \hlA{yəmīšūn} raḡlēhɛm wə·lō \hlA{yəhallēḵū} lō·\hlA{yɛhgū} bi·ḡrōnå̄m
	}
	{%
		\verseno{4}~Their idols \textit{are} silver and gold, the work of men’s hands.\ %
		\verseno{5}~They have mouths, but they \hlA{speak} not: eyes have they, but they \hlA{see} not:\ %
		\verseno{6}~They have ears, but they \hlA{hear} not: noses have they, but they \hlA{smell} not:\ %
		\verseno{7}~They have hands, but they \hlA{handle} not: feet have they, but they \hlA{walk} not: neither \hlA{speak} they through their throat.
	}
\end{example}

\begin{example}{Ps.}{135}{16–17}{}{}
	\quoling
	{%
		\verseno{16}~פֶּֽה־לָ֭הֶם וְלֹ֣א יְדַבֵּ֑רוּ עֵינַ֥יִם לָ֝הֶ֗ם וְלֹ֣א \hlA{יִרְאֽוּ}׃\ %
		\verseno{17}~אָזְנַ֣יִם לָ֭הֶם וְלֹ֣א \hlA{יַאֲזִ֑ינוּ} אַ֝֗ף אֵין־יֶשׁ־ר֥וּחַ בְּפִיהֶֽם׃
	}
	{%
		\verseno{16}~Genau [ſydd] iddynt, ac ni lefarant: llygaid [ſydd] ganddynt, ac ni \hlA{welant}.\ %
		\verseno{17}~[Y mae] cluſtiau iddynt, ac ni \hlA{chlywant}: nid oes ychwaith anadl yn eu genau.
	}
	{%
		\verseno{16}~pɛ-lå̄hɛm wə-lō yəḏabbērū ʿēnayim lå̄hɛm wə-lō \hlA{yirʾū}\ %
		\verseno{17}~ʾå̄znayim lå̄hɛm wə-lō \hlA{yaʾăzīnū} ʾap̄ ʾēn-yɛš-rūaḥ bə-p̄īhɛm
	}
	{%
		\verseno{16}~They have mouths, but they speak not; eyes have they, but they \hlA{see} not;\ %
		\verseno{17}~They have ears, but they \hlA{hear} not; neither is there any breath in their mouths.
	}
\end{example}

\begin{example}{Dan.}{5}{23}{}{}
	\quoling
	{[…] וְלֵֽאלָהֵ֣י כַסְפָּֽא־וְ֠דַהֲבָא נְחָשָׁ֨א פַרְזְלָ֜א אָעָ֣א וְאַבְנָ֗א דִּ֠י לָֽא־\hlA{חָזַ֧יִן} וְלָא־\hlA{שָׁמְעִ֛ין} וְלָ֥א יָדְעִ֖ין שַׁבַּ֑חְתָּ […]}
	{[…], a thi a foliennaiſt dduwiau o arian, ac o aur, o bꝛês, o haiarn, o bꝛenn, ac o faen: y rhai ni \hlA{welant}, ac ni \hlA{chlywant}, ac ni ŵyddant [ddim:] […]}
	{wlʾlhy kspʾ-wdhbʾ nḥšʾ przlʾ ʾʿʾ wʾbnʾ dy lʾ-\hlA{ḥzyn} wlʾ-\hlA{šmʿyn} wlʾ ydʿyn šbḥt}
	{[…] and thou hast praised the gods of silver, and gold, of brass, iron, wood, and stone, which \hlA{see} not, nor \hlA{hear}, nor know: […]}
\end{example}



\paragraph{Other listings}

\begin{example}{Deut.}{29}{3}{}{}
	\quoling
	{וְלֹֽא־נָתַן֩ יְהוָ֨ה לָכֶ֥ם לֵב֙ לָדַ֔עַת וְעֵינַ֥יִם \hlA{לִרְא֖וֹת} וְאָזְנַ֣יִם \hlA{לִשְׁמֹ֑עַ} עַ֖ד הַיּ֥וֹם הַזֶּֽה׃}
	{Ond ni roddodd yꝛ Arglwydd i chwi galon i wybod, na llygaid i \hlA{weled}, na chluſtiau i \hlA{glywed} hyd y dydd hwn.}
	{wə·lō·nå̄ṯan {\YHWH} lå̄ḵɛm lēḇ lå̄·ḏaʿaṯ wə·ʿēnayim li·\hlA{rʾōṯ} wə·ʾå̄znayim li·\hlA{šmōaʿ} ʿaḏ hay·yōm haz·zɛ}
	{Yet the {\LORD} hath not given you an heart to perceive, and eyes to \hlA{see}, and ears to \hlA{hear}, unto this day.}
\end{example}

\begin{example}{Jer.}{5}{21}{}{}
	\quoling
	{שִׁמְעוּ־נָ֣א זֹ֔את עַ֥ם סָכָ֖ל וְאֵ֣ין לֵ֑ב עֵינַ֤יִם לָהֶם֙ וְלֹ֣א \hlA{יִרְא֔וּ} אָזְנַ֥יִם לָהֶ֖ם וְלֹ֥א \hlA{יִשְׁמָֽעוּ}׃}
	{Gwꝛando hyn ti bobl ynfyd, a digalon, llygaid [ſydd] iddynt, ac ni \hlA{welant}: cluſtiau [ſydd] yddynt, ac ni \hlA{chlywant}.}
	{šimʿū·nå̄ zōṯ ʿam så̄ḵå̄l wə·ʾēn lēḇ ʿēnayim lå̄hɛm wə·lō \hlA{yirʾū} ʾå̄znayim lå̄hɛm wə·lō \hlA{yišmå̄ʿū}}
	{Hear now this, O foolish people, and without understanding; which have eyes, and \hlA{see} not; which have ears, and \hlA{hear} not:}
\end{example}

\begin{example}{Ezek.}{12}{2}{}{}
	\quoling
	{בֶּן־אָדָ֕ם בְּת֥וֹךְ בֵּית־הַמֶּ֖רִי אַתָּ֣ה יֹשֵׁ֑ב אֲשֶׁ֣ר עֵינַיִם֩ לָהֶ֨ם \hlA{לִרְא֜וֹת} וְלֹ֣א \hlA{רָא֗וּ} אָזְנַ֨יִם לָהֶ֤ם \hlA{לִשְׁמֹ֙עַ֙} וְלֹ֣א \hlA{שָׁמֵ֔עוּ} כִּ֛י בֵּ֥ית מְרִ֖י הֵֽם׃}
	{Trigo’r ydwyt ti fab dyn, yng-hanol tŷ anufyddgar, y rhai [y mae] llygaid iddynt i \hlA{weled}, ac ni \hlA{welant}, a chluſtiau iddynt i \hlA{glywed}, ac ni \hlA{chlywant}: canys tŷ anufyddgar ydynt hwy.}
	{bɛn·ʾå̄ḏå̄m bə·ṯōḵ bēṯ·ham·mɛrī ʾattå̄ yōšēḇ ʾăšɛr ʿēnayim lå̄hɛm li·\hlA{rʾōṯ} wə·lō \hlA{rå̄ʾū} ʾå̄znayim lå̄hɛm li·\hlA{šmōaʿ} wə·lō \hlA{šå̄mēʿū} kī bēṯ mərī hēm}
	{Son of man, thou dwellest in the midst of a rebellious house, which have eyes to \hlA{see}, and \hlA{see} not; they have ears to \hlA{hear}, and \hlA{hear} not: for they \kjvit{are} a rebellious house.}
\end{example}

\begin{example}{Ezek.}{44}{5}{}{}
	\quoling
	{וַיֹּ֨אמֶר אֵלַ֜י יְהֹוָ֗ה בֶּן־אָדָ֡ם שִׂ֣ים לִבְּךָ֩ \hlA{וּרְאֵ֨ה} בְעֵינֶ֜יךָ וּבְאָזְנֶ֣יךָ \hlA{שְּׁמָ֗ע} אֵ֣ת כָּל־אֲשֶׁ֤ר אֲנִי֙ מְדַבֵּ֣ר אֹתָ֔ךְ […]}
	{Yna y dywedodd yꝛ Arglwydd wꝛthif, goſot dy galon fab dŷn, a \hlA{gwel} a’th lygaid, \hlA{clyw} hefyd a’th gluſtiau ’r hyn oll yꝛ ydwyfyn ei ddywedyd wꝛthit […]}
	{way-yōmɛr ʾēla-y {\YHWH} bɛn-ʾå̄ḏå̄m śīm libbəḵå̄ ū-\hlA{rʾē} ḇə-ʿēnɛḵå̄ ū-ḇ-ʾå̄znɛḵå̄ \hlA{šəmå̄ʿ} ʾēṯ kå̄l-ʾăšɛr ʾănī məḏabbēr ʾōṯå̄-ḵ […]}
	{And the {\LORD} said unto me, Son of man, mark well, and \hlA{behold} with thine eyes, and \hlA{hear} with thine ears all that I say unto thee […]}
\end{example}



\subsubsection{\C{gweled}:\C{edrych} and \C{clywed}:\C{gwrando}}

\begin{paper}
	{\click} The \hl{key lexical distinctions} Morgan had to make when translating instances of the common verbs \bh{rå̄ʾå̄} ‘to see’ and \bh{šå̄maʿ} ‘to hear’ are between \C{gweled} and \C{edrych}, and \C{clywed} and \C{gwrando}, respectively.

	{\click} The \hl{value} of these choices is, in the nutshell, that:
	\begin{compactitem}
		\item \C{gweled} and \C{clywed} designate simple, semantically unmarked \hl{sensory perception}, \hl{understanding} (with \C{gweled}), and perception of \hl{content} (seeing or hearing something is such and such).
		\item \C{edrych} and \C{gwrando}, on the other hand, are marked with \hl{non-sensory meaning}, whether involving actual sensory perception or not.
			\begin{compactitem}
				\item For \C{edrych} this includes facing (with indication of direction or without it), looking at or for, accepting, examining, inspecting, etc.
				\item For \C{gwrando} this includes obeying, accepting, following, judging, interpreting, etc.
			\end{compactitem}
	\end{compactitem}

	So, as one can expect from this generalization, in the lists of modalities we’ve talked about it’s \C{gweled} and \C{clywed}, not \C{edrych} and \C{gwrando}, that are found.
\end{paper}

%*** \begin{hopoint}
%*** 	\begin{tabular}{l|cc}
%*** 		& {sensory} & {\ruby{additional}{(non-sensory)} meaning}\\
%*** 		\hline
%*** 		\C{gweled}, \C{clywed}  & + & -\\
%*** 		\C{edrych}, \C{gwrando} & ± & +
%*** 	\end{tabular}
%*** \end{hopoint}

\tounfold{דוגמה להכניס איפשהו:}

\begin{example}{Job}{35}{13}{}{}
	\quoling
	{אַךְ־שָׁ֭וְא לֹא־\highlight{יִשְׁמַ֥ע} ׀ אֵ֑ל וְ֝שַׁדַּ֗י לֹ֣א \highlight{יְשׁוּרֶֽנָּה}׃}
	{Diau na \highlight{wꝛendu} Duw oferedd: ac nad \highlight{edꝛych} yꝛ Holl-alluoc arno.}
	{ʾaḵ·šå̄w lō·\highlight{yišmaʿ} ʾēl wə·šadday lō \highlight{yəšūrɛnnå̄}}
	{Surely God will not \highlight{hear} vanity, neither will the Almighty \highlight{regard} it.}
\end{example}

\tounfold{***יכולת כללית תמיד הפשוט; פעולה מורכבת מסויימת תמיד המורכב}
