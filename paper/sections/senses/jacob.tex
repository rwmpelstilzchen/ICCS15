\subsection[Jacob gaining Isaac’s blessing]{Jacob gaining Isaac’s blessing (\llyfr{Gen.}~27)}

\begin{paper}
	{\click} Now, before we finish, some homework… On your handout you can see the story of how Jacob gained his father’s blessing through deception, typeset bilingually with the relevant words marked. In this story the sensorium is crucial, and all senses are present: sight, hearing, touch, smell, taste. Read it in your free time…
\end{paper}

\begin{Parallel}{0.425\textwidth}{0.545\textwidth}
	\ParallelLText{
	\footnotesize\fontspec[AutoFakeBold=3, Mapping=ehll]{Gentium}
	\verseno{1}~way·hī kī·zå̄qēn yiṣḥå̄q wat·tiḵhɛnå̄ ʿēnå̄w mē·\textbf{rʾōṯ} way·yiqrå̄ ʾɛṯ·ʿēśå̄w bənō hag·gå̄ḏōl way·yōmɛr ʾēlå̄·w bənī way·yōmɛr ʾēlå̄·w hinnēnī·\ %
	\verseno{2}~way·yōmɛr hinnē·nå̄ zå̄qantī lō yå̄ḏaʿtī yōm mōṯī\ %
	\verseno{3}~wə·ʿattå̄ śå̄·nå̄ ḵēlɛḵå̄ tɛlyəḵå̄ wə·qaštɛḵå̄ wə·ṣē haś·śå̄ḏɛ wə·ṣūḏå̄ lī· ṣå̄yiḏ\ %
	\verseno{4}~wa·ʿăśē·lī· \textbf{maṭʿammīm} ka·ʾăšɛr ʾå̄haḇtī wə·hå̄ḇīʾå̄ lī· wə·ʾōḵēlå̄ ba·ʿăḇūr təḇå̄rɛḵəḵå̄ nap̄šī bə·ṭɛrɛm ʾå̄mūṯ\ %
	\verseno{5}~wə·riḇqå̄ \textbf{šōmaʿaṯ} bə·ḏabbēr yiṣḥå̄q ʾɛl·ʿēśå̄w bənō way·yēlɛḵ ʿēśå̄w haś·śå̄ḏɛ lå̄·ṣūḏ ṣayiḏ lə·hå̄ḇī\ %
	\verseno{6}~wə·riḇqå̄ ʾå̄mrå̄ ʾɛl·yaʿăqōḇ bənå̄h lē·mōr hinnē \textbf{šå̄maʿtī} ʾɛṯ·ʾå̄ḇīḵå̄ məḏabbēr ʾɛl·ʿēśå̄w ʾå̄ḥīḵå̄ lē·mōr\ %
	\verseno{7}~hå̄ḇīʾå̄ lī· ṣayiḏ wa·ʿăśē·lī· \textbf{maṭʿammīm} wə·ʾōḵēlå̄ wa·ʾăḇå̄rɛḵəḵå̄ li·p̄nē {\YHWH} li·p̄nē mōṯī\ %
	\verseno{8}~wə·ʿattå̄ ḇənī \textbf{šəmaʿ} bə·qōlī la·ʾăšɛr ʾănī məṣawwå̄ ʾōṯå̄·ḵ\ %
	\verseno{9}~lɛḵ·nå̄ ʾɛl·haṣ·ṣōn wə·qaḥ·lī· miš·šå̄m šənē gəḏå̄yē ʿizzīm ṭōḇīm wə·ʾɛʿɛ̆śɛ ʾōṯå̄·m \textbf{maṭʿammīm} lə·ʾå̄ḇīḵå̄ ka·ʾăšɛr ʾå̄hēḇ\ %
	\verseno{10}~wə·hēḇēṯå̄ lə·ʾå̄ḇīḵå̄ wə·ʾå̄ḵå̄l ba·ʿăḇūr ʾăšɛr yəḇå̄rɛḵəḵå̄ li·p̄nē mōṯō\ %
	\verseno{11}~way·yōmɛr yaʿăqōḇ ʾɛl·riḇqå̄ ʾimmō hēn ʿēśå̄w ʾå̄ḥī ʾīš śå̄ʿīr wə·ʾå̄nōḵī ʾīš ḥå̄lå̄q\ %
	\verseno{12}~ʾūlay \textbf{yəmuššēnī} ʾå̄ḇī wə·hå̄yīṯī ḇə·ʿēnå̄w ki·mṯaʿtēaʿ wə·hēḇēṯī ʿå̄la·y qəlå̄lå̄ wə·lō ḇərå̄ḵå̄\ %
	\verseno{13}~wat·tōmɛr l·ō ʾimmō ʿå̄la·y qīləlå̄ṯḵå̄ bənī ʾaḵ \textbf{šəmaʿ} bə·qōlī wə·lēḵ qaḥ·lī·\ %
	\verseno{14}~way·yēlɛḵ way·yiqqaḥ way·yå̄ḇē lə·ʾimmō wat·taʿaś ʾimmō \textbf{maṭʿammīm} ka·ʾăšɛr ʾå̄hēḇ ʾå̄ḇīw\ %
	\verseno{15}~wat·tiqqaḥ riḇqå̄ ʾɛṯ·biḡḏē ʿēśå̄w bənå̄h hag·gå̄ḏōl ha·ḥămūḏōṯ ʾăšɛr ʾittå̄·h b·ab·bå̄yiṯ wat·talbēš ʾɛṯ·yaʿăqōḇ bənå̄h haq·qå̄ṭå̄n\ %
	\verseno{16}~wə·ʾēṯ ʿōrōṯ gəḏå̄yē hå̄·ʿizzīm hilbīšå̄ ʿal·yå̄ḏå̄w wə·ʿal ḥɛlqaṯ ṣawwå̄rå̄w\ %
	\verseno{17}~wat·tittēn ʾɛṯ·ham·\textbf{maṭʿammīm} wə·ʾɛṯ·hal·lɛḥɛm ʾăšɛr ʿå̄śå̄ṯå̄ bə·yaḏ yaʿăqōḇ bənå̄h\ %
	\verseno{18}~way·yå̄ḇō ʾɛl·ʾå̄ḇīw way·yōmɛr ʾå̄ḇī way·yōmɛr hinnɛnnī· mī ʾattå̄ bənī\ %
	\verseno{19}~way·yōmɛr yaʿăqōḇ ʾɛl·ʾå̄ḇīw ʾå̄nōḵī ʿēśå̄w bəḵōrɛḵå̄ ʿå̄śīṯī ka·ʾăšɛr dibbartå̄ ʾēlå̄·y qūm·nå̄ šəḇå̄ wə·ʾå̄ḵlå̄ miṣ·ṣēḏī ba·ʿăḇūr təḇå̄răḵannī nap̄šɛḵå̄\ %
	\verseno{20}~way·yōmɛr yiṣḥå̄q ʾɛl·bənō ma·zɛ mīhartå̄ li·mṣō bənī way·yōmɛr kī hiqrå̄ {\YHWH} ʾɛ̆lōhɛḵå̄ lə·p̄å̄nå̄y\ %
	\verseno{21}~way·yōmɛr yiṣḥå̄q ʾɛl·yaʿăqōḇ gəšå̄·nå̄ wa·\textbf{ʾămušḵå̄} bənī ha·ʾattå̄ zɛ bənī ʿēśå̄w ʾim·lō\ %
	\verseno{22}~way·yiggaš yaʿăqōḇ ʾɛl·yiṣḥå̄q ʾå̄ḇīw \textbf{waymuššēhū} way·yōmɛr haq·qōl qōl yaʿăqōḇ wə·hay·yå̄ḏayim yəḏē ʿēśå̄w\ %
	\verseno{23}~wə·lō hikkīrō kī·hå̄yū yå̄ḏå̄w kī·ḏē ʿēśå̄w ʾå̄ḥīw śəʿīrōṯ wayḇå̄rḵēhū\ %
	\verseno{24}~way·yōmɛr ʾattå̄ zɛ bənī ʿēśå̄w way·yōmɛr ʾå̄nī\ %
	\verseno{25}~way·yōmɛr haggīšå̄ lī· wə·ʾōḵlå̄ miṣ·ṣēḏ bənī lə·maʿan təḇå̄rɛḵəḵå̄ nap̄šī way·yaggɛš·l·ō way·yōḵal way·yå̄ḇē l·ō yayin way·yēšt\ %
	\verseno{26}~way·yōmɛr ʾēlå̄·w yiṣḥå̄q ʾå̄ḇīw gəšå̄·nå̄ ū·šqå̄·lī· bənī\ %
	\verseno{27}~way·yiggaš way·yiššaq·l·ō way·\textbf{yå̄raḥ} ʾɛṯ·\textbf{rēaḥ} bəḡå̄ḏå̄w wayḇå̄răḵēhū way·yōmɛr rəʾē \textbf{rēaḥ} bənī kə·\textbf{rēaḥ} śå̄ḏɛ ʾăšɛr bērăḵō {\YHWH}
	[…]
}
\ParallelRText{
	\footnotesize\fontspec{Vesper Pro}
	\verseno{1}~A bu wedi heneiddio o Iſaac, a thywyllu ei lygaid fel na \textbf{wele}, alw o honaw ef Eſau ei fâb hynaf, a dywedyd wꝛtho, fy mâb: yntef a ddywedodd wꝛtho ef, wele fi.\ %
	\verseno{2}~Ac efe a ddywedodd wele mi a heneiddiais yn awꝛ, nid adwen ddydd fy marwolaeth.\ %
	\verseno{3}~Ac yn awꝛ cymmer attolwg dy offer, dy gawell ſaethau, a’th fwa, a dos allan i’r maes, a hela i mi helfa.\ %
	\verseno{4}~A gwna i mi \textbf{ddainteithion} oꝛ fâth a garaf, a dŵg [hwynt] attafi, fel y bwytawyf, er mwyn dy fendithio o’m henaid cynn fy marw.\ %
	\verseno{5}~A Rebecca a \textbf{glybu} pan ddywedodd Iſaac wꝛth Eſau ei fâb: ac Eſau aeth i’r maes i hela helfa iw dwyn.\ %
	\verseno{6}~Yna Rebecca a lefarodd wꝛth Iacob ei mâb, gan ddywedyd: wele \textbf{clywais} dy dâd yn llefaru wꝛth Eſau dy frawd gan-ddywedyd.\ %
	\verseno{7}~Dŵg i mi helfa, a gwna di i mi \textbf{ddainteithion}: fel y bwyttawyf, ac i’th fendithiwyf, ger bꝛon yꝛ Arglwydd o flaen fy marw.\ %
	\verseno{8}~Ond yn awꝛ fy mâb \textbf{gwꝛando} ar fy llais i, am yꝛ hynn a oꝛchymynnaf i ti.\ %
	\verseno{9}~Dos yn awꝛ i’r pꝛaidd, a chymmer oddi yno ddau fynn gafr da, a mi ai harlwyaf hwynt yn \textbf{ddainteithion} i’th dâd, fel y câr efe.\ %
	\verseno{10}~A thi ai dygi [hwynt] i’th dâd, fel y bwyttao megis i’th fendithio o flaen ei farw.\ %
	\verseno{11}~Yna y dywedodd Iacob wꝛth Rebecca ei fam, wele Eſau fy mrawd yn ŵꝛ blewoc, a minne yn ŵꝛ llyfn.\ %
	\verseno{12}~Fy nhad ond odid a’m \textbf{teimla}, yna y byddaf yn ei olwg ef fel twyll-wꝛ: felly y dygaf arnaf felldith, ac nid bendith.\ %
	\verseno{13}~Ai fam a ddywedodd wꝛtho ef, arnafi [y byddo] dy felldith fy mâb, yn vnic \textbf{gwꝛando} ar fy llais, dôs, a dŵg i mi.\ %
	\verseno{14}~Ac efe a aeth, ac a gymmerth [y mynnod] ac ai dygodd at ei fam: ai fam a wnaeth \textbf{ddainteithion} fel y care ei dâd.\ %
	\verseno{15}~Rebecca hefyd a gymmerodd hoff wiſcoed Eſau ei mab hynaf, y rhai [oeddynt] gyd a hi yn tŷ, ac a wiſcodd Iacob ei mab ieuangaf.\ %
	\verseno{16}~Gwiſcodd hi hefyd grwyn y mynnod geifr am ei ddwylo ef, ac am lyfndꝛa ei wddf ef.\ %
	\verseno{17}~Ac a roddes y \textbf{dainteithion}, a’r bara y rhai a arlwyaſe hi yn llaw Iacob ei mab.\ %
	\verseno{18}~Ac efe a ddaeth at ei dâd, ac a ddywedodd, fy-nhâd, yntef a ddywedodd, wele fi, pwy ydwyt ti fy mâb’:\ %
	\verseno{19}~Yna y dywedodd Iacob wꝛth ei dâd, my fi [ydwyf] Eſau dy gyntaf-anedic, gwneuthym fel y dywedaiſt wꝛthif: cyfot yn awꝛ, eiſtedd, a bwytta o’m helfa, fel i’m bendithio dy enaid.\ %
	\verseno{20}~Ac Iſaac a addywedodd wꝛth ei fâb, pa fodd fy mab y cefaiſt moꝛr fuan a hynn’: Yntef a ddywedodd, am i’r Arglwydd dy Dduw beri [iddo] ddigwyddo o’m blaen.\ %
	\verseno{21}~Yna y dywedodd Iſaac wꝛth Iacob, tyꝛet yn nes yn awꝛ fel i’th \textbf{deimlwyf} fy mâb: ai ty di [yw] fy mâb Eſau, ai nad e.\ %
	\verseno{22}~Yna y neſſaodd Iacob at Iſaac ei dâd, yntef ai \textbf{teimlodd}, ac a ddywedodd, y llais, [yw] llais Iacob, a’r dwylo, dwylo Eſau [ydynt.]\ %
	\verseno{23}~Ac nid adnabu efe ef, am fod ei ddwylo, fel dwylo ei frawd Eſau, yn flewoc: am hynny efe ai bendithiodd ef.\ %
	\verseno{24}~Dywedodd hefyd, ai ti [ſydd] ymma fy mâb Eſau’: yntef a ddywedodd myfi.\ %
	\verseno{25}~Ac efe a ddywedodd dŵg di attafi fel y bwyttawyf o helfa fy mâb megis i’th fendithio fy enaid: yna y dûg atto ef ac efe a fwyttâodd, dûg iddo ef win hefyd ac efe a yfodd.\ %
	\verseno{26}~Yna y dywedodd Iſaac ei dâd wꝛtho ef tyꝛet ti yn nês yn awꝛ fel i’th guſſanwyf fy mâb.\ %
	\verseno{27}~Yna y daeth efe yn nês, ac yntef ai cuſſanodd ef, ac a \textbf{ſawyꝛodd} \textbf{aroglau} ei wiſcoedd ef, ac ai bendithiodd ef, ac a ddywedodd wele \textbf{aroglau} fy mâb, fel \textbf{arogl} maes yꝛ hwn a fendithiodd yꝛ Arglwydd.
	[…]
}
\end{Parallel}

%\textsuperscript{1}~וַיְהִי֙ כִּֽי־זָקֵ֣ן יִצְחָ֔ק וַתִּכְהֶ֥יןָ עֵינָ֖יו מֵרְאֹ֑ת וַיִּקְרָ֞א אֶת־עֵשָׂ֣ו ׀ בְּנ֣וֹ הַגָּדֹ֗ל וַיֹּ֤אמֶר אֵלָיו֙ בְּנִ֔י וַיֹּ֥אמֶר אֵלָ֖יו הִנֵּֽנִי׃
%\textsuperscript{2}~וַיֹּ֕אמֶר הִנֵּה־נָ֖א זָקַ֑נְתִּי לֹ֥א יָדַ֖עְתִּי י֥וֹם מוֹתִֽי׃
%\textsuperscript{3}~וְעַתָּה֙ שָׂא־נָ֣א כֵלֶ֔יךָ תֶּלְיְךָ֖ וְקַשְׁתֶּ֑ךָ וְצֵא֙ הַשָּׂדֶ֔ה וְצ֥וּדָה לִּ֖י *צידה **צָֽיִד׃
%\textsuperscript{4}~וַעֲשֵׂה־לִ֨י מַטְעַמִּ֜ים כַּאֲשֶׁ֥ר אָהַ֛בְתִּי וְהָבִ֥יאָה לִּ֖י וְאֹכֵ֑לָה בַּעֲב֛וּר תְּבָרֶכְךָ֥ נַפְשִׁ֖י בְּטֶ֥רֶם אָמֽוּת׃
%\textsuperscript{5}~וְרִבְקָ֣ה שֹׁמַ֔עַת בְּדַבֵּ֣ר יִצְחָ֔ק אֶל־עֵשָׂ֖ו בְּנ֑וֹ וַיֵּ֤לֶךְ עֵשָׂו֙ הַשָּׂדֶ֔ה לָצ֥וּד צַ֖יִד לְהָבִֽיא׃
%\textsuperscript{6}~וְרִבְקָה֙ אָֽמְרָ֔ה אֶל־יַעֲקֹ֥ב בְּנָ֖הּ לֵאמֹ֑ר הִנֵּ֤ה שָׁמַ֙עְתִּי֙ אֶת־אָבִ֔יךָ מְדַבֵּ֛ר אֶל־עֵשָׂ֥ו אָחִ֖יךָ לֵאמֹֽר׃
%\textsuperscript{7}~הָבִ֨יאָה לִּ֥י צַ֛יִד וַעֲשֵׂה־לִ֥י מַטְעַמִּ֖ים וְאֹכֵ֑לָה וַאֲבָרֶכְכָ֛ה לִפְנֵ֥י יְהוָ֖ה לִפְנֵ֥י מוֹתִֽי׃
%\textsuperscript{8}~וְעַתָּ֥ה בְנִ֖י שְׁמַ֣ע בְּקֹלִ֑י לַאֲשֶׁ֥ר אֲנִ֖י מְצַוָּ֥ה אֹתָֽךְ׃
%\textsuperscript{9}~לֶךְ־נָא֙ אֶל־הַצֹּ֔אן וְקַֽח־לִ֣י מִשָּׁ֗ם שְׁנֵ֛י גְּדָיֵ֥י עִזִּ֖ים טֹבִ֑ים וְאֶֽעֱשֶׂ֨ה אֹתָ֧ם מַטְעַמִּ֛ים לְאָבִ֖יךָ כַּאֲשֶׁ֥ר אָהֵֽב׃
%\textsuperscript{10}~וְהֵבֵאתָ֥ לְאָבִ֖יךָ וְאָכָ֑ל בַּעֲבֻ֛ר אֲשֶׁ֥ר יְבָרֶכְךָ֖ לִפְנֵ֥י מוֹתֽוֹ׃
%\textsuperscript{11}~וַיֹּ֣אמֶר יַעֲקֹ֔ב אֶל־רִבְקָ֖ה אִמּ֑וֹ הֵ֣ן עֵשָׂ֤ו אָחִי֙ אִ֣ישׁ שָׂעִ֔ר וְאָנֹכִ֖י אִ֥ישׁ חָלָֽק׃
%\textsuperscript{12}~אוּלַ֤י יְמֻשֵּׁ֙נִי֙ אָבִ֔י וְהָיִ֥יתִי בְעֵינָ֖יו כִּמְתַעְתֵּ֑עַ וְהֵבֵאתִ֥י עָלַ֛י קְלָלָ֖ה וְלֹ֥א בְרָכָֽה׃
%\textsuperscript{13}~וַתֹּ֤אמֶר לוֹ֙ אִמּ֔וֹ עָלַ֥י קִלְלָתְךָ֖ בְּנִ֑י אַ֛ךְ שְׁמַ֥ע בְּקֹלִ֖י וְלֵ֥ךְ קַֽח־לִֽי׃
%\textsuperscript{14}~וַיֵּ֙לֶךְ֙ וַיִּקַּ֔ח וַיָּבֵ֖א לְאִמּ֑וֹ וַתַּ֤עַשׂ אִמּוֹ֙ מַטְעַמִּ֔ים כַּאֲשֶׁ֖ר אָהֵ֥ב אָבִֽיו׃
%\textsuperscript{15}~וַתִּקַּ֣ח רִ֠בְקָה אֶת־בִּגְדֵ֨י עֵשָׂ֜ו בְּנָ֤הּ הַגָּדֹל֙ הַחֲמֻדֹ֔ת אֲשֶׁ֥ר אִתָּ֖הּ בַּבָּ֑יִת וַתַּלְבֵּ֥שׁ אֶֽת־יַעֲקֹ֖ב בְּנָ֥הּ הַקָּטָֽן׃
%\textsuperscript{16}~וְאֵ֗ת עֹרֹת֙ גְּדָיֵ֣י הָֽעִזִּ֔ים הִלְבִּ֖ישָׁה עַל־יָדָ֑יו וְעַ֖ל חֶלְקַ֥ת צַוָּארָֽיו׃
%\textsuperscript{17}~וַתִּתֵּ֧ן אֶת־הַמַּטְעַמִּ֛ים וְאֶת־הַלֶּ֖חֶם אֲשֶׁ֣ר עָשָׂ֑תָה בְּיַ֖ד יַעֲקֹ֥ב בְּנָֽהּ׃
%\textsuperscript{18}~וַיָּבֹ֥א אֶל־אָבִ֖יו וַיֹּ֣אמֶר אָבִ֑י וַיֹּ֣אמֶר הִנֶּ֔נִּי מִ֥י אַתָּ֖ה בְּנִֽי׃
%\textsuperscript{19}~וַיֹּ֨אמֶר יַעֲקֹ֜ב אֶל־אָבִ֗יו אָנֹכִי֙ עֵשָׂ֣ו בְּכֹרֶ֔ךָ עָשִׂ֕יתִי כַּאֲשֶׁ֥ר דִּבַּ֖רְתָּ אֵלָ֑י קֽוּם־נָ֣א שְׁבָ֗ה וְאָכְלָה֙ מִצֵּידִ֔י בַּעֲב֖וּר תְּבָרֲכַ֥נִּי נַפְשֶֽׁךָ׃
%\textsuperscript{20}~וַיֹּ֤אמֶר יִצְחָק֙ אֶל־בְּנ֔וֹ מַה־זֶּ֛ה מִהַ֥רְתָּ לִמְצֹ֖א בְּנִ֑י וַיֹּ֕אמֶר כִּ֥י הִקְרָ֛ה יְהוָ֥ה אֱלֹהֶ֖יךָ לְפָנָֽי׃
%\textsuperscript{21}~וַיֹּ֤אמֶר יִצְחָק֙ אֶֽל־יַעֲקֹ֔ב גְּשָׁה־נָּ֥א וַאֲמֻֽשְׁךָ֖ בְּנִ֑י הַֽאַתָּ֥ה זֶ֛ה בְּנִ֥י עֵשָׂ֖ו אִם־לֹֽא׃
%\textsuperscript{22}~וַיִּגַּ֧שׁ יַעֲקֹ֛ב אֶל־יִצְחָ֥ק אָבִ֖יו וַיְמֻשֵּׁ֑הוּ וַיֹּ֗אמֶר הַקֹּל֙ ק֣וֹל יַעֲקֹ֔ב וְהַיָּדַ֖יִם יְדֵ֥י עֵשָֽׂו׃
%\textsuperscript{23}~וְלֹ֣א הִכִּיר֔וֹ כִּֽי־הָי֣וּ יָדָ֗יו כִּידֵ֛י עֵשָׂ֥ו אָחִ֖יו שְׂעִרֹ֑ת וַֽיְבָרְכֵֽהוּ׃
%\textsuperscript{24}~וַיֹּ֕אמֶר אַתָּ֥ה זֶ֖ה בְּנִ֣י עֵשָׂ֑ו וַיֹּ֖אמֶר אָֽנִי׃
%\textsuperscript{25}~וַיֹּ֗אמֶר הַגִּ֤שָׁה לִּי֙ וְאֹֽכְלָה֙ מִצֵּ֣יד בְּנִ֔י לְמַ֥עַן תְּבָֽרֶכְךָ֖ נַפְשִׁ֑י וַיַּגֶּשׁ־לוֹ֙ וַיֹּאכַ֔ל וַיָּ֧בֵא ל֦וֹ יַ֖יִן וַיֵּֽשְׁתְּ׃
%\textsuperscript{26}~וַיֹּ֥אמֶר אֵלָ֖יו יִצְחָ֣ק אָבִ֑יו גְּשָׁה־נָּ֥א וּשְׁקָה־לִּ֖י בְּנִֽי׃
%\textsuperscript{27}~וַיִּגַּשׁ֙ וַיִּשַּׁק־ל֔וֹ וַיָּ֛רַח אֶת־רֵ֥יחַ בְּגָדָ֖יו וַֽיְבָרֲכֵ֑הוּ וַיֹּ֗אמֶר רְאֵה֙ רֵ֣יחַ בְּנִ֔י כְּרֵ֣יחַ שָׂדֶ֔ה אֲשֶׁ֥ר בֵּרֲכ֖וֹ יְהוָֽה׃
