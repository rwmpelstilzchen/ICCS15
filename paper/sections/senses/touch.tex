\subsection{Touch}

\subsubsectionoccurances{\bh{må̄šaš}}{6}

\begin{paper}
	{\click} The verb \bh{må̄šaš} ‘to touch (by hand), to feel’ is translated in most cases using \C{teimlo}, as we’ve seen in the list of modalities in chapter 115 of Psalms. \C{palfalu} or \C{ymbalfalu} are used in verses concerning trying to grasp the environment in darkness or blindness. When in \vref{Judg.}{16}{26}{} Samson asks the lad to let him feel the pillars, Morgan may foreshadow the later event by choosing to translate \bh{hămīšēnī} by \C{gad i mi gael gafel} (=\C{gafael}).

	\begin{leftbar}
		\begin{compactitem}
			\item The danger of English mediation: \bh{nå̄gaʿ}, ‘touch’, and \C{cyffwrdd} ‘touch, meet, adjoin’.
		\end{compactitem}
	\end{leftbar}
\end{paper}

\begin{example}{Gen.}{27}{22}{}{}
	\quoling
	{וַיִּגַּ֧שׁ יַעֲקֹ֛ב אֶל־יִצְחָ֥ק אָבִ֖יו \lhl{וַיְמֻשֵּׁ֑הוּ} וַיֹּ֗אמֶר הַקֹּל֙ ק֣וֹל יַעֲקֹ֔ב וְהַיָּדַ֖יִם יְדֵ֥י עֵשָֽׂו׃}
	{Yna y neſſaodd Iacob at Iſaac ei dâd, yntef ai \lhl{teimlodd}, ac a ddywedodd, y llais, [yw] llais Iacob, a’r dwylo, dwylo Eſau [ydynt.]}
	{way·yiggaš yaʿăqōḇ ʾɛl·yiṣḥå̄q ʾå̄ḇīw \lhl{waymuššēhū} way·yōmɛr haq·qōl qōl yaʿăqōḇ wə·hay·yå̄ḏayim yəḏē ʿēśå̄w}
	{And Jacob went near unto Isaac his father; and he \lhl{felt} him, and said, The voice \kjvit{is} Jacob’s voice, but the hands \kjvit{are} the hands of Esau.}
\end{example}

\begin{example}{Deut.}{28}{29}{}{}
	\quoling
	{וְהָיִ֜יתָ \lhl{מְמַשֵּׁ֣שׁ} בַּֽצָּהֳרַ֗יִם כַּאֲשֶׁ֨ר \lhl{יְמַשֵּׁ֤שׁ} הָעִוֵּר֙ בָּאֲפֵלָ֔ה וְלֹ֥א תַצְלִ֖יחַ אֶת־דְּרָכֶ֑יךָ […]׃}
	{Byddi hefyd yn \lhl{ymbalfalu} ganol dydd, fel yꝛ \lhl{ymbalfale} y dall yn y tywyllwch, ac ni lwyddi yn dy ffyꝛdd di: […]}
	{wə·hå̄yīṯå̄ \lhl{məmaššēš} b·aṣ·ṣå̄hå̄̆rayim ka·ʾăšɛr \lhl{yəmaššēš} hå̄·ʿiwwēr b·å̄·ʾăp̄ēlå̄ wə·lō ṯaṣlīaḥ ʾɛṯ·dərå̄ḵɛḵå̄ […]}
	{And thou shalt \lhl{grope} at noonday, as the blind \lhl{gropeth} in darkness, and thou shalt not prosper in thy ways: […]}
\end{example}

\begin{example}{Judg.}{16}{26}{}{}
	\quoling
	{וַיֹּ֨אמֶר שִׁמְשׁ֜וֹן אֶל־הַנַּ֨עַר הַמַּחֲזִ֣יק בְּיָדוֹ֮ הַנִּ֣יחָה אוֹתִי֒ *והימשני **\lhl{וַהֲמִשֵׁ֙נִי֙} אֶת־הָֽעַמֻּדִ֔ים אֲשֶׁ֥ר הַבַּ֖יִת נָכ֣וֹן עֲלֵיהֶ֑ם וְאֶשָּׁעֵ֖ן עֲלֵיהֶֽם׃}
	{Yna Samſon a ddyweddodd wꝛth y llangc yꝛ hwn oedd yn ymaflyd yn ei law ef, gollwng, a gad i mi gael \lhl{gafel} ar y colofnau y rhai y ſicerhauwyd y tŷ arnynt: fel y pwyſwyf arnynt.}
	{way·yōmɛr šimšōn ʾɛl·han·naʿar ham·maḥăzīq bə·yå̄ḏō hannīḥå̄ ʾōṯī· wa·\lhl{hămīšēnī} ʾɛṯ·hå̄·ʿammūḏīm ʾăšɛr hab·bayiṯ nå̄ḵōn ʿălēhɛm wə·ʾɛššå̄ʿēn ʿălēhɛm}
	{And Samson said unto the lad that held him by the hand, Suffer me that I may \lhl{feel} the pillars whereupon the house standeth, that I may lean upon them.}
\end{example}



\subsubsectiontext{\bh{giššēš}}{One occurance}

\begin{paper}
	The verb \bh{giššēš} has only one occurance in the Bible, which is similar to \bh{må̄šaš} when used for grasping the environment in darkness or blindness; accordingly it is translated by \C{palfalu}.
\end{paper}

\begin{example}{Isa.}{59}{10}{}{}
	\quoling
	{\lhl{נְגַֽשְׁשָׁ֤ה} כַֽעִוְרִים֙ קִ֔יר וּכְאֵ֥ין עֵינַ֖יִם \lhl{נְגַשֵּׁ֑שָׁה} כָּשַׁ֤לְנוּ בַֽצָּהֳרַ֙יִם֙ כַּנֶּ֔שֶׁף בָּאַשְׁמַנִּ֖ים כַּמֵּתִֽים׃}
	{\lhl{Palfalaſom} fell deillion a’r pared, ie fel [rhai] heb lygaid y \lhl{palfalaſom} \typo{palfalafom}: tramgwyddaſom ar hanner dydd fel y cyfnos, [oeddem] ym myſc y beddau fel rhai meirw.}
	{\lhl{nəḡašəšå̄} ḵ·a·ʿiwrīm qīr ū·ḵ·ʾēn ʿēnayim \lhl{nəḡaššēšå̄} kå̄šalnū ḇ·aṣ·ṣå̄hå̄̆rayim k·an·nɛšɛp̄ b·å̄·ʾašmannīm k·am·mēṯīm}
	{We \lhl{grope} for the wall like the blind, and we \lhl{grope} as if \kjvit{we had} no eyes: we stumble at noonday as in the night; \kjvit{we are} in desolate places as dead \kjvit{men}.}
\end{example}
