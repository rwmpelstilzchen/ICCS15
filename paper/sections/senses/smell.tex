\subsection{Smell}

{\click} The sense of smell provides us a less complex situation.



\subsubsectionoccurances[\bh{hērīaḥ}]{\bh{hērīaḥ} ‘to smell’}{11}

\begin{paper}
	\bh{hērīaḥ} ‘to smell’ is usually translated using \C{arogli}, except when the Hebrew idiomatics seems not to be applicable in Welsh; the possibility of interference by a mediating translation is yet to be researched; see exx.~\vref{Judg.}{16}{9}{} and \vref{Isa.}{11}{3}{}.
\end{paper}

\begin{example}{Gen.}{8}{21}{}{}
	\quoling
	{\lhl{וַיָּ֣רַח} יְהוָה֮ אֶת־רֵ֣יחַ הַנִּיחֹחַ֒ וַיֹּ֨אמֶר יְהוָ֜ה אֶל־לִבּ֗וֹ לֹֽא־אֹ֠סִף לְקַלֵּ֨ל ע֤וֹד אֶת־הָֽאֲדָמָה֙ בַּעֲב֣וּר הָֽאָדָ֔ם […]}
	{Yna’r \lhl{aroglodd} yꝛ Arglwydd arogl eſmwyth, a dywedodd yꝛ Arglwydd yn ei galon ni chwanegaf felldithio y ddaiar mwy er mwyn dŷn: […]}
	{way·\lhl{yå̄raḥ} {\YHWH} ʾɛṯ·rēaḥ han·nīḥōaḥ way·yōmɛr {\YHWH} ʾɛl·libbō lō·ʾōsip̄ lə·qallēl ʿōḏ ʾɛṯ·hå̄·ʾăḏå̄må̄ ba·ʿăḇūr hå̄·ʾå̄ḏå̄m […]}
	{And the {\LORD} \lhl{smelled} a sweet savour; and the {\LORD} said in his heart, I will not again curse the ground any more for man’s sake; […]}
\end{example}

\begin{example}{Judg.}{16}{9}{}{}
	\quoling
	{[…] וַיְנַתֵּק֙ אֶת־הַיְתָרִ֔ים כַּאֲשֶׁ֨ר יִנָּתֵ֤ק פְּתִֽיל־הַנְּעֹ֙רֶת֙ \lhl{בַּהֲרִיח֣וֹ} אֵ֔שׁ וְלֹ֥א נוֹדַ֖ע כֹּחֽוֹ׃}
	{[…]: ac efe a doꝛrodd y gwdyn, fel y toꝛrir edef garth wedi \lhl{cyffwꝛdd} [a’r] tân, felly nid adnabuwyd ei gryfder ef.}
	{way·nattēq ʾɛṯ·ha·yṯå̄rīm ka·ʾăšɛr yinnå̄ṯēq pəṯīl·han·nəʿōrɛṯ ba·\lhl{hărīḥō} ʾēš wə·lō nōḏaʿ kōḥō}
	{And he brake the withs, as a thread of tow is broken when it \lhl{toucheth} the fire. So his strength was not known.}
\end{example}
\begin{compactdesc}\small
	\item [Vulgate:] […] quomodo si rumpat quis filum de stuppae tortum putamine cum \lhl{odorem} ignis acceperit […]
	\item [Geneva:] […] as a threede of towe is broken, when it \lhl{feeleth} fire […]
	\item [Bishops’:] […] as a stryng of towe breaketh when it \lhl{fealeth} fire […]
\end{compactdesc}

\begin{example}{Isa.}{11}{3}{}{}
	\quoling
	{\lhl{וַהֲרִיח֖וֹ} בְּיִרְאַ֣ת יְהוָ֑ה וְלֹֽא־לְמַרְאֵ֤ה עֵינָיו֙ יִשְׁפּ֔וֹט וְלֹֽא־לְמִשְׁמַ֥ע אָזְנָ֖יו יוֹכִֽיחַ׃}
	{Ei \lhl{ddeall} hefyd [fydd] yn ofn yꝛ Arglwydd, ac nid wꝛth olwg ei lygaid y barn efe, nac wꝛth glywediad ei gluſtiau y cerydda efe.}
	{wa·\lhl{hărīḥō} bə·yirʾaṯ {\YHWH} wə·lō·lə·marʾē ʿēnå̄w yišpōṭ wə·lō·lə·mišmaʿ ʾå̄znå̄w yōḵīaḥ}
	{And shall make him of quick \lhl{understanding} in the fear of the {\LORD}: and he shall not judge after the sight of his eyes, neither reprove after the hearing of his ears:}
\end{example}
\begin{compactdesc}\small
	\item [Vulgate:] et \lhl{replebit eum spiritus} timoris Domini […]
	\item [Geneva:] And shall make him \lhl{prudent} in the feare of the Lord: […]
	\item [Bishops’:] And shall make hym of \lhl{deepe iudgement} in the feare of God: […]
\end{compactdesc}



\subsubsection{\bh{rēaḥ}}

\begin{paper}
	The noun \bh{rēaḥ} ‘smell’ is translated by \C{arogl}, with \C{aroglau} and \C{arogledd} as variation. The very common collocation \bh{rēaḥ nīḥōaḥ} ‘pleasant smell’, pertaining to the Ancient Hebrew religion, is translated by \C{arogl esmwyth}.
\end{paper}

\begin{example}{Song}{4}{11}{}{}
	\quoling
	{נֹ֛פֶת תִּטֹּ֥פְנָה שִׂפְתוֹתַ֖יִךְ כַּלָּ֑ה דְּבַ֤שׁ וְחָלָב֙ תַּ֣חַת לְשׁוֹנֵ֔ךְ \lhl{וְרֵ֥יחַ} שַׂלְמֹתַ֖יִךְ \lhl{כְּרֵ֥יחַ} לְבָנֽוֹן׃}
	{Dy wefuſau [fy] nyweddi ydynt yn diferu [fel] dil mêl, [y mae] mêl a llaeth tann dy dafod, ac \lhl{arogl} dy wiſcoedd fel \lhl{arogl} Libanus.}
	{nōp̄ɛṯ tiṭṭōp̄nå̄ śip̄ṯōṯayiḵ kallå̄ dəḇaš wə-ḥå̄lå̄ḇ taḥaṯ ləšōnēḵ wə-\lhl{rēaḥ} śalmōṯayiḵ kə-\lhl{rēaḥ} ləḇå̄nōn}
	{Thy lips, O \kjvit{my} spouse, drop \kjvit{as} the honeycomb: honey and milk \kjvit{are} under thy tongue; and the \lhl{smell} of thy garments \kjvit{is} like the \lhl{smell} of Lebanon.}
\end{example}

\begin{example}{Ex.}{29}{18}{}{}
	\quoling
	{הִקְטַרְתָּ֤ אֶת־כָּל־הָאַ֙יִל֙ הַמִּזְבֵּ֔חָה עֹלָ֥ה ה֖וּא לַֽיהוָ֑ה \lhl{רֵ֣יחַ נִיח֔וֹחַ} אִשֶּׁ֥ה לַיהוָ֖ה הֽוּא׃}
	{Felly y lloſci yꝛ hwꝛdd ar yꝛ alloꝛ, poeth offrwm i’r Arglwydd yw: \lhl{arogl eſmwyth} [ac] aberth tanllyd i’r Arglwydd yw.}
	{wə-hiqṭartå̄ ʾɛṯ-kå̄l-hå̄-ʾayil ham-mizbēḥ-å̄ ʿōlå̄ hū la-{\YHWH} \lhl{rēaḥ nīḥōaḥ} ʾiššɛ la-{\YHWH} hū}
	{And thou shalt burn the whole ram upon the altar: it \kjvit{is} a burnt offering unto the {\LORD}: it \kjvit{is} a \lhl{sweet savour}, an offering made by fire unto the {\LORD}.}
\end{example}
