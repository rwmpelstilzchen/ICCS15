\subsection{Sight}

\subsubsection{Overview}

\newcommand{\minor}[1]{{\footnotesize{(#1)}}}
\newcommand{\opp}{\enskip}

\begin{hopoint}
	\begin{tabular}{l@{\quad→\quad}l}
		\bh{rå̄ʾå̄}                 & \C{gweled}\opp:\opp\C{edrych}\opp\minor{:\C{canfod}}\\
		\bh{hɛrʾå̄} (\gram{caus.}) & \C{dangos}\opp\minor{:\C{peri}\opp+ \C{gweled}}\\
		\bh{nirʾå̄} (\gram{pass.}) & \C{ymddangos}\opp\minor{:\C{gweled (\gram{pass.})}}\\
		\bh{på̄nå̄}                 & \C{edrych}\opp:\opp\C{troi}\opp\minor{:\C{dychwelyd}\opp:\opp\C{wynebu}\opp:\opp\C{…})}\\
		\bh{hibbīṭ}               & \C{edrych}\opp\minor{:…}\\
		\bh{ḥå̄zå̄}                 & \C{gweled}\opp\minor{:…}\\
		\bh{ṣå̄p̄å̄}                 & \C{edrych}\opp:\opp\C{disgwil}\opp:\opp\C{gwilio}\opp:\opp\C{craffu}\opp:\opp\C{canfod}\\
		\bh{hišgīaḥ}              & \C{edrych}\\
		\hline
		\bh{rōʾɛ}                 & \C{gweledudd}\\
		\bh{ḥōzɛ}                 & \C{gweledudd}\\
		\bh{ṣōp̄ɛ}                 & \C{gwiliedudd}\opp:\opp\C{gwili-wr}\opp\minor{:\C{disgwil-wr}}\\
	\end{tabular}
\end{hopoint}



\subsubsectiontext[\bh{rå̄ʾå̄}]{\bh{rå̄ʾå̄} ‘to see’}{381 occurances in the Pentateuch}

\begin{paper}
	{\click} Let’s begin with \bh{rå̄ʾå̄} ‘to see’. The majority of occurances are translated by the unmarked \C{gweled}, the minority by \C{edrych} and a few by \C{canfod}. The causative form \bh{hɛrʾå̄} and the passive \bh{nirʾå̄} are translated by \C{dangos} and \C{ymddangos}, respectively, in almost all cases.

	As I’ve said \C{gweled} is used for simple sensory perception, understanding and content, while \C{edrych} is used in the sense of facing, looking at and for, accepting, examining, etc.
\end{paper}


\paragraph{\C{gweled}}

\subparagraphtext{Sensory perception}{most of \C{\faded gweled} occurances}

\begin{example}{Ex.}{10}{23}{}{}
	\quoling
	{לֹֽא־\lhl{רָא֞וּ} אִ֣ישׁ אֶת־אָחִ֗יו וְלֹא־קָ֛מוּ אִ֥ישׁ מִתַּחְתָּ֖יו שְׁלֹ֣שֶׁת יָמִ֑ים וּֽלְכָל־בְּנֵ֧י יִשְׂרָאֵ֛ל הָ֥יָה א֖וֹר בְּמוֹשְׁבֹתָֽם׃}
	{Ni \lhl{wele} neb ei gilydd, ac ni chododd neb oi le dꝛi diwꝛnod: ond yꝛ ydoedd goleuni i holl feibion Iſrael yn eu trigfannau.}
	{lō·\lhl{rå̄ʾū} ʾīš ʾɛṯ·ʾå̄ḥīw wə·lō·qå̄mū ʾīš mit·taḥtå̄w šəlōšɛṯ yå̄mīm ū·l·ḵå̄l·bənē yiśrå̄ʾēl hå̄yå̄ ʾōr bə·mōšḇōṯå̄m}
	{They \lhl{saw} not one another, neither rose any from his place for three days: but all the children of Israel had light in their dwellings.}
\end{example}
\begin{paper}
	\explain Ex.~\vref{Ex.}{10}{23}{} describes the plague of darkness, in which the Egyptians couldn’t \hl{see}.
\end{paper}

\begin{example}{Num.}{13}{33}{}{}
	\quoling
	{וְשָׁ֣ם \lhl{רָאִ֗ינוּ} אֶת־הַנְּפִילִ֛ים בְּנֵ֥י עֲנָ֖ק מִן־הַנְּפִלִ֑ים וַנְּהִ֤י בְעֵינֵ֙ינוּ֙ כַּֽחֲגָבִ֔ים וְכֵ֥ן הָיִ֖ינוּ בְּעֵינֵיהֶֽם׃}
	{Ac yno y \lhl{gwelſom} feibion Anac y cawꝛi [y rhai a ddaethant] o’ꝛ cawꝛi, ac yꝛ oeddem yn ein golwg ein hunain fel ceiliogod rhedyn, ac felly yꝛ oeddem yn eu golwg hwyntau.}
	{wə·šå̄m \lhl{rå̄ʾīnū} ʾɛṯ·han·nəp̄īlīm bənē ʿănå̄q min·han·nəp̄īlīm wan·nəhī ḇə·ʿēnēnū ka·ḥăḡå̄ḇīm wə·ḵēn hå̄yīnū bə·ʿēnēhɛm}
	{And there we \lhl{saw} the giants, the sons of Anak, \kjvit{which come} of the giants: and we were in our own sight as grasshoppers, and so we were in their sight.}
\end{example}
\begin{paper}
	\explain On ex.~\vref{Num.}{13}{33}{} the spies tell account of what they’ve \hl{seen} on the Land of Canaan.
\end{paper}



\subparagraphoccurances[Content]{Content (\bh{kī} ‘that’)}{51}

\begin{paper}
	{\click} When \bh{rå̄ʾå̄} is complemented by a \bh{kī} ‘that’ phrase, only \C{clywed} is selectable.
\end{paper}

\begin{example}{Gen.}{1}{4}{}{}
	\quoling
	{\lhl{וַיַּ֧רְא} אֱלֹהִ֛ים אֶת־הָא֖וֹר \hlA{כִּי}־ט֑וֹב וַיַּבְדֵּ֣ל אֱלֹהִ֔ים בֵּ֥ין הָא֖וֹר וּבֵ֥ין הַחֹֽשֶׁךְ׃}
	{Yna Duw a \lhl{welodd} y goleuni \hlA{mai} dâ [oedd,] a Duw a wahanodd rhwng y goleuni a’r tywyllwch.}
	{way·\lhl{yar} ʾɛ̆lōhīm ʾɛṯ·hå̄·ʾōr \hlA{kī}·ṭōḇ way·yaḇdēl ʾɛ̆lōhīm bēn hå̄·ʾōr ū·ḇēn ha·ḥōšɛḵ}
	{And God saw the light, \hlA{that} \kjvit{it was} good: and God divided the light from the darkness.}
\end{example}

\begin{example}{Gen.}{37}{4}{}{}
	\quoling
	{\lhl{וַיִּרְא֣וּ} אֶחָ֗יו \hlA{כִּֽי}־אֹת֞וֹ אָהַ֤ב אֲבִיהֶם֙ מִכָּל־אֶחָ֔יו וַֽיִּשְׂנְא֖וּ אֹת֑וֹ וְלֹ֥א יָכְל֖וּ דַּבְּר֥וֹ לְשָׁלֹֽם׃}
	{Pan \lhl{welodd} ei frodyꝛ, \hlA{fod} eu tâd \hlA{yn} ei garu ef yn fwy nai holl frodyꝛ: yna hwy ai caſaſant ef, ac ni fedꝛent ymddiddan [ag] ef yn heddychol.}
	{way·\lhl{yirʾū} ʾɛḥå̄w \hlA{kī}·ʾōṯ·ō ʾå̄haḇ ʾăḇīhɛm mik·kå̄l·ʾɛḥå̄w way·yiśnəʾū ʾōṯ·ō wə·lō yå̄ḵlū dabbərō lə·šå̄lōm}
	{And when his brethren \lhl{saw} \hlA{that} their father loved him more than all his brethren, they hated him, and could not speak peaceably unto him.}
\end{example}

\begin{paper}
	{\click}{\click} These are the two uses of \C{gweled}; the uses of \C{edrych} are more diverse:
\end{paper}



\paragraph{\C{edrych}}

\subparagraph{Inspection}

\begin{paper}
	If on \exvref{Num.}{13}{33}{} the spies’ account of what they’ve seen with their eyes is expressed using \C{gweled}, inspecting the land is expressed using \C{edrych}.
\end{paper}

\begin{example}{Num.}{13}{18}{}{}
	\quoling
	{\lhl{וּרְאִיתֶ֥ם} אֶת־הָאָ֖רֶץ מַה־הִ֑וא וְאֶת־הָעָם֙ הַיֹּשֵׁ֣ב עָלֶ֔יהָ הֶחָזָ֥ק הוּא֙ הֲרָפֶ֔ה הַמְעַ֥ט ה֖וּא אִם־רָֽב׃}
	{Ac \lhl{edꝛychwch} y wlad beth yw hi, a’r bobl ſydd yn trigo ynddi, ai cryf, ai gwan, ai llawer [ydynt.]}
	{ū·\lhl{rʾīṯɛm} ʾɛṯ·hå̄·ʾå̄rɛṣ ma·hī wə·ʾɛṯ·hå̄·ʿå̄m hay·yōšēḇ ʿå̄lɛhå̄ hɛ·ḥå̄zå̄q hū hă·rå̄p̄ɛ ha·mʿaṭ hū ʾim·rå̄ḇ}
	{And \lhl{see} the land, what it \kjvit{is}; and the people that dwelleth therein, whether they \kjvit{be} strong or weak, few or many;}
\end{example}

\begin{example}{Num.}{32}{8}{}{}
	\quoling
	{כֹּ֥ה עָשׂ֖וּ אֲבֹתֵיכֶ֑ם בְּשָׁלְחִ֥י אֹתָ֛ם מִקָּדֵ֥שׁ בַּרְנֵ֖עַ \lhl{לִרְא֥וֹת} אֶת־הָאָֽרֶץ׃}
	{Felly y gwnaeth eich tadau, pan anfonais hwynt o Cades Barnea i \lhl{edꝛych} y tîr.}
	{kō ʿå̄śū ʾăḇōṯēḵɛm bə·šå̄lḥī ʾōṯå̄·m miq·qå̄ḏēš barnēaʿ li·\lhl{rʾōṯ} ʾɛṯ·hå̄·ʾå̄rɛṣ}
	{Thus did your fathers, when I sent them from Kadeshbarnea to \lhl{see} the land.}
\end{example}



\subparagraphoccurances[Leprosy]{Priestly examination of leprosy}{32}

\begin{paper}
	The third chapter of \emph{Leviticus} deals with examination and purification of leprosy by priests. This religious, quasi-medical examination is expressed with \bh{rå̄ʾå̄}, which is translated using \C{edrych}.
\end{paper}

\begin{example}{Lev.}{13}{8}{}{}
	\quoling
	{\lhl{וְרָאָה֙} הַכֹּהֵ֔ן וְהִנֵּ֛ה פָּשְׂתָ֥ה הַמִּסְפַּ֖חַת בָּע֑וֹר וְטִמְּא֥וֹ הַכֹּהֵ֖ן צָרַ֥עַת הִֽוא׃}
	{Ac \lhl{edꝛyched} yꝛ offeiriad, ac os lledodd y grammen yn y croen, yna barned yꝛ offeiriad ef yn aflan: gwahan-glwyf yw hwnnw.}
	{wə·\lhl{rå̄ʾå̄} hak·kōhēn wə·hinnē på̄śṯå̄ ham·mispaḥaṯ b·å̄·ʿōr wə·ṭimməʾō hak·kōhēn ṣå̄raʿaṯ hī}
	{And \kjvit{if} the priest \lhl{see} that, behold, the scab spreadeth in the skin, then the priest shall pronounce him unclean: it \kjvit{is} a leprosy.}
\end{example}



\tounfold{ראה והנה}

\tounfold{ישא את עיניו וירא, והנה}

\tounfold{להוסיף עוד של \C{edrych}}

\subsubsectionoccurances[\bh{på̄nå̄}]{\bh{på̄nå̄} ‘to turn, to face, to look’}{135}

\begin{paper}
	Now let’s, well, turn to another verb, \bh{på̄nå̄}, which has the same root of the noun \bh{på̄nīm} ‘a face’. It is translated mostly using \C{edrych} (comparable here with English \textit{to face} to a certain degree) or \C{troi} ‘to turn’. This distinction is not expressed in the original Hebrew text using lexically, and Morgan had to choose in each occurance of \bh{på̄nå̄} whether to use the one or the other, or in fewer cases a completely different verb.
\end{paper}

\begin{paper}
	\C{edrych} is used in these meanings:
	\begin{compactitem}
		\item looking (having one’s face, one’s gaze, set or turned towards something),
		\item accepting or willing, with \C{ar} or \C{am},
		\item expressing a geographical direction, with \C{tua}.
	\end{compactitem}
\end{paper}

\begin{example}{Ex.}{2}{12}{}{}
	\quoling
	{\lhl{וַיִּ֤פֶן} כֹּה֙ וָכֹ֔ה וַיַּ֖רְא כִּ֣י אֵ֣ין אִ֑ישׁ וַיַּךְ֙ אֶת־הַמִּצְרִ֔י וַֽיִּטְמְנֵ֖הוּ בַּחֽוֹל׃}
	{Ac efe a \lhl{edꝛychodd} ymma, ac accw, a phan welodd nad [oedd yno] neb: yna efe a laddodd yꝛ Aiphtiad, ac ai cuddiodd yn y tyfod.}
	{way·\lhl{yīp̄ɛn} kō wå̄·ḵō way·yar kī ʾēn ʾīš way·yaḵ ʾɛṯ·ham·miṣrī wayyiṭmənēhū b·a·ḥōl}
	{And he \lhl{looked} this way and that way, and when he saw that \kjvit{there was} no man, he slew the Egyptian, and hid him in the sand.}
\end{example}

\begin{example}{2 Chr.}{6}{19}{}{}
	\quoling
	{\lhl{וּפָנִ֜יתָ} \hlA{אֶל}־תְּפִלַּ֧ת עַבְדְּךָ֛ וְאֶל־תְּחִנָּת֖וֹ יְהוָ֣ה אֱלֹהָ֑י לִשְׁמֹ֤עַ אֶל־הָרִנָּה֙ וְאֶל־הַתְּפִלָּ֔ה אֲשֶׁ֥ר עַבְדְּךָ֖ מִתְפַּלֵּ֥ל לְפָנֶֽיךָ׃}
	{\lhl{Edꝛych} gan hynny \hlA{ar} weddi dy wâs, ac ar ei ddeiſyfiad ef ô Arglwydd fy Nuw: i wꝛando ar y llêf, ac ar y weddi yꝛ hon y mae dy wâs yn ei gweddio ger dy fron di.}
	{ū·\lhl{p̄å̄nīṯå̄} \hlA{ʾɛl}·təp̄illaṯ ʿaḇdəḵå̄ wə·ʾɛl·təḥinnå̄ṯō {\YHWH} ʾɛ̆lōhå̄y li·šmōaʿ ʾɛl·hå̄·rinnå̄ wə·ʾɛl·hat·təp̄illå̄ ʾăšɛr ʿaḇdəḵå̄ miṯpallēl lə·p̄å̄nɛḵå̄}
	{\lhl{Have respect} therefore \hlA{to} the prayer of thy servant, and to his supplication, O {\LORD} my God, to hearken unto the cry and the prayer which thy servant prayeth before thee:}
\end{example}

\begin{example}{Ezek.}{43}{1}{}{}
	\quoling
	{וַיּוֹלִכֵ֖נִי אֶל־הַשָּׁ֑עַר שַׁ֕עַר אֲשֶׁ֥ר \lhl{פֹּנֶ֖ה} דֶּ֥רֶךְ הַקָּדִֽים׃}
	{Ac efe a’m dug fi i’r poꝛth [ſef] y poꝛth yꝛ hwn ſydd yn \lhl{edꝛych} \hlA{tua} ’r dwyꝛain.}
	{wayyōlīḵēnī ʾɛl·haš·šå̄ʿar šaʿar ʾăšɛr \lhl{pōnɛ} dɛrɛḵ haq·qå̄ḏīm}
	{Afterward he brought me to the gate, \kjvit{even} the gate that \lhl{looketh} \hlA{toward} the east:}
\end{example}

\begin{paper}
	\C{troi} is used in these meanings:
	\begin{compactitem}
		\item turning in the physical sense, commonly adjoined with verbs of motion or expressing motion in itself,
		\item turning, metaphorically, unto other gods.
	\end{compactitem}
\end{paper}

\begin{example}{Gen.}{18}{22}{}{}
	\quoling
	{\lhl{וַיִּפְנ֤וּ} מִשָּׁם֙ הָֽאֲנָשִׁ֔ים וַיֵּלְכ֖וּ סְדֹ֑מָה וְאַ֨בְרָהָ֔ם עוֹדֶ֥נּוּ עֹמֵ֖ד לִפְנֵ֥י יְהוָֽה׃}
	{A [dau] oꝛ gwŷꝛ a \lhl{dꝛoeſant} oddi yno, ac a aethant tua Sodoma, ac Abꝛaham yn ſefyll \typo{fefyll} etto ger bꝛon yꝛ Arglwydd.}
	{way·\lhl{yip̄nū} miš·šå̄m hå̄·ʾănå̄šīm way·yēlḵū səḏōm·å̄ wə·ʾaḇrå̄hå̄m ʿōḏɛnnū ʿōmēḏ li·p̄nē {\YHWH}}
	{And the men \lhl{turned their faces} from thence, and went toward Sodom: but Abraham stood yet before the {\LORD}.}
\end{example}

\begin{example}{Deut.}{31}{18}{}{}
	\quoling
	{וְאָנֹכִ֗י הַסְתֵּ֨ר אַסְתִּ֤יר פָּנַי֙ בַּיּ֣וֹם הַה֔וּא עַ֥ל כָּל־הָרָעָ֖ה אֲשֶׁ֣ר עָשָׂ֑ה כִּ֣י \lhl{פָנָ֔ה} אֶל־אֱלֹהִ֖ים אֲחֵרִֽים׃}
	{Canys myfi gan guddio a guddiaf fy wyneb y dydd hwnnw, am yꝛ holl ddꝛygioni ’r hyn a wnaeth efe, pan \lhl{dꝛôdd} at dduwiau dieithꝛ.}
	{wə·ʾå̄nōḵī hastēr ʾastīr på̄nay b·ay·yōm ha·hū ʿal kå̄l·hå̄·rå̄ʿå̄ ʾăšɛr ʿå̄śå̄ kī \lhl{p̄å̄nå̄} ʾɛl·ʾɛ̆lōhīm ʾăḥērīm}
	{And I will surely hide my face in that day for all the evils which they shall have wrought, in that they are \lhl{turned} unto other gods.}
\end{example}



\subsubsectionoccurances[\bh{hibbīṭ}]{\bh{hibbīṭ} ‘to look’}{69}

Almost all occurances of \bh{hibbīṭ} ‘to look’ are translated by \C{edrych}, in what seems quite an automatic translation.



\subsubsectionoccurances[\bh{ḥå̄zå̄}]{\bh{ḥå̄zå̄} ‘to see, to prophesy’}{53}

\begin{paper}
	{\click} While the Hebrew verb \bh{nibbå̄} ‘to prophesy’ has a narrower meaning, \bh{ḥå̄zå̄} can mean ‘to prophesy’ but also broader ‘to see’. In almost all cases it is translated by \C{gweled}, even when the meaning is obviously not a simple visual ‘seeing’.

	Both \bh{ḥōzɛ} and \bh{rōʾɛ}, meaning ‘a \hl{seer}’ in the sense of ‘a prophet’, are translated using \C{gweledudd} in almost all cases; they are derived from \bh{ḥå̄zå̄} and \bh{rå̄ʿå̄} respectively.
\end{paper}

\begin{example}{Isa.}{2}{1}{}{}
	\quoling
	{הַדָּבָר֙ אֲשֶׁ֣ר \lhl{חָזָ֔ה}	יְשַֽׁעְיָ֖הוּ בֶּן־אָמ֑וֹץ עַל־יְהוּדָ֖ה וִירוּשָׁלִָֽם׃}
	{Y gair yꝛ hwn a \lhl{welodd} Eſay mab Amos am Iuda, ac Ieruſalem.}
	{had·då̄ḇå̄r ʾăšɛr \lhl{ḥå̄zå̄} yəšaʿyå̄hū bɛn·ʾå̄mōṣ ʿal·yəhūḏå̄ wī·rūšå̄lå̄yim}
	{The word that Isaiah the son of Amoz \lhl{saw} concerning Judah and Jerusalem.}
\end{example}
(See also \bcite{Am.}{1}{1}, \bcite{Hab.}{1}{1}, \bcite{Isa.}{1}{1}, \bcite{Isa.}{13}{1} and \bcite{Mic.}{1}{1}.)

\begin{paper}
	I will use \exvref{Isa.}{30}{10}{} to demonstrate an interesting phenomenon. It is well known that the Biblical language, both in prose and poetry, is fond of \hl{parallelisms}, saying the same thing twice in a different manner for rhetoric purposes. In \exvref{Isa.}{30}{10}{} Morgan avoids repetition of \C{gweled} translating both \bh{rå̄ʾå̄} and \bh{ḥå̄zå̄}, and changes  \bh{ḥå̄zå̄} to \C{canfod} in order to retain \hl{lexical differentiation}. He doesn’t always do that, but it is certainly a feature of his translation practice and as such requires scholarly attention and description; see the ‘Further research’ section on the handout for more examples of this kind.
\end{paper}

\begin{example}{Isa.}{30}{10}{}{}
	\quoling
	{אֲשֶׁ֨ר אָמְר֤וּ \hlA{לָֽרֹאִים֙} לֹ֣א \hlA{תִרְא֔וּ} \lhl{וְלַ֣חֹזִ֔ים} לֹ֥א \lhl{תֶחֱזוּ}־לָ֖נוּ נְכֹח֑וֹת דַּבְּרוּ־לָ֣נוּ חֲלָק֔וֹת \lhl{חֲז֖וּ} מַהֲתַלּֽוֹת׃}
	{Y rhai a ddywedaſant wꝛth y rhai a \hlA{welant}, na \hlA{welwch}, ac wꝛth y rhai a \lhl{ganfyddant}, na \lhl{chanfyddwch} i ni gymwyſdꝛa: treuthwch i ni weniaith, \lhl{cenfyddwch} i ni ſiomedigaeth.}
	{ʾăšɛr ʾå̄mrū l·å̄·\hlA{rōʾīm} lō \hlA{ṯirʾū} wə·l·a·\lhl{ḥōzīm} lō \lhl{ṯɛḥɛ̆zū}·lå̄nū nəḵōḥōṯ dabbərū·lå̄nū ḥălå̄qōṯ \lhl{ḥăzū} mahăṯallōṯ}
	{Which say to the \hlA{seers}, \hlA{See} not; and to the \lhl{prophets}, \lhl{Prophesy} not unto us right things, speak unto us smooth things, \lhl{prophesy} deceits:}
\end{example}



\subsubsectionoccurances[\bh{ṣå̄p̄å̄}]{\bh{ṣå̄p̄å̄} ‘to watch’}{36}

\begin{paper}
	{\click} If \bh{rå̄ʾå̄} and \bh{på̄nå̄} show limited lexical distribution to two main lexemes each in the translation, and \bh{hibbīṭ} and\bh{ḥå̄zå̄} show uniformity, \bh{ṣå̄p̄å̄} ‘to watch’ is more finely tuned according to meaning; for example \C{disgwil} when \bh{ṣå̄p̄å̄} is used in the sense of anticipation. \tounfold{להתייחס להביט}

	The derived noun \bh{ṣōp̄ɛ} ‘a watchman’ is translated using \C{gwiliedudd} in most cases.
\end{paper}

\tounfold{להוסיף דוגמאות}

\subsubsectionoccurances{\bh{šå̄r}}{23}
\subsubsectionoccurances{\bh{hišqīp̄}~/ \bh{nišqå̄p̄}}{22}
\subsubsectionoccurances{\bh{på̄qaḥ} + \bh{ʿayin}}{18}
\subsubsectionoccurances{\bh{šå̄zap̄}}{3}



\subsubsectionoccurances{\bh{hišgīaḥ}}{3}

\tounfold{***}

\begin{example}{Song}{2}{9}{}{}
	\quoling
	{דּוֹמֶ֤ה דוֹדִי֙ לִצְבִ֔י א֖וֹ לְעֹ֣פֶר הָֽאַיָּלִ֑ים הִנֵּה־זֶ֤ה עוֹמֵד֙ אַחַ֣ר כָּתְלֵ֔נוּ \lhl{מַשְׁגִּ֙יחַ֙} מִן־הַֽחֲלֹּנ֔וֹת מֵצִ֖יץ מִן־הַֽחֲרַכִּֽים׃}
	{Tebyg yw fy annwylyd i iwꝛch neu lwdn hydd, wele ef yn ſefyll dann ein pared yn \lhl{edꝛych} trwy ’r ffeneſtri, [ac] yn ymddangos trwy ’r pilêrau.}
	{dōmɛ ḏōḏī li-ṣḇī ʾō lə-ʿōp̄ɛr hå̄-ʾayyå̄līm hinnē-zɛ ʿōmēḏ ʾaḥar kå̄ṯlēnū \lhl{mašgīaḥ} min-ha-ḥallōnōṯ mēṣīṣ min-ha-ḥărakkīm}
	{My beloved is like a roe or a young hart: behold, he standeth behind our wall, he \lhl{looketh} forth at the windows, shewing himself through the lattice.}
\end{example}



\begin{paper}
	\subsubsection{Conclusion}

	{\click} So, we’ve seen several types of translation relationships:
	\begin{compactitem}
		\item Most show simple conversion, with very little or no variation:
			\bh{hɛrʾå̄} and \bh{nirʾå̄} (the causative and passive of \bh{rå̄ʾå̄}), \bh{hibbīṭ}, \bh{ḥå̄zå̄}, \bh{hišgīaḥ}, \bh{rōʾɛ} and \bh{ḥōzɛ}.
		\item Meaning-bearing opposition of two main options:
			\bh{rå̄ʾå̄} and \bh{på̄nå̄}.
		\item More variance:
			\bh{ṣå̄p̄å̄}.
	\end{compactitem}

	As you can see in point~X, the main lexical burdon lays on two Welsh verbs, \C{gweled} and \C{edrych}.
\end{paper}
