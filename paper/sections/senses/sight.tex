\subsection{Sight}

\subsubsectionoccurances{\bh{råʾå}}{381}

\begin{paper}
	{\click} Let’s begin with \bh{råʾå} ‘to see’. The majority of occurances are translated by the unmarked \C{gweled}, the minority by \C{edrych} and a few by \C{canfod}. The causative form \bh{hɛrʾå} and the passive \bh{nirʾå} are translated by \C{dangos} and \C{ymddangos}, respectively, in almost all cases.

	As I’ve said \C{gweled} is used for simple sensory perception and content, while \C{edrych} is used in the sense of facing, looking at and for, accepting, examining, etc.
\end{paper}


\paragraph{\C{gweled}}

\subparagraphtext{Sensory perception}{most of \C{\faded gweled} occurances}

\begin{example}{Ex.}{10}{23}{}{}
	\quoling
	{לֹֽא־\lhl{רָא֞וּ} אִ֣ישׁ אֶת־אָחִ֗יו וְלֹא־קָ֛מוּ אִ֥ישׁ מִתַּחְתָּ֖יו שְׁלֹ֣שֶׁת יָמִ֑ים וּֽלְכָל־בְּנֵ֧י יִשְׂרָאֵ֛ל הָ֥יָה א֖וֹר בְּמוֹשְׁבֹתָֽם׃}
	{Ni \lhl{wele} neb ei gilydd, ac ni chododd neb oi le dꝛi diwꝛnod: ond yꝛ ydoedd goleuni i holl feibion Iſrael yn eu trigfannau.}
	{lō·\lhl{råʾū} ʾīš ʾɛṯ·ʾåḥīw wə·lō·qåmū ʾīš mit·taḥtåw šəlōšɛṯ yåmīm ū·l·ḵål·bənē yiśråʾēl håyå ʾōr bə·mōšḇōṯåm}
	{They \lhl{saw} not one another, neither rose any from his place for three days: but all the children of Israel had light in their dwellings.}
\end{example}
\begin{paper}
	\explain Ex.~\vref{Ex.}{10}{23}{} describes the plague of darkness, in which the Egyptians couldn’t \hl{see}.
\end{paper}

\begin{example}{Num.}{13}{33}{}{}
	\quoling
	{וְשָׁ֣ם \lhl{רָאִ֗ינוּ} אֶת־הַנְּפִילִ֛ים בְּנֵ֥י עֲנָ֖ק מִן־הַנְּפִלִ֑ים וַנְּהִ֤י בְעֵינֵ֙ינוּ֙ כַּֽחֲגָבִ֔ים וְכֵ֥ן הָיִ֖ינוּ בְּעֵינֵיהֶֽם׃}
	{Ac yno y \lhl{gwelſom} feibion Anac y cawꝛi [y rhai a ddaethant] o’ꝛ cawꝛi, ac yꝛ oeddem yn ein golwg ein hunain fel ceiliogod rhedyn, ac felly yꝛ oeddem yn eu golwg hwyntau.}
	{wə·šåm \lhl{råʾīnū} ʾɛṯ·han·nəp̄īlīm bənē ʿănåq min·han·nəp̄īlīm wan·nəhī ḇə·ʿēnēnū ka·ḥăḡåḇīm wə·ḵēn håyīnū bə·ʿēnēhɛm}
	{And there we \lhl{saw} the giants, the sons of Anak, \emph{which come} of the giants: and we were in our own sight as grasshoppers, and so we were in their sight.}
\end{example}
\begin{paper}
	\explain On ex.~\vref{Num.}{13}{33}{} the spies tell account of what they’ve \hl{seen} on the Land of Canaan.
\end{paper}



\subparagraphoccurances{Content (\bh{kī} ‘that’)}{51}

\begin{paper}
	{\click} When \bh{råʾå} is complemented by a \bh{kī} ‘that’ phrase, only \C{clywed} is selectable.
\end{paper}

\begin{example}{Gen.}{1}{4}{}{}
	\quoling
	{\lhl{וַיַּ֧רְא} אֱלֹהִ֛ים אֶת־הָא֖וֹר \hlA{כִּי}־ט֑וֹב וַיַּבְדֵּ֣ל אֱלֹהִ֔ים בֵּ֥ין הָא֖וֹר וּבֵ֥ין הַחֹֽשֶׁךְ׃}
	{Yna Duw a \lhl{welodd} y goleuni \hlA{mai} dâ [oedd,] a Duw a wahanodd rhwng y goleuni a’r tywyllwch.}
	{way·\lhl{yar} ʾɛ̆lōhīm ʾɛṯ·hå·ʾōr \hlA{kī}·ṭōḇ way·yaḇdēl ʾɛ̆lōhīm bēn hå·ʾōr ū·ḇēn ha·ḥōšɛḵ}
	{And God saw the light, \hlA{that} \emph{it was} good: and God divided the light from the darkness.}
\end{example}

\begin{example}{Gen.}{37}{4}{}{}
	\quoling
	{\lhl{וַיִּרְא֣וּ} אֶחָ֗יו \hlA{כִּֽי}־אֹת֞וֹ אָהַ֤ב אֲבִיהֶם֙ מִכָּל־אֶחָ֔יו וַֽיִּשְׂנְא֖וּ אֹת֑וֹ וְלֹ֥א יָכְל֖וּ דַּבְּר֥וֹ לְשָׁלֹֽם׃}
	{Pan \lhl{welodd} ei frodyꝛ, \hlA{fod} eu tâd \hlA{yn} ei garu ef yn fwy nai holl frodyꝛ: yna hwy ai caſaſant ef, ac ni fedꝛent ymddiddan [ag] ef yn heddychol.}
	{way·\lhl{yirʾū} ʾɛḥåw \hlA{kī}·ʾōṯ·ō ʾåhaḇ ʾăḇīhɛm mik·kål·ʾɛḥåw way·yiśnəʾū ʾōṯ·ō wə·lō yåḵlū dabbərō lə·šålōm}
	{And when his brethren \lhl{saw} \hlA{that} their father loved him more than all his brethren, they hated him, and could not speak peaceably unto him.}
\end{example}

\begin{paper}
	{\click}{\click} These are the two uses of \C{gweled}; the uses of \C{edrych} are more diverse:
\end{paper}



\paragraph{\C{edrych}}

\subparagraph{Inspection*** \tounfold{ריגול}}

\begin{paper}
	If on \exvref{Num.}{13}{33}{} the spies’ account of what they’ve seen with their eyes is expressed using \C{gweled}, inspecting \tounfold{***} the land is expressed using \C{edrych}.
\end{paper}

\begin{example}{Num.}{13}{18}{}{}
	\quoling
	{\lhl{וּרְאִיתֶ֥ם} אֶת־הָאָ֖רֶץ מַה־הִ֑וא וְאֶת־הָעָם֙ הַיֹּשֵׁ֣ב עָלֶ֔יהָ הֶחָזָ֥ק הוּא֙ הֲרָפֶ֔ה הַמְעַ֥ט ה֖וּא אִם־רָֽב׃}
	{Ac \lhl{edꝛychwch} y wlad beth yw hi, a’r bobl ſydd yn trigo ynddi, ai cryf, ai gwan, ai llawer [ydynt.]}
	{\tounfold{***}}
	{\tounfold{***}}
\end{example}

\begin{example}{Num.}{32}{8}{}{}
	\quoling
	{כֹּ֥ה עָשׂ֖וּ אֲבֹתֵיכֶ֑ם בְּשָׁלְחִ֥י אֹתָ֛ם מִקָּדֵ֥שׁ בַּרְנֵ֖עַ \lhl{לִרְא֥וֹת} אֶת־הָאָֽרֶץ׃}
	{Felly y gwnaeth eich tadau, pan anfonais hwynt o Cades Barnea i \lhl{edꝛych} y tîr.}
	{\tounfold{***}}
	{\tounfold{***}}
\end{example}



\subparagraphoccurances{Priestly examination of leprosy}{32}

\begin{paper}
	The third chapter of \emph{Leviticus} deals with examination and purification of leprosy by priests. This religious, quasi-medical examination is expressed with \bh{råʾå}, which is translated using \C{edrych}.
\end{paper}

\begin{example}{Lev.}{13}{8}{}{}
	\quoling
	{\lhl{וְרָאָה֙} הַכֹּהֵ֔ן וְהִנֵּ֛ה פָּשְׂתָ֥ה הַמִּסְפַּ֖חַת בָּע֑וֹר וְטִמְּא֥וֹ הַכֹּהֵ֖ן צָרַ֥עַת הִֽוא׃}
	{Ac \lhl{edꝛyched} yꝛ offeiriad, ac os lledodd y grammen yn y croen, yna barned yꝛ offeiriad ef yn aflan: gwahan-glwyf yw hwnnw.}
	{wə·\lhl{råʾå} hak·kōhēn wə·hinnē påśṯå ham·mispaḥaṯ b·å·ʿōr wə·ṭimməʾō hak·kōhēn ṣåraʿaṯ hī}
	{And \emph{if} the priest \lhl{see} that, behold, the scab spreadeth in the skin, then the priest shall pronounce him unclean: it \emph{is} a leprosy.}
\end{example}



\tounfold{ראה והנה}

\tounfold{ישא את עיניו וירא, והנה}

\subsubsectionoccurances{\bh{pånå}}{135}

\tounfold{\C{troi} מול \C{edrych} בעיקר, עם כמה לקסמות נוספות}



\subsubsectionoccurances{\bh{ḥåzå}}{53}
\subsubsectionoccurances{\bh{ṣåp̄å}}{36}
\subsubsectionoccurances{\bh{šår}}{23}
\subsubsectionoccurances{\bh{nišqåp̄}~/ \bh{hišqīp̄}}{22}
\subsubsectionoccurances{\bh{påqaḥ} + \bh{ʿayin}}{18}

\subsubsection{***misc}
\paragraphoccurances{ŠZP***}{3}


%isa	30	10 גיוון המבע***\tounfold{***}



\subsubsection{\bh{šåzap̄}}

\tounfold{לעשות טבלה של החפיפה בין דרכי התרגום השונות. לכתוב תפוצה ואולי לתת לזה גם ביטוי גרפי}
