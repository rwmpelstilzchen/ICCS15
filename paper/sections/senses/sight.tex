\subsection{Sight}

\subsubsectionoccurances{\bh{råʾå}}{381}

\begin{paper}
	{\click} Let’s begin with \bh{råʾå} ‘to see’. The majority of occurances are translated by the unmarked \C{gweled}, the minority by \C{edrych} and a few by \C{canfod}. The causative form \bh{hɛrʾå} and the passive \bh{nirʾå} are translated by \C{dangos} and \C{ymddangos}, respectively, in almost all cases.

	As I’ve said \C{gweled} is used for simple sensory perception and content, while \C{edrych} is used in the sense of facing, looking for, accepting, examining etc.
\end{paper}

\tounfold{ראה והנה}

\tounfold{ישא את עיניו וירא, והנה}



\paragraphoccurances{Content (\bh{kī} ‘that’)}{50}

\begin{paper}
	When \bh{råʾå} is complemented by a \bh{kī} ‘that’ phrase, only \C{clywed} is selectable.
\end{paper}

\begin{example}{Gen.}{1}{4}{}{}
	\quoling
	{\lhl{וַיַּ֧רְא} אֱלֹהִ֛ים אֶת־הָא֖וֹר \hlA{כִּי}־ט֑וֹב וַיַּבְדֵּ֣ל אֱלֹהִ֔ים בֵּ֥ין הָא֖וֹר וּבֵ֥ין הַחֹֽשֶׁךְ׃}
	{Yna Duw a \lhl{welodd} y goleuni \hlA{mai} dâ [oedd,] a Duw a wahanodd rhwng y goleuni a’r tywyllwch.}
	{way·\lhl{yar} ʾɛ̆lōhīm ʾɛṯ·hå·ʾōr \hlA{kī}·ṭōḇ way·yaḇdēl ʾɛ̆lōhīm bēn hå·ʾōr ū·ḇēn ha·ḥōšɛḵ}
	{And God saw the light, \hlA{that} \emph{it was} good: and God divided the light from the darkness.}
\end{example}

\begin{example}{Gen.}{37}{4}{}{}
	\quoling
	{\lhl{וַיִּרְא֣וּ} אֶחָ֗יו \hlA{כִּֽי}־אֹת֞וֹ אָהַ֤ב אֲבִיהֶם֙ מִכָּל־אֶחָ֔יו וַֽיִּשְׂנְא֖וּ אֹת֑וֹ וְלֹ֥א יָכְל֖וּ דַּבְּר֥וֹ לְשָׁלֹֽם׃}
	{Pan \lhl{welodd} ei frodyꝛ, \hlA{fod} eu tâd \hlA{yn} ei garu ef yn fwy nai holl frodyꝛ: yna hwy ai caſaſant ef, ac ni fedꝛent ymddiddan [ag] ef yn heddychol.}
	{way·\lhl{yirʾū} ʾɛḥåw \hlA{kī}·ʾōṯ·ō ʾåhaḇ ʾăḇīhɛm mik·kål·ʾɛḥåw way·yiśnəʾū ʾōṯ·ō wə·lō yåḵlū dabbərō lə·šålōm}
	{And when his brethren \lhl{saw} \hlA{that} their father loved him more than all his brethren, they hated him, and could not speak peaceably unto him.}
\end{example}



\paragraphoccurances{Priestly examination of leprosy}{32}

\begin{paper}
	The third chapter of \emph{Leviticus} deals with examination and purification of leprosy by priests. This religious-medical examination is expressed with \bh{råʾå}, which is translated using \C{edrych}.
\end{paper}

\begin{example}{Lev.}{13}{8}{}{}
	\quoling
	{\lhl{וְרָאָה֙} הַכֹּהֵ֔ן וְהִנֵּ֛ה פָּשְׂתָ֥ה הַמִּסְפַּ֖חַת בָּע֑וֹר וְטִמְּא֥וֹ הַכֹּהֵ֖ן צָרַ֥עַת הִֽוא׃}
	{Ac \lhl{edꝛyched} yꝛ offeiriad, ac os lledodd y grammen yn y croen, yna barned yꝛ offeiriad ef yn aflan: gwahan-glwyf yw hwnnw.}
	{wə·\lhl{råʾå} hak·kōhēn wə·hinnē påśṯå ham·mispaḥaṯ b·å·ʿōr wə·ṭimməʾō hak·kōhēn ṣåraʿaṯ hī}
	{And \emph{if} the priest \lhl{see} that, behold, the scab spreadeth in the skin, then the priest shall pronounce him unclean: it \emph{is} a leprosy.}
\end{example}




\subsubsection{\bh{pånå}}



\subsubsection{\bh{nišqåp̄}~/ \bh{hišqīp̄}}



\subsubsection{\bh{šår}}



\subsubsection{\bh{påqaḥ}}



\subsubsection{\bh{ḥåzå}}

%isa	30	10 גיוון המבע***\tounfold{***}



\subsubsection{\bh{šåzap̄}}

\tounfold{לעשות טבלה של החפיפה בין דרכי התרגום השונות. לכתוב תפוצה ואולי לתת לזה גם ביטוי גרפי}
