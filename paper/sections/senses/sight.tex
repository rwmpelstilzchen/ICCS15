\subsection{Sight}

\setcounter{subsubsection}{-1}
\subsubsection{Overview}

\begin{paper}
	Let’s begin with \hl{sight}, surveying the translational relationships of the following verbs:
	\bh{rå̄ʾå̄} and \bh{på̄nå̄}, which are of the aforementioned {\typeA},
	\bh{hibbīṭ}, \bh{ḥå̄zå̄}, \bh{hišqīp̄} and \bh{hišgīaḥ}, which are of {\typeC},
	and	\bh{ṣå̄p̄å̄} and \bh{šå̄r}, {\typeB}.
	%-%and the derived nouns \bh{rōʾɛ}, \bh{ḥōzɛ} and \bh{ṣōp̄ɛ}.
\end{paper}

\newcommand{\minor}[1]{{\footnotesize{(#1)}}}
\newcommand{\opp}{\enskip}

\begin{hopoint}
	\begin{tabular}{l@{\quad→\quad}l}
		\typeA: &\\
		\bh{rå̄ʾå̄}                 & \C{gweled}\opp:\opp\C{edrych}\opp\minor{:\C{canfod}}\\
		\bh{på̄nå̄}                 & \C{edrych}\opp:\opp\C{troi}\opp\minor{:\C{dychwelyd}\opp:\opp\C{wynebu}\opp:\opp\C{…})}\\
		\typeC: &\\
		\bh{hɛrʾå̄} (\gram{caus.}) & \C{dangos}\opp\minor{:\C{peri}~+ \C{gweled}}\\
		\bh{nirʾå̄} (\gram{pass.}) & \C{ymddangos}\opp\minor{:\C{gweled (\gram{pass.})}}\\
		\bh{hibbīṭ}               & \C{edrych}\opp\minor{:…}\\
		\bh{ḥå̄zå̄}                 & \C{gweled}\opp\minor{:…}\\
		\bh{hišqīp̄}               & \C{edrych}\\
		\bh{nišqå̄p̄}               & \C{edrych}\opp:\opp\C{gweled (\gram{pass.})}\\
		\bh{hišgīaḥ}              & \C{edrych}\\
		\bh{(rōʾɛ)}               & \C{gweledudd}\\
		\bh{(ḥōzɛ)}               & \C{gweledudd}\\
		\typeB: &\\
		\bh{ṣå̄p̄å̄}                 & \C{edrych}\opp:\opp\C{disgwil}\opp:\opp\C{gwilio}\opp:\opp\C{craffu}\opp:\opp\C{canfod}\\
		\bh{šå̄r}                  & \C{edrych}\opp:\opp\C{canfod}\opp\minor{:\C{gweled}\opp:\opp{deall}\opp:\opp\C{cyfeirio}}\\
		\hline
		\bh{(ṣōp̄ɛ)}               & \C{gwiliedudd}\opp:\opp\C{gwili-wr}\opp\minor{:\C{disgwil-wr}}\\
	\end{tabular}
\end{hopoint}



\subsubsectiontext[\bh{rå̄ʾå̄}]{\bh{rå̄ʾå̄} ‘to see’ {\small\darkfaded(\typeA\footnote{\bh{rå̄ʾå̄} is {\typeA}; \bh{hɛrʾå̄} and \bh{nirʾå̄} are {\typeC}.})}}{Pentateuch: 381 occ.; total: 1299}

\begin{paper}
	{\click} The majority of occurances of the common verb \bh{rå̄ʾå̄} ‘to see’ are translated by the unmarked \C{gweled}, the minority by \C{edrych} and a few by \C{canfod}.

	As I’ve said \C{gweled} in this opposition is used for simple concrete visual perception and perception of content,
	%-%and for seeing as understanding,
	while \C{edrych} is used in the sense of facing, looking at and looking for, accepting, examining, inspecting, etc.

	The causative form \bh{hɛrʾå̄} and the passive \bh{nirʾå̄} are translated by \C{dangos} and \C{ymddangos}, respectively, in almost all cases.

	Let’s have a look at these three \hl{translation choices} for \C{rå̄ʾå̄}: \C{gweled}, \C{edrych} and \C{canfod}.
\end{paper}



\paragraph{\C{gweled}}

\subparagraphtext{Visual perception}{most of \C{\faded gweled} occurances}\label{sight:rå̄ʾå̄:gweled:sensory}

\begin{example}{Ex.}{10}{23}{}{}
	\quoling
	{לֹֽא־\lhl{רָא֞וּ} אִ֣ישׁ אֶת־אָחִ֗יו וְלֹא־קָ֛מוּ אִ֥ישׁ מִתַּחְתָּ֖יו שְׁלֹ֣שֶׁת יָמִ֑ים וּֽלְכָל־בְּנֵ֧י יִשְׂרָאֵ֛ל הָ֥יָה א֖וֹר בְּמוֹשְׁבֹתָֽם׃}
	{Ni \lhl{wele} neb ei gilydd, ac ni chododd neb oi le dꝛi diwꝛnod: ond yꝛ ydoedd goleuni i holl feibion Iſrael yn eu trigfannau.}
	{lō·\lhl{rå̄ʾū} ʾīš ʾɛṯ·ʾå̄ḥīw wə·lō·qå̄mū ʾīš mit·taḥtå̄w šəlōšɛṯ yå̄mīm ū·l·ḵå̄l·bənē yiśrå̄ʾēl hå̄yå̄ ʾōr bə·mōšḇōṯå̄m}
	{They \lhl{saw} not one another, neither rose any from his place for three days: but all the children of Israel had light in their dwellings.}
\end{example}
\begin{paper}
	\explain Ex.~\vref{Ex.}{10}{23}{} describes the plague of darkness, in which the Egyptians were not able to actually \hl{see}.
\end{paper}

\begin{example}{Num.}{13}{33}{}{}
	\quoling
	{וְשָׁ֣ם \lhl{רָאִ֗ינוּ} אֶת־הַנְּפִילִ֛ים בְּנֵ֥י עֲנָ֖ק מִן־הַנְּפִלִ֑ים וַנְּהִ֤י בְעֵינֵ֙ינוּ֙ כַּֽחֲגָבִ֔ים וְכֵ֥ן הָיִ֖ינוּ בְּעֵינֵיהֶֽם׃}
	{Ac yno y \lhl{gwelſom} feibion Anac y cawꝛi [y rhai a ddaethant] o’ꝛ cawꝛi, ac yꝛ oeddem yn ein golwg ein hunain fel ceiliogod rhedyn, ac felly yꝛ oeddem yn eu golwg hwyntau.}
	{wə·šå̄m \lhl{rå̄ʾīnū} ʾɛṯ·han·nəp̄īlīm bənē ʿănå̄q min·han·nəp̄īlīm wan·nəhī ḇə·ʿēnēnū ka·ḥăḡå̄ḇīm wə·ḵēn hå̄yīnū bə·ʿēnēhɛm}
	{And there we \lhl{saw} the giants, the sons of Anak, \kjvit{which come} of the giants: and we were in our own sight as grasshoppers, and so we were in their sight.}
\end{example}
\begin{paper}
	\explain And on ex.~\vref{Num.}{13}{33}{} the spies tell account of what they’ve \hl{seen} with their own eyes on the Land of Canaan.
\end{paper}



\subparagraphtext[Content]{Content (\bh{kī} ‘that’)}{51 occurances (Pentateuch)}

\begin{paper}
	{\click} When \bh{rå̄ʾå̄} is complemented by a \bh{kī} ‘that’ phrase, only \C{gweled} is selectable.
\end{paper}

\begin{example}{Gen.}{1}{4}{}{}
	\quoling
	{\lhl{וַיַּ֧רְא} אֱלֹהִ֛ים אֶת־הָא֖וֹר \hlA{כִּי}־ט֑וֹב וַיַּבְדֵּ֣ל אֱלֹהִ֔ים בֵּ֥ין הָא֖וֹר וּבֵ֥ין הַחֹֽשֶׁךְ׃}
	{Yna Duw a \lhl{welodd} y goleuni \hlA{mai} dâ [oedd,] a Duw a wahanodd rhwng y goleuni a’r tywyllwch.}
	{way·\lhl{yar} ʾɛ̆lōhīm ʾɛṯ·hå̄·ʾōr \hlA{kī}·ṭōḇ way·yaḇdēl ʾɛ̆lōhīm bēn hå̄·ʾōr ū·ḇēn ha·ḥōšɛḵ}
	{And God saw the light, \hlA{that} \kjvit{it was} good: and God divided the light from the darkness.}
\end{example}

\begin{example}{Gen.}{37}{4}{}{}
	\quoling
	{\lhl{וַיִּרְא֣וּ} אֶחָ֗יו \hlA{כִּֽי}־אֹת֞וֹ אָהַ֤ב אֲבִיהֶם֙ מִכָּל־אֶחָ֔יו וַֽיִּשְׂנְא֖וּ אֹת֑וֹ וְלֹ֥א יָכְל֖וּ דַּבְּר֥וֹ לְשָׁלֹֽם׃}
	{Pan \lhl{welodd} ei frodyꝛ, \hlA{fod} eu tâd \hlA{yn} ei garu ef yn fwy nai holl frodyꝛ: yna hwy ai caſaſant ef, ac ni fedꝛent ymddiddan [ag] ef yn heddychol.}
	{way·\lhl{yirʾū} ʾɛḥå̄w \hlA{kī}·ʾōṯ·ō ʾå̄haḇ ʾăḇīhɛm mik·kå̄l·ʾɛḥå̄w way·yiśnəʾū ʾōṯ·ō wə·lō yå̄ḵlū dabbərō lə·šå̄lōm}
	{And when his brethren \lhl{saw} \hlA{that} their father loved him more than all his brethren, they hated him, and could not speak peaceably unto him.}
\end{example}



\paragraph{\C{edrych}}

\subparagraph{Looking}

\begin{paper}
	Exx.~\vref{Gen.}{13}{14}{}, \vref{Ex.}{3}{4}{} and \vref{Gen.}{24}{63}{} are examples for \bh{rå̄ʾå̄} in the sense of ‘looking’, which is translated by \C{edrych}.
\end{paper}

\begin{example}{Gen.}{13}{14}{}{}
	\quoling
	{[…] שָׂ֣א נָ֤א עֵינֶ֙יךָ֙ \lhl{וּרְאֵ֔ה} מִן־הַמָּק֖וֹם אֲשֶׁר־אַתָּ֣ה שָׁ֑ם צָפֹ֥נָה וָנֶ֖גְבָּה וָקֵ֥דְמָה וָיָֽמָּה׃}
	{[…], cyfod dy lygaid, ac \lhl{edꝛych} o’ꝛ lle yꝛ hwn yꝛ ydwyt ynddo tu a’r gogledd, a’r dehau, a’r dwyꝛain, a’r goꝛllewyn.}
	{[…] śå̄ nå̄ ʿēnɛḵå̄ ū·\lhl{rʾē} min·ham·må̄qōm ʾăšɛr·ʾattå̄ šå̄m ṣå̄p̄ōn·å̄ wå̄·nɛḡb·å̄ wå̄·qēḏm·å̄ wå̄·yå̄mm·å̄}
	{[…], Lift up now thine eyes, and \lhl{look} from the place where thou art northward, and southward, and eastward, and westward:}
\end{example}

\begin{example}{Ex.}{3}{4}{}{}
	\quoling
	{וַיַּ֥רְא יְהוָ֖ה כִּ֣י סָ֣ר \lhl{לִרְא֑וֹת} וַיִּקְרָא֩ אֵלָ֨יו אֱלֹהִ֜ים מִתּ֣וֹךְ הַסְּנֶ֗ה וַיֹּ֛אמֶר מֹשֶׁ֥ה מֹשֶׁ֖ה וַיֹּ֥אמֶר הִנֵּֽנִי׃}
	{Pan welodd yꝛ Arglwydd mai cilio yꝛ oedd efe i \lhl{edꝛych}: yna Duw a alwodd arno o ganol y berth ac a ddywedodd Moſes, Moſes: a dywedodd yntef wele fi.}
	{way·yar {\YHWH} kī så̄r li·\lhl{rʾōṯ} way·yiqrå̄ ʾēlå̄·w ʾɛ̆lōhīm mit·tōḵ has·sənɛ way·yōmɛr mōšɛ mōšɛ way·yōmɛr hinnēnī}
	{And when the {\LORD} saw that he turned aside to see, God called unto him out of the midst of the bush, and said, Moses, Moses. And he said, Here \kjvit{am} I.}
\end{example}

\begin{paper}
	The Hebrew collocation of \bh{rå̄ʾå̄} with \bh{wə-hinnē} expresses looking followed by what has appeared before the one who looks from their point of view. It is translated quite literally, by \C{edrych} and \C{ac wele}. Ex.~\vref{Gen.}{24}{63}{} is representative of this kind.

	\begin{leftbar}
		A syntagm of lifting the eyes (\bh{way·yiśśå̄ ʿēnå̄w}, \C{ac a dderchafodd ei lygaid}) commonly precedes this construction, but it is not obligatory.
	\end{leftbar}
\end{paper}

\begin{example}{Gen.}{24}{63}{}{}
	\quoling
	{וַיֵּצֵ֥א יִצְחָ֛ק לָשׂ֥וּחַ בַּשָּׂדֶ֖ה לִפְנ֣וֹת עָ֑רֶב וַיִּשָּׂ֤א עֵינָיו֙ \lhl{וַיַּ֔רְא} \hlA{וְהִנֵּ֥ה} גְמַלִּ֖ים בָּאִֽים׃}
	{Yna y daeth Iſaac allan i fyfyꝛrio yn y maes ym mîn yꝛ hwyꝛ, ac a dderchafodd ei lygaid, ac a \lhl{edꝛychodd}, \hlA{ac wele} gamelod yn dyfod.}
	{way·yēṣē yiṣḥå̄q lå̄·śūaḥ b·aś·śå̄ḏɛ li·p̄nōṯ ʿå̄rɛḇ way·yiśśå̄ ʿēnå̄w way·\lhl{yar} \hlA{wə·hinnē} ḡəmallīm bå̄ʾīm}
	{And Isaac went out to meditate in the field at the eventide: and he lifted up his eyes, and \lhl{saw}, \hlA{and, behold}, the camels \kjvit{were} coming.}
\end{example}



\subparagraph{Scouting out}

\begin{paper}
	If on \exvref{Num.}{13}{33}{} the spies’ account of what they’ve seen with their eyes is expressed using \C{gweled}, act of \hl{scouting out} the land as a whole is expressed using \C{edrych}.
\end{paper}

\begin{example}{Num.}{13}{18}{}{}
	\quoling
	{\lhl{וּרְאִיתֶ֥ם} אֶת־הָאָ֖רֶץ מַה־הִ֑וא וְאֶת־הָעָם֙ הַיֹּשֵׁ֣ב עָלֶ֔יהָ הֶחָזָ֥ק הוּא֙ הֲרָפֶ֔ה הַמְעַ֥ט ה֖וּא אִם־רָֽב׃}
	{Ac \lhl{edꝛychwch} y wlad beth yw hi, a’r bobl ſydd yn trigo ynddi, ai cryf, ai gwan, ai llawer [ydynt.]}
	{ū·\lhl{rʾīṯɛm} ʾɛṯ·hå̄·ʾå̄rɛṣ ma·hī wə·ʾɛṯ·hå̄·ʿå̄m hay·yōšēḇ ʿå̄lɛhå̄ hɛ·ḥå̄zå̄q hū hă·rå̄p̄ɛ ha·mʿaṭ hū ʾim·rå̄ḇ}
	{And \lhl{see} the land, what it \kjvit{is}; and the people that dwelleth therein, whether they \kjvit{be} strong or weak, few or many;}
\end{example}

\begin{example}{Num.}{32}{8}{}{}
	\quoling
	{כֹּ֥ה עָשׂ֖וּ אֲבֹתֵיכֶ֑ם בְּשָׁלְחִ֥י אֹתָ֛ם מִקָּדֵ֥שׁ בַּרְנֵ֖עַ \lhl{לִרְא֥וֹת} אֶת־הָאָֽרֶץ׃}
	{Felly y gwnaeth eich tadau, pan anfonais hwynt o Cades Barnea i \lhl{edꝛych} y tîr.}
	{kō ʿå̄śū ʾăḇōṯēḵɛm bə·šå̄lḥī ʾōṯå̄·m miq·qå̄ḏēš barnēaʿ li·\lhl{rʾōṯ} ʾɛṯ·hå̄·ʾå̄rɛṣ}
	{Thus did your fathers, when I sent them from Kadeshbarnea to \lhl{see} the land.}
\end{example}

\begin{example}{Gen.}{42}{9}{}{}
	\quoling
	{וַיִּזְכֹּ֣ר יוֹסֵ֔ף אֵ֚ת הַחֲלֹמ֔וֹת אֲשֶׁ֥ר חָלַ֖ם לָהֶ֑ם וַיֹּ֤אמֶר אֲלֵהֶם֙ מְרַגְּלִ֣ים אַתֶּ֔ם \lhl{לִרְא֛וֹת} אֶת־עֶרְוַ֥ת הָאָ֖רֶץ בָּאתֶֽם׃}
	{Ioſeph wꝛth hynny a gofiodd ei freuddwydion, yꝛhai (\sic) a freuddwydiaſe ef ’am danynt hwy, ac a ddywedodd wꝛthynt: ſpiwyꝛ [ydych] chwi, i \lhl{edꝛych} noethder y wlâd y daethoch.}
	{way·yizkōr yōsēp̄ ʾēṯ ha·ḥălōmōṯ ʾăšɛr ḥå̄lam lå̄hɛm way·yōmɛr ʾălēhɛm məraggəlīm ʾattɛm li·\lhl{rʾōṯ} ʾɛṯ·ʿɛrwaṯ hå̄·ʾå̄rɛṣ bå̄ṯɛm}
	{And Joseph remembered the dreams which he dreamed of them, and said unto them, Ye \kjvit{are} spies; to \lhl{see} the nakedness of the land ye are come.}
\end{example}


\subparagraphtext[Leprosy]{Priestly examination of leprosy}{32 occurances (Pentateuch)}

\begin{paper}
	The third chapter of \emph{Leviticus} deals with examination and purification of leprosy by priests. This religious, quasi-medical examination is expressed with \bh{rå̄ʾå̄}, which is translated using \C{edrych}.
\end{paper}

\begin{example}{Lev.}{13}{8}{}{}
	\quoling
	{\lhl{וְרָאָה֙} הַכֹּהֵ֔ן וְהִנֵּ֛ה פָּשְׂתָ֥ה הַמִּסְפַּ֖חַת בָּע֑וֹר וְטִמְּא֥וֹ הַכֹּהֵ֖ן צָרַ֥עַת הִֽוא׃}
	{Ac \lhl{edꝛyched} yꝛ offeiriad, ac os lledodd y grammen yn y croen, yna barned yꝛ offeiriad ef yn aflan: gwahan-glwyf yw hwnnw.}
	{wə·\lhl{rå̄ʾå̄} hak·kōhēn wə·hinnē på̄śṯå̄ ham·mispaḥaṯ b·å̄·ʿōr wə·ṭimməʾō hak·kōhēn ṣå̄raʿaṯ hī}
	{And \kjvit{if} the priest \lhl{see} that, behold, the scab spreadeth in the skin, then the priest shall pronounce him unclean: it \kjvit{is} a leprosy.}
\end{example}



\paragraphtext{\C{canfod}}{14 occurances (Pentateuch)}

\begin{paper}
	The third option for translating \bh{rå̄ʾå̄} is \C{canfod}. It is by far less common, and it is used in narrative. Judging by the verses Morgan chose to use it, it has a subtle sense of coming across someone or something.
\end{paper}

\begin{example}{Gen.}{21}{19}{}{}
	\quoling
	{וַיִּפְקַ֤ח אֱלֹהִים֙ אֶת־עֵינֶ֔יהָ \lhl{וַתֵּ֖רֶא} בְּאֵ֣ר מָ֑יִם וַתֵּ֜לֶךְ וַתְּמַלֵּ֤א אֶת־הַחֵ֙מֶת֙ מַ֔יִם וַתַּ֖שְׁקְ אֶת־הַנָּֽעַר׃}
	{Yna Duw a agoꝛodd ei llygaid hi, a hi a \lhl{ganfu} bydew dwfr, ac hi aeth ac a lanwodd y goſtrel [o’ꝛ] dwfr, ac a ddiododd y llangc.}
	{way·yip̄qaḥ ʾɛ̆lōhīm ʾɛṯ·ʿēnɛhå̄ wat·\lhl{tērɛ} bəʾēr må̄yim wat·tēlɛḵ wat·təmallē ʾɛṯ·ha·ḥēmɛṯ mayim wat·tašq ʾɛṯ·han·nå̄ʿar}
	{And God opened her eyes, and she \lhl{saw} a well of water; and she went, and filled the bottle with water, and gave the lad drink.}
\end{example}

\begin{example}{Ex.}{2}{5}{}{}
	\quoling
	{וַתֵּ֤רֶד בַּת־פַּרְעֹה֙ לִרְחֹ֣ץ עַל־הַיְאֹ֔ר וְנַעֲרֹתֶ֥יהָ הֹלְכֹ֖ת עַל־יַ֣ד הַיְאֹ֑ר \lhl{וַתֵּ֤רֶא} אֶת־הַתֵּבָה֙ בְּת֣וֹךְ הַסּ֔וּף וַתִּשְׁלַ֥ח אֶת־אֲמָתָ֖הּ וַתִּקָּחֶֽהָ‪[1]‬}
	{Yna merch Pharao a ddaeth i wared i’r afon i ymolchi, (ai llangceſau oeddynt yn rhodio ger llaw ’r afon:) a hi a \lhl{ganfu} y cawell ynghanol yꝛ heſc ac a anfonodd ei llaw-foꝛwyn iw gyꝛchu.}
	{wat·tērɛḏ baṯ·parʿō li·rḥōṣ ʿal·ha·yʾōr wə·naʿărōṯɛhå̄ hōlḵōṯ ʿal·yaḏ ha·yʾōr wat·\lhl{tērɛ} ʾɛṯ·hat·tēḇå̄ bə·ṯōḵ has·sūp̄ wat·tišlaḥ ʾɛṯ·ʾămå̄ṯå̄h wattiqqå̄ḥɛhå̄}
	{And the daughter of Pharaoh came down to wash \kjvit{herself} at the river; and her maidens walked along by the river's side; and when she \lhl{saw} the ark among the flags, she sent her maid to fetch it.}
\end{example}



\subsubsectionoccurances[\bh{på̄nå̄}]{\bh{på̄nå̄} ‘to turn, to face, to look’\typeAsec}{135}

\begin{paper}
	Now let’s, well, turn to another verb, \bh{på̄nå̄}, which has the same root of the noun \bh{på̄nīm} ‘a face’. It is translated mostly using \C{troi} ‘to turn’ or \C{edrych}. This distinction is not expressed in the original Hebrew text lexically, and Morgan had to choose in each occurance of \bh{på̄nå̄} whether to use the one or the other, or in fewer cases a completely different verb.
\end{paper}

\begin{paper}
	\C{edrych} is used in these meanings:
	\begin{compactitem}
		\item looking (having one’s face, one’s gaze, set or turned towards something),
		\item accepting or willing, with \C{ar} or \C{am},
		\item and expressing a geographical direction, with \C{tua}.
	\end{compactitem}
\end{paper}

\begin{example}{Ex.}{2}{12}{}{}
	\quoling
	{\lhl{וַיִּ֤פֶן} כֹּה֙ וָכֹ֔ה וַיַּ֖רְא כִּ֣י אֵ֣ין אִ֑ישׁ וַיַּךְ֙ אֶת־הַמִּצְרִ֔י וַֽיִּטְמְנֵ֖הוּ בַּחֽוֹל׃}
	{Ac efe a \lhl{edꝛychodd} ymma, ac accw, a phan welodd nad [oedd yno] neb: yna efe a laddodd yꝛ Aiphtiad, ac ai cuddiodd yn y tyfod.}
	{way·\lhl{yīp̄ɛn} kō wå̄·ḵō way·yar kī ʾēn ʾīš way·yaḵ ʾɛṯ·ham·miṣrī wayyiṭmənēhū b·a·ḥōl}
	{And he \lhl{looked} this way and that way, and when he saw that \kjvit{there was} no man, he slew the Egyptian, and hid him in the sand.}
\end{example}

\begin{example}{2 Chr.}{6}{19}{}{}
	\quoling
	{\lhl{וּפָנִ֜יתָ} \hlA{אֶל}־תְּפִלַּ֧ת עַבְדְּךָ֛ וְאֶל־תְּחִנָּת֖וֹ יְהוָ֣ה אֱלֹהָ֑י לִשְׁמֹ֤עַ אֶל־הָרִנָּה֙ וְאֶל־הַתְּפִלָּ֔ה אֲשֶׁ֥ר עַבְדְּךָ֖ מִתְפַּלֵּ֥ל לְפָנֶֽיךָ׃}
	{\lhl{Edꝛych} gan hynny \hlA{ar} weddi dy wâs, ac ar ei ddeiſyfiad ef ô Arglwydd fy Nuw: i wꝛando ar y llêf, ac ar y weddi yꝛ hon y mae dy wâs yn ei gweddio ger dy fron di.}
	{ū·\lhl{p̄å̄nīṯå̄} \hlA{ʾɛl}·təp̄illaṯ ʿaḇdəḵå̄ wə·ʾɛl·təḥinnå̄ṯō {\YHWH} ʾɛ̆lōhå̄y li·šmōaʿ ʾɛl·hå̄·rinnå̄ wə·ʾɛl·hat·təp̄illå̄ ʾăšɛr ʿaḇdəḵå̄ miṯpallēl lə·p̄å̄nɛḵå̄}
	{\lhl{Have respect} therefore \hlA{to} the prayer of thy servant, and to his supplication, O {\LORD} my God, to hearken unto the cry and the prayer which thy servant prayeth before thee:}
\end{example}

\begin{example}{Ezek.}{43}{1}{}{}
	\quoling
	{וַיּוֹלִכֵ֖נִי אֶל־הַשָּׁ֑עַר שַׁ֕עַר אֲשֶׁ֥ר \lhl{פֹּנֶ֖ה} דֶּ֥רֶךְ הַקָּדִֽים׃}
	{Ac efe a’m dug fi i’r poꝛth [ſef] y poꝛth yꝛ hwn ſydd yn \lhl{edꝛych} \hlA{tua} ’r dwyꝛain.}
	{wayyōlīḵēnī ʾɛl·haš·šå̄ʿar šaʿar ʾăšɛr \lhl{pōnɛ} dɛrɛḵ haq·qå̄ḏīm}
	{Afterward he brought me to the gate, \kjvit{even} the gate that \lhl{looketh} \hlA{toward} the east:}
\end{example}

\begin{paper}
	\C{troi}, on the other hand, is used in these meanings:
	\begin{compactitem}
		\item turning in the physical sense, commonly adjoined with verbs of motion or expressing motion in itself,
		\item and turning, metaphorically, unto other gods.
	\end{compactitem}
\end{paper}

\begin{example}{Gen.}{18}{22}{}{}
	\quoling
	{\lhl{וַיִּפְנ֤וּ} מִשָּׁם֙ הָֽאֲנָשִׁ֔ים וַיֵּלְכ֖וּ סְדֹ֑מָה וְאַ֨בְרָהָ֔ם עוֹדֶ֥נּוּ עֹמֵ֖ד לִפְנֵ֥י יְהוָֽה׃}
	{A [dau] oꝛ gwŷꝛ a \lhl{dꝛoeſant} oddi yno, ac a aethant tua Sodoma, ac Abꝛaham yn ſefyll \typo{fefyll} etto ger bꝛon yꝛ Arglwydd.}
	{way·\lhl{yip̄nū} miš·šå̄m hå̄·ʾănå̄šīm way·yēlḵū səḏōm·å̄ wə·ʾaḇrå̄hå̄m ʿōḏɛnnū ʿōmēḏ li·p̄nē {\YHWH}}
	{And the men \lhl{turned their faces} from thence, and went toward Sodom: but Abraham stood yet before the {\LORD}.}
\end{example}

\begin{example}{Deut.}{31}{18}{}{}
	\quoling
	{וְאָנֹכִ֗י הַסְתֵּ֨ר אַסְתִּ֤יר פָּנַי֙ בַּיּ֣וֹם הַה֔וּא עַ֥ל כָּל־הָרָעָ֖ה אֲשֶׁ֣ר עָשָׂ֑ה כִּ֣י \lhl{פָנָ֔ה} אֶל־אֱלֹהִ֖ים אֲחֵרִֽים׃}
	{Canys myfi gan guddio a guddiaf fy wyneb y dydd hwnnw, am yꝛ holl ddꝛygioni ’r hyn a wnaeth efe, pan \lhl{dꝛôdd} at dduwiau dieithꝛ.}
	{wə·ʾå̄nōḵī hastēr ʾastīr på̄nay b·ay·yōm ha·hū ʿal kå̄l·hå̄·rå̄ʿå̄ ʾăšɛr ʿå̄śå̄ kī \lhl{p̄å̄nå̄} ʾɛl·ʾɛ̆lōhīm ʾăḥērīm}
	{And I will surely hide my face in that day for all the evils which they shall have wrought, in that they are \lhl{turned} unto other gods.}
\end{example}



\subsubsectionoccurances[\bh{hibbīṭ}]{\bh{hibbīṭ} ‘to look’\typeCsec}{69}

\begin{paper}
	Almost all occurances of \bh{hibbīṭ} are translated by \C{edrych}, in what seems quite an automatic translation.
\end{paper}

\begin{example}{Prov.}{4}{25}{}{}
	\quoling
	{עֵ֭ינֶיךָ לְנֹ֣כַח \lhl{יַבִּ֑יטוּ} וְ֝עַפְעַפֶּ֗יךָ יַיְשִׁ֥רוּ נֶגְדֶּֽךָ׃}
	{\lhl{Edꝛyched} dy lygaid ar yꝛ iniawn, a chyfarwydded amrantau dy lygaid dy ffoꝛdd o’th flaen.}
	{ʿēnɛḵå̄ lə·nōḵaḥ \lhl{yabbīṭū} wə·ʿap̄ʿappɛḵå̄ yayšīrū nɛḡdɛḵå̄}
	{Let thine eyes \lhl{look} right on, and let thine eyelids look straight before thee.}
\end{example}



\subsubsectionoccurances[\bh{ḥå̄zå̄}]{\bh{ḥå̄zå̄} ‘to see, to prophesy’\typeCsec}{53}\label{sight:ḥå̄zå̄}

\begin{paper}
	{\click} While the Hebrew verb \bh{nibbå̄} ‘to prophesy’ has a narrower meaning, \bh{ḥå̄zå̄} can mean ‘to prophesy’ but also more broadly ‘to see’. Interestingly, in almost all cases it is automatically translated by \C{gweled}, extending the sense of seeing to prophetic sight.

	Both derived nouns \bh{ḥōzɛ} and \bh{rōʾɛ}, which mean ‘a \hl{seer}’ in the sense of ‘a prophet’, are translated using \C{gweledudd} in almost all cases.
\end{paper}

\begin{example}{Isa.}{2}{1}{}{}
	\quoling
	{הַדָּבָר֙ אֲשֶׁ֣ר \lhl{חָזָ֔ה}	יְשַֽׁעְיָ֖הוּ בֶּן־אָמ֑וֹץ עַל־יְהוּדָ֖ה וִירוּשָׁלִָֽם׃}
	{Y gair yꝛ hwn a \lhl{welodd} Eſay mab Amos am Iuda, ac Ieruſalem.}
	{had·då̄ḇå̄r ʾăšɛr \lhl{ḥå̄zå̄} yəšaʿyå̄hū bɛn·ʾå̄mōṣ ʿal·yəhūḏå̄ wī·rūšå̄lå̄yim}
	{The word that Isaiah the son of Amoz \lhl{saw} concerning Judah and Jerusalem.}
\end{example}
(See also \bcite{Am.}{1}{1}, \bcite{Hab.}{1}{1}, \bcite{Isa.}{1}{1}, \bcite{Isa.}{13}{1} and \bcite{Mic.}{1}{1}.)




\subsubsectionoccurances[\bh{hišqīp̄}~/ \bh{nišqå̄p̄}]{\bh{hišqīp̄}~/ \bh{nišqå̄p̄} ‘to view, to look from above’\typeCsec}{22}

\begin{paper}
	The \bh{hišqīp̄} is generally translated using \C{edrych}, which is expectable. \tounfold{לסייג; מורפוסינטקס}

	\begin{leftbar}
		Three occurances of impersonal \C{gwelir} translate \bh{nišqå̄p̄}, which is morphologically a passive form although is also used actively.
	\end{leftbar}
\end{paper}

\bh{hišqīp̄} (active form):
\begin{example}{2 Kings}{9}{30}{}{}
	\quoling
	{וַיָּב֥וֹא יֵה֖וּא יִזְרְעֶ֑אלָה וְאִיזֶ֣בֶל שָׁמְעָ֗ה וַתָּ֨שֶׂם בַּפּ֤וּךְ עֵינֶ֙יהָ֙ וַתֵּ֣יטֶב אֶת־רֹאשָׁ֔הּ \lhl{וַתַּשְׁקֵ֖ף} בְּעַ֥ד הַחַלּֽוֹן׃}
	{Pan ddaeth Iehu i Iezrahel: yna Iezabel a glybu [hynny,] ac a golurodd ei hwyneb, ac a wiſcodd yn wŷch am ei phenn, ac a \lhl{edꝛychodd} dꝛwy ’r ffeneſtr.}
	{way·yå̄ḇō yēhū yizrəʿɛl·å̄ wə·ʾīzɛḇɛl šå̄mʿå̄ wat·tå̄śɛm b·ap·pūḵ ʿēnɛhå̄ wat·tēṭɛḇ ʾɛṯ·rōšå̄h wat·\lhl{tašqēp̄} bəʿaḏ ha·ḥallōn}
	{And when Jehu was come to Jezreel, Jezebel heard \kjvit{of it}; and she painted her face, and tired her head, and \lhl{looked out} at a window.}
\end{example}

\bh{nišqå̄p̄} (morphologically a passive form, altough it is used in a manner similar to active \C{hišqīp̄}):
\begin{example}{Prov.}{7}{6}{}{}
	\quoling
	{כִּ֭י בְּחַלּ֣וֹן בֵּיתִ֑י בְּעַ֖ד אֶשְׁנַבִּ֣י \lhl{נִשְׁקָֽפְתִּי}׃}
	{Canys a mi yn ffeneſtr fy nhŷ, mi a \lhl{edꝛychais} dꝛwy fy ffeneſtr.}
	{kī bə·ḥallōn bēṯī bəʿaḏ ʾɛšnabbī \lhl{nišqå̄p̄tī}}
	{For at the window of my house I \lhl{looked} through my casement,}
\end{example}

Compare this:
\begin{example}{Num.}{21}{20}{}{}
	\quoling
	{וּמִבָּמ֗וֹת הַגַּיְא֙ אֲשֶׁר֙ בִּשְׂדֵ֣ה מוֹאָ֔ב רֹ֖אשׁ הַפִּסְגָּ֑ה \lhl{וְנִשְׁקָ֖פָה} עַל־פְּנֵ֥י הַיְשִׁימֹֽן׃}
	{Ac o Bamoth [yn] y dyffryn, yꝛ hwn [ſydd] yng-wlad Moab, i ben y bꝛyn, yꝛ hwn ſydd yn \lhl{edꝛych} tua yꝛ diffaethwch.}
	{ū·mib·bå̄mōṯ hag·gay ʾăšɛr bi·śḏē mōʾå̄ḇ rōš hap·pisgå̄ wə·\lhl{nišqå̄p̄å̄} ʿal·pənē ha·yšīmōn}
	{And from Bamoth \lhl{in} the valley, that \lhl{is} in the country of Moab, to the top of Pisgah, which looketh toward Jeshimon.}
\end{example}

With this (in three verses \C{welir} translates \bh{nišqå̄p̄}):
\begin{example}{Num.}{23}{28}{}{}
	\quoling
	{וַיִּקַּ֥ח בָּלָ֖ק אֶת־בִּלְעָ֑ם רֹ֣אשׁ הַפְּע֔וֹר \lhl{הַנִּשְׁקָ֖ף} עַל־פְּנֵ֥י הַיְשִׁימֹֽן׃}
	{A Balac a gymmerodd Balaam i ben Peoꝛ, yꝛ hwn a \lhl{welir} tua ’r diffaethwch.}
	{way·yiqqaḥ bå̄lå̄q ʾɛṯ·bilʿå̄m rōš hap·pəʿōr han·\lhl{nišqå̄p̄} ʿal·pənē ha·yšīmōn}
	{And Balak brought Balaam unto the top of Peor, that \lhl{looketh} toward Jeshimon.}
\end{example}



\subsubsectionoccurances[\bh{hišgīaḥ}]{\bh{hišgīaḥ}\typeCsec}{3}

\begin{example}{Song}{2}{9}{}{}
	\quoling
	{דּוֹמֶ֤ה דוֹדִי֙ לִצְבִ֔י א֖וֹ לְעֹ֣פֶר הָֽאַיָּלִ֑ים הִנֵּה־זֶ֤ה עוֹמֵד֙ אַחַ֣ר כָּתְלֵ֔נוּ \lhl{מַשְׁגִּ֙יחַ֙} מִן־הַֽחֲלֹּנ֔וֹת מֵצִ֖יץ מִן־הַֽחֲרַכִּֽים׃}
	{Tebyg yw fy annwylyd i iwꝛch neu lwdn hydd, wele ef yn ſefyll dann ein pared yn \lhl{edꝛych} trwy ’r ffeneſtri, [ac] yn ymddangos trwy ’r pilêrau.}
	{dōmɛ ḏōḏī li·ṣḇī ʾō lə·ʿōp̄ɛr hå̄·ʾayyå̄līm hinnē·zɛ ʿōmēḏ ʾaḥar kå̄ṯlēnū \lhl{mašgīaḥ} min·ha·ḥallōnōṯ mēṣīṣ min·ha·ḥărakkīm}
	{My beloved is like a roe or a young hart: behold, he standeth behind our wall, he \lhl{looketh} forth at the windows, shewing himself through the lattice.}
\end{example}



\subsubsectionoccurances[\bh{ṣå̄p̄å̄}]{\bh{ṣå̄p̄å̄} ‘to watch’\typeBsec}{36}

\begin{paper}
	{\click} If \bh{rå̄ʾå̄} and \bh{på̄nå̄} show limited lexical dispersion in the translation, to two just main lexemes each (\typeA), and \bh{hibbīṭ}, \bh{ḥå̄zå̄}, \bh{hišqīp̄} and \bh{hišgīaḥ} show a high degree of uniformity (\typeC), \bh{ṣå̄p̄å̄} ‘to watch’ and \bh{šå̄r} ‘to behold’ are more finely tuned according to meaning (\typeB); for example \C{disgwil} is used in the sense of anticipation (\exvref{Lam.}{4}{17}{}) and negated \C{cael} is added to \C{gweled} when describing inability to see because of concealment (\exvref{Job}{24}{15}{}).

	The derived noun \bh{ṣōp̄ɛ} ‘a watchman’ is translated using \C{gwiliedudd} in most cases.
\end{paper}

\begin{example}{Gen.}{31}{49}{}{}
	\quoling
	{\hlA{וְהַמִּצְפָּה֙} אֲשֶׁ֣ר אָמַ֔ר \lhl{יִ֥צֶף} יְהוָ֖ה בֵּינִ֣י וּבֵינֶ֑ךָ כִּ֥י נִסָּתֵ֖ר אִ֥ישׁ מֵרֵעֵֽהוּ׃}
	{A \hlA{Miſpah} hefyd, o blegit efe a ddywedaſe, \lhl{edꝛyched} yꝛ Arglwydd rhwng o fi a thithe, pan fôm ni bob vn o olwg ei gilydd.}
	{wə·ham·\hlA{miṣpå̄} ʾăšɛr ʾå̄mar \lhl{yīṣɛp̄} {\YHWH} bēnī ū·ḇēnɛḵå̄ kī nisså̄ṯēr ʾīš mē·rēʿēhū}
	{And \hlA{Mizpah}; for he said, The {\LORD} \lhl{watch} between me and thee, when we are absent one from another.}
\end{example}

\begin{example}{Lam.}{4}{17}{}{}
	\quoling
	{*עודינה **עוֹדֵ֙ינוּ֙ תִּכְלֶ֣ינָה עֵינֵ֔ינוּ אֶל־עֶזְרָתֵ֖נוּ הָ֑בֶל \lhl{בְּצִפִּיָּתֵ֣נוּ} \lhl{צִפִּ֔ינוּ} אֶל־גּ֖וֹי לֹ֥א יוֹשִֽׁעַ׃}
	{Ein llygaid a ballaſant tra yꝛ oeddem ni yn \lhl{diſgwil} am ein cynnoꝛthwy ofer: \lhl{diſgwil} yꝛ oeddem ni yn vnic wꝛth genhedlaeth yꝛ hon nid achube.}
	{ʿōḏēnū tiḵlɛnå̄ ʿēnēnū ʾɛl·ʿɛzrå̄ṯēnū hå̄ḇɛl bə·\lhl{ṣippiyyå̄ṯēnū} \lhl{ṣippīnū} ʾɛl·gōy lō yōšīaʿ}
	{As for us, our eyes as yet failed for our vain help: in our watching we have watched for a nation \kjvit{that} could not save us.}
\end{example}

\begin{example}{2 Sam.}{18}{24}{}{}
	\quoling
	{וְדָוִ֥ד יוֹשֵׁ֖ב בֵּין־שְׁנֵ֣י הַשְּׁעָרִ֑ים וַיֵּ֨לֶךְ \lhl{הַצֹּפֶ֜ה} אֶל־גַּ֤ג הַשַּׁ֙עַר֙ אֶל־הַ֣חוֹמָ֔ה וַיִּשָּׂ֤א אֶת־עֵינָיו֙ וַיַּ֔רְא וְהִנֵּה־אִ֖ישׁ רָ֥ץ לְבַדּֽוֹ׃}
	{A Dafydd oedd yn eiſtedd rhwng y ddau boꝛth: a’r \lhl{gwiliedudd} aeth ar nēn y poꝛth ar y mûr, ac a dderchafodd ei lygaid, ac a edꝛychodd, ac wele ŵꝛ yn rhedeg ei hunan.}
	{wə·ḏå̄wīḏ yōšēḇ bēn·šənē haš·šəʿå̄rīm way·yēlɛḵ haṣ·\lhl{ṣōp̄ɛ} ʾɛl·gaḡ haš·šaʿar ʾɛl·ha·ḥōmå̄ way·yiśśå̄ ʾɛṯ·ʿēnå̄w way·yar wə·hinnē·ʾīš rå̄ṣ lə·ḇaddō}
	{And David sat between the two gates: and the \lhl{watchman} went up to the roof over the gate unto the wall, and lifted up his eyes, and looked, and behold a man running alone.}
\end{example}


\subsubsectionoccurances[\bh{šå̄r}]{\bh{šå̄r} ‘to behold’\typeBsec}{17}

\begin{example}{Job}{24}{15}{}{}
	\quoling
	{וְעֵ֤ין נֹאֵ֨ף ׀ שָׁ֤מְרָֽה נֶ֣שֶׁף לֵ֭אמֹר לֹא־\lhl{תְשׁוּרֵ֣נִי} עָ֑יִן וְסֵ֖תֶר פָּנִ֣ים יָשִֽׂים׃}
	{A llygad y godineb-wꝛ ſydd yn gwilied y cyfnos gan ddywedyd, ni \hlA{chaiff} llygad fyng-\lhl{weled}: ac efe a eſyd hûg ar ei wyneb.}
	{wə·ʿēn nōʾēp̄ šå̄mrå̄ nɛšɛp̄ lē·mōr lō·\lhl{ṯəšūrēnī} ʿå̄yin wə·sēṯɛr på̄nīm yå̄śīm}
	{The eye also of the adulterer waiteth for the twilight, saying, No eye shall \lhl{see} me: and disguiseth his face.}
\end{example}



%-%\subsubsectionoccurances{\bh{på̄qaḥ} + \bh{ʿayin}}{18}
%-%\subsubsectionoccurances{\bh{šå̄zap̄}}{3}



\subsubsection{Conclusion}

\begin{paper}
	{\click} So, we’ve seen several types of translational relationships:
\end{paper}

\begin{hopoint}
	Types of translational relationships:
	\begin{compactitem}
		\item Most show simple conversion, with very little or no variation (\typeC):
			\bh{hɛrʾå̄} → \C{dangos},
			\bh{nirʾå̄} → \C{ymddangos},
			\bh{hibbīṭ} → \C{edrych},
			\bh{ḥå̄zå̄} → \C{gweled},
			\bh{hišqīp̄} → \C{edrych},
			\bh{hišgīaḥ} → \C{edrych},
			\bh{rōʾɛ} → \C{gweledudd} and \bh{ḥōzɛ} → \C{gweledudd}.
		\item Meaning-bearing opposition of two main options (\typeA):
			\bh{rå̄ʾå̄} and \bh{på̄nå̄}.
		\item More variance (\typeB):
			\bh{ṣå̄p̄å̄} and \bh{šå̄r}.
	\end{compactitem}
\end{hopoint}

\begin{paper}
	As one can see, in the translation the main lexical burdon lays on two Welsh verbs: \C{edrych} and \C{gweled}.
\end{paper}
