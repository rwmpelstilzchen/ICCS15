\begin{paper}
	\section{Conclusion}

	In twenty minutes one can only scratch the surface of the linguistic complexity of Morgan’s work. Nevertheless, I hope I did manage to do \hl{two things}:
	\begin{compactitem}
		\item to present a \hl{formal method} for approaching the subtleties of the translation, by describing the linguistic system that emerges from the text
		\item to exemplify how Morgan produced a \hl{new, natural text} using the Welsh linguistic toolbox, re-creating it according to his interpretation of the original text.
	\end{compactitem}

	Both languages being rich in their ways of expression, the translator had to bridge the structual lexical differences seemingly opposite ways:
	\begin{compactitem}
	\item dispersing one Hebrew lexeme to several Welsh lexemes, as we’ve seen with \bh{på̄nå̄} being translated by both \C{edrych} and \C{troi} or \bh{šå̄maʿ} by \C{clywed} and \C{gwrando}.\tounfold{***}
	\end{compactitem}

	Thank you very much!
\end{paper}
