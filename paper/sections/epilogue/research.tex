\section{Further research}

\begin{paper}
	{\click} In the handout you can see, in a random order, several points I think worthy of \hl{further research}.
	
	In the spirit of \hl{open research}, at the last page of your handout you can see a URL for downloading the handout, presentation and transcription, as well as a subset view of the database. So I’d be really happy to \hl{discuss} this topic with you and hear your comments and ideas, in the question session in a minute, or afterwards.
\end{paper}

\tounfold{להתאים ולעדכן את הנושאים והניסוח}

\tounfold{להוסיף את העניין של ההבדלה הלקסיקלית. לתת דוגמאות של הבדלה, אבל גם של אי־הבדלה. להגיד שכל העניין דורש מחקר מיוחד}

\begin{enumerate}[A.]
	\item Comparing, typologically, the situation here with that of other translations of the Bible: other Welsh translations (the 1620 edition and the new translation of 1988/2004), other Celtic translations and translations into other, unrelated, languages. It seems to me especially enriching not only to draw general conclusions about the systemic differences and similarities, but also to examine closely particular verses and see the ways different translators in different languages, periods and cultures choose to interpret the text.


	\item Expanding the procedural working method used here to other kinds of linguistic entities: not only verbs, or even lexemic elements, but grammemic elements as well. Of special interest is, to me, the second person pronouns in languages which have, like Welsh, a \mbox{\textit{tu}:\textit{vu}} (\mbox{\C{ti}:\C{chi}(:\C{chdi})}) distinction. By examining the translator’s choices in translating from a language which has no such distinction (Biblical Hebrew, Modern English, …) into a language which does have, we can map the social hierarchies between the characters in the narrative: who is talking to whom with \textit{tu}/\C{ti} in what situations, and who with \textit{vu}/\C{chi}.\footnote{Such a distinction, I believe, will not be seen in machine translation in any foreseeable future, as it involves a deep understanding of {human} social structures.}


	\item \cite{morgan.p:1988:beibl} writes:
		\begin{changemargin}{1cm}{0cm}
			\begin{bilingquote}
				\transline
				{{\addfontfeature{HyphenChar=None}Y mae copi o Feibl Hebraeg yn perthyn i William Morgan wedi goroesi. Bu ym meddiant Arglwyddes Llanofer ac y mae heddiw yn un o drysorau Llyfrgell Genedlaethol Cymru yn Aberystwyth. Gellir gweld ar ymylon y ddalen nodiadau manwl William Morgan ar ystyron geiriau Hebraeg.}}
				{The copy of the Hebrew Bible belonging to William Morgan has survived to this day, preserved first by Lady Llanover, and now kept as one of the treasures of the National Library of Wales at Aberystwyth. In the margins of the pages may be seen detailed notes on the meanings of abstruse Hebrew words.}
			\end{bilingquote}
		\end{changemargin}
		\vspace{-0.5cm}
		A single photograph from this copy is reproduced there, as Figure~19, which is illegible in the web edition.

		Being able to read these notes can be a peerless aid to a fuller understanding of the 1588 translation: being able to glance at a record of the translation process itself. As far as I know, no facsimile edition of the copy has been made. Is there any detailed record as for the content of the notes? Is there any chance to gain access to the actual book and copy them?

	\item Comparing the system emerging from Morgan’s Bible with other kinds of Welsh. Is the same holds true in (native) Modern Welsh (literary or colloquial)? What about the contemporary Welsh (c.~1588)? What about older stages of Welsh?

	\item Morgan translated the Hebrew Bible from the Hebrew text. That’s certain. But it is also evident that he did use existing translations in addition: the classical translations (the Greek Septuagint and the Latin Vulgate) and maybe some English translations. To what extant did they affect the translation in question? When did Morgan made use of them, and for what purposes?

	\item Expanding the scope to the whole Hebrew Bible. Just as in the Pentateuch, many of the occurrences of {\shama} in the rest of the Hebrew Bible are quite trivial with regards to the opposition \mbox{\C{clywed}:\C{gwꝛando}}. Nevertheless, I am sure there are many which are interesting, i.e.\ which required careful reading of the text in order to choose one over the other, which required interpreting the text according the translator understanding thereof.

		Of special interest seems to me the cases in which \C{clywed} is used as an inclusive verb of perception (cf.\ Italian \textit{sentire}), use (\textit{b}) in the GPC:
		\begin{changemargin}{1cm}{0cm}
			\begin{bilingquote}
				\transline
				{{\addfontfeature{HyphenChar=None}Canfod neu dderbyn argraffiadau drwy’r synhwyrau (ag eithrio’r golwg); teimlo; arogleuo (cf.\ Gwydd.~C.\ \textit{atcluinti a boludh}, \textit{RC} xvi.~54); blasu; \textit{to feel}; \textit{smell}; \textit{taste}.}}
				{Perceive or receive impressions through the senses (except vision); to feel; to smell (cf.\ Mid.~Irish \textit{atcluinti a boludh}, \textit{RC} xvi.~54); to taste: […]}
			\end{bilingquote}
		\end{changemargin}
		\vspace{-0.5cm}
		The GPC gives one example from the Bible, Ps.~58:9 (here I give the whole verse), in which \C{clywed} does not translate {\shama} but another Hebrew verb:
		\begin{changemargin}{1cm}{0cm}
			\begin{bilingquote}
				\quoling
				{בְּטֶ֤רֶם \BHhighlight{יָבִ֣ינוּ} סִּֽירֹתֵיכֶ֣ם אָטָ֑ד כְּמוֹ־חַ֥י כְּמוֹ־חָ֝ר֗וֹן יִשְׂעָרֶֽנּוּ׃}
				{Cythꝛyblied ef megis [mewn] llid, fel [peth] amrwd cynn i’ch crochanau \highlight{glywed} gwꝛês y mieri.}
				{bə-ṭɛrɛm \highlight{yå̄ḇīnū} sīrōṯēḵɛm ʾå̄ṭå̄ḏ kəmō-ḥay kəmō-ḥå̄rōn yiśʿå̄rɛnnū}
				{Before your pots can \highlight{feel} the thorns, he shall take them away as with a whirlwind, both living, and in \textit{his} wrath.}
			\end{bilingquote}
		\end{changemargin}
		\vspace{-0.5cm}
\end{enumerate}
