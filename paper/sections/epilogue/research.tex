\section{Further research}

\begin{paper}
	{\click} In the handout you can see several points I think worthy of \hl{further research}; and, in the spirit of \hl{open research}, at the last page you can see a URL and a QR-code for downloading the handout, presentation and transcription, as well as a subset view of the database. So I’d be really happy to \hl{discuss} this topic with you and hear your comments and ideas, in the question session in a minute, or afterwards.
\end{paper}

\begin{enumerate}[A.]
	\item Comparing, typologically, the situation here with that of other translations of the Bible: other Welsh translations (the 1620 edition and the new translation of 1988/2004), other Celtic translations and translations into other, unrelated, languages. It seems to me especially enriching not only to draw general conclusions about the systemic differences and similarities, but also to examine closely particular verses and see the ways different translators in different languages, periods and cultures choose to interpret the text.


	\item Expanding the procedural working method used here to other kinds of linguistic entities: not only verbs, or even lexemic elements in general, but grammemic elements as well. Of special interest is, to me, the second person pronouns in languages which have, like Welsh, a \mbox{\textit{tu}:\textit{vu}} (\mbox{\C{ti}:\C{chi}(:\C{chdi})}) distinction. By examining the translator’s choices in translating from a language which has no such distinction (Biblical Hebrew, Modern English, etc.) into a language which does have it, we can map the social hierarchies between the characters in the narrative through the translator’s eyes: who is talking to whom with \textit{tu}/\C{ti} in what situations, and who with \textit{vu}/\C{chi}.\footnote{Such a distinction, I believe, will not be seen in machine translation in any foreseeable future, as it involves a deep understanding of {human} social structures.}


	\item \cite{morgan.p:1988:beibl} writes:
		\begin{changemargin}{1cm}{0cm}
			\begin{bilingquote}
				\transline
				{\Cpar{Y mae copi o Feibl Hebraeg yn perthyn i William Morgan wedi goroesi. Bu ym meddiant Arglwyddes Llanofer ac y mae heddiw yn un o drysorau Llyfrgell Genedlaethol Cymru yn Aberystwyth. Gellir gweld ar ymylon y ddalen nodiadau manwl William Morgan ar ystyron geiriau Hebraeg.}}
				{The copy of the Hebrew Bible belonging to William Morgan has survived to this day, preserved first by Lady Llanover, and now kept as one of the treasures of the National Library of Wales at Aberystwyth. In the margins of the pages may be seen detailed notes on the meanings of abstruse Hebrew words.}
			\end{bilingquote}
		\end{changemargin}
		\vspace{-0.5cm}
		A single photograph from this copy is reproduced there, as Figure~19, which is illegible in the web edition.

		Being able to read these notes can be a peerless aid to a fuller understanding of the 1588 translation, by giving us a glance at a record of the translation process itself. As far as I know, no facsimile edition of the copy has been made. Is there any detailed record as for the content of the notes? Is there any chance to gain access to the actual book and copy them?

	\item Comparing the linguistic system emerging from Morgan’s Bible with other kinds of Welsh. Is the same still holds true in (native) Modern Welsh (literary or colloquial)? What about the contemporary Welsh (c.~1588)? What about older stages of Welsh?

	\item Morgan translated the Hebrew Bible from the Hebrew text. That’s certain. But it is also evident that he did use existing translations in addition: the classical translations (the Greek Septuagint and the Latin Vulgate) and maybe some English translations. To what extant did they affect the translation in question? When did Morgan made use of them, and for what purposes?

	\item When translating a verse with a parallelism in which the Hebrew text says the same thing twice in a different manner for rhetoric purposes and the relevant phrases on both parts of the parallelism should have normally translated by the same target phrase, the translator has two options:
		\begin{compactitem}
			\item one is to retain lexical differentiation and avoid repetition by changing the translation of one part; exx.~\vref{Job}{23}{9}{} and \vref{Isa.}{2}{1}{} are examples for this strategy.
			\item the other is to flatten the lexical differentiation and translate both using the same target phrase; \exvref{Isa.}{33}{17}{} is an example for this strategy.
		\end{compactitem}
		As far as I know, this tension between two forces (retaining lexical differentiation and retaining the usual translation) is yet to be researched in the Welsh Bible. A general description, not limited to the senses, should be made.
\end{enumerate}

\begin{example}{Job}{23}{9}{}{}
	\quoling
	{שְׂמֹ֣אול בַּעֲשֹׂת֣וֹ וְלֹא־\hlA{אָ֑חַז} יַעְטֹ֥ף יָ֝מִ֗ין וְלֹ֣א \hlB{אֶרְאֶֽה}׃}
	{[Os] ar y llaw aſſwy lle y mae efe yn gweithio ni \hlA{chanfyddaf} [ef:] [os] ar y llaw ddechau, y mae efe yn ymguddio, fel na chaf [ei] \hlB{weled}.}
	{śəmōl ba·ʿăśōṯō wə·lō·\hlA{ʾå̄ḥaz} yaʿṭōp̄ yå̄mīn wə·lō \hlB{ʾɛrʾɛ}}
	{On the left hand, where he doth work, but I cannot \hlB{behold} \kjvit{him}: he hideth himself on the right hand, that I cannot \hlB{see} \kjvit{him}:}
\end{example}
\begin{compactdesc}
	\item[A:] \bh{ḥå̄zå̄} → \C{gweled} (§\ref{sight:ḥå̄zå̄})
	\item[B:] \bh{rå̄ʾå̄} → \C{gweled} (visual perception; §\ref{sight:rå̄ʾå̄:gweled:sensory})
	\item[A+B:] \bh{ḥå̄zå̄} \sout{→ \C{gweled}} → \C{canfod}; \bh{rå̄ʾå̄} → \C{gweled}
\end{compactdesc}

\begin{example}{Isa.}{33}{17}{}{}
	\quoling
	{מֶ֥לֶךְ בְּיָפְי֖וֹ \hlA{תֶּחֱזֶ֣ינָה} עֵינֶ֑יךָ \hlB{תִּרְאֶ֖ינָה} אֶ֥רֶץ מַרְחַקִּֽים׃}
	{Dy lygaid a \hlA{welant} y bꝛenin yn ei degwch: \hlB{gwelant} dir pell.}
	{mɛlɛḵ bə·yå̄p̄yō \hlA{tɛḥɛ̆zɛnå̄} ʿēnɛḵå̄ \hlB{tirʾɛnå̄} ʾɛrɛṣ marḥaqqīm}
	{Thine eyes shall \hlA{see} the king in his beauty: they shall \hlB{behold} the land that is very far off.}
\end{example}
\begin{compactdesc}
	\item[A:] \bh{ḥå̄zå̄} → \C{gweled} (§\ref{sight:ḥå̄zå̄})
	\item[B:] \bh{rå̄ʾå̄} → \C{gweled} (visual perception; §\ref{sight:rå̄ʾå̄:gweled:sensory})
	\item[A+B:] \bh{ḥå̄zå̄} → \C{gweled}; \bh{rå̄ʾå̄} → \C{gweled}
\end{compactdesc}
