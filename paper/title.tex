\setstretch{1.00}

\title{\vspace{-1cm}
	\setstretch{0.75}
	{
		\Large
		\C{llygaid [\char"017F ydd] ganddynt, ac ni welant}\emph{:}\\
		Mediating Senses through Translation Choices
	}
}
\author{%
	\normalsize Júda Ronén\\[-0.1em]
	\small The Hebrew University of Jerusalem\\[-0.1em]
	\small Department of Linguistics
}
\date{
	\setstretch{1.25}
	\normalsize 15mh Còmhdhail Eadar-Nàiseanta na Ceiltis\\
	\normalsize Glaschu 2015
}
\maketitle
\thispagestyle{empty}


\vspace{-0.5cm}
{\small
\noindent
In 1588 William Morgan published his monumental Welsh translation of the Bible. This work is notable, among other aspects, in that it has the Old Testament translated directly from the original Hebrew. This fact invites comparative study of the Welsh and Hebrew texts, which may shed light on the Welsh text and language, the translation process, and (the translator’s reading of) the original text.

\textbf{In this paper I will attempt a close examination of the lexical means by which Morgan translated Hebrew phrases concerning the senses (chiefly verbs of perception).} The Hebrew and the Welsh lexicon and grammar are structured differently; that obliges the translator to make constant meaning-bearing choices, interpreting the text according to their reading thereof. Structural description of Morgan’s lexical choices will be at the heart of the paper.

I hope the proposed description, which is based on formal linguistic grounds and aims at understanding (Bible) translations through the lens of structural linguistic analysis, will contribute to our understanding of the 1588 Bible and its language.

(This paper broadens the scope of a paper delivered at ICCS14, in which the semantic field of ‘hearing’ was in focus. Attendance at the previous paper is not required nor assumed.)
}

~

\hrule

\vfill

\begin{center}
	\LR{\includegraphics[width=0.50\textwidth]{../ps_115-5.cy.png}}
	\hfill
	\LR{\includegraphics[width=0.44\textwidth]{../ps_115-5.he.png}}\\
	Top: Psalms~115:5, Leningrad Codex, 1008/9\\
	Bottom: Psalms~115:5, William Morgan’s Bible, 1588
\end{center}

