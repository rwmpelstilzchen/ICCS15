\setcounter{section}{-1}
\section{Introduction}

\begin{frame}{}
	\begin{center}
		Introduction
	\end{center}
\end{frame}



\subsection{Translation as a decision process}

\begin{frame}{Bridging linguistic gap between languages}
	\includegraphics[width=0.50\textwidth]{../ps_115-5.cy.png}
	\hfill
	\includegraphics[width=0.44\textwidth]{../ps_115-5.he.png}

	\vfill

	Left: Psalms~115:5, William Morgan’s Bible, 1588\\
	Right: Psalms~115:5, Leningrad Codex, 1008/9
\end{frame}



\begin{frame}{***}
\end{frame}



\subsection{Corpus}

\begin{frame}{William Morgan’s 1588 edition}
	\begin{center}
		\includegraphics[width=0.9\textwidth]{images/1588.jpg}

		\vfill

		1588 printed Bible. Digital facsimile
	\end{center}
\end{frame}



\subsection{Methodology}

\begin{frame}{Methodology}
	\begin{enumerate}
		\item Making a bilingual database using a Biblical concordance
		\item Grouping verses according to translations choices
		\item Generalisation and finding patterns
	\end{enumerate}
\end{frame}



\subsection{Corpus extent}

\begin{frame}{Corpus extent}
	\begin{itemize}
		\item \bh{rå̄ʾå̄} ‘see’ and \bh{šå̄maʿ} ‘hear’:
			\begin{itemize}
				\item {Pentateuch \quad\scriptsize (Torah, \bh{tōrå̄})}
			\end{itemize}
		\item all other words and collocations: the whole bible
			\begin{itemize}
				\item {Pentateuch \quad\scriptsize (Torah, \bh{tōrå̄})}
				\item {Prophets \quad\scriptsize (Nevi’im, \bh{nəḇīʾīm})}
				\item {Writings \quad\scriptsize (Ketuvim, \bh{kəṯūḇīm})}
			\end{itemize}
	\end{itemize}
\end{frame}



\subsection{ICCS14}

\begin{frame}{ICCS14}
	\begin{center}
		\includegraphics[width=0.9\textwidth]{images/ICCS14.pdf}
	\end{center}
\end{frame}
