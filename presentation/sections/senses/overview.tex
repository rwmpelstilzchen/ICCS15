\subsection{Overview}

\subsubsection{Model}

\begin{frame}{\hopoint The senses (pentasensory model)}
	\begin{tabular}{ll@{\quad→\quad}l}
		\textbf{sight}   & \bh{rå̄ʾå̄}   & \C{gweled, edrych, …}\\
		\textbf{hearing} & \bh{šå̄maʿ}  & \C{clywed, gwrando, …}\\
		\textbf{touch}   & \bh{må̄šaš}  & \C{teimlo, …}\\
		\textbf{smell}   & \bh{hērīaḥ} & \C{arogli, …}\\
		\textbf{taste}   & \bh{ṭå̄ʿam}  & \C{archwaithu, …}
	\end{tabular}
\end{frame}



\subsubsection{Lists of modalities}

\begin{frame}{Inability of idols}
	\begin{tabular}{l|ccc}
		& Ps.~115:4–7           & Ps.~135:14–17 & Deut.~4:28\\
		\hline
		\hl{Sight}     & +                     & +             & +\\
		\hl{Hearing}   & +                     & +             & +\\
		\hl{Speech}    & \phantom{(×2) }+ (×2) & +             & \\
		\hl{Eating}    &                       &               & +\\
		\hl{Breathing} &                       & +             & \\
		\hl{Smell}     & +                     &               & +\\
		\hl{Touch}     & +                     &               & \\
		\hl{Walking}   & +                     &               &
	\end{tabular}

	\flushright{\cite[p.~68]{avrahami.y:2012:senses}}
\end{frame}



\begin{frame}{\ex\ Inability of idols \hfill (Deut.~4:28)}
	\begin{example}{Deut.}{4}{28}{}{}
		\quoling
		{וַעֲבַדְתֶּם־שָׁ֣ם אֱלֹהִ֔ים מַעֲשֵׂ֖ה יְדֵ֣י אָדָ֑ם עֵ֣ץ וָאֶ֔בֶן אֲשֶׁ֤ר לֹֽא־\hlA{יִרְאוּן֙} וְלֹ֣א \hlA{יִשְׁמְע֔וּן} וְלֹ֥א \hlA{יֹֽאכְל֖וּן} וְלֹ֥א \hlA{יְרִיחֻֽן}׃}
		{Ac yno y gwaſanaethwch dduwiau [o] waith dwylo dŷn, [ſef] pꝛen, a maen, y rhai ni \hlA{welant}, ac ni \hlA{chlywant}, ac ni \hlA{fwyttânt}, ac ni \hlA{aroglant}.}
		{wa·ʿăḇaḏtɛm·šå̄m ʾɛ̆lōhīm maʿăśē yəḏē ʾå̄ḏå̄m ʿēṣ wå̄·ʾɛḇɛn ʾăšɛr lō·\hlA{yirʾūn} wə·lō \hlA{yišməʿūn} wə·lō \hlA{yōḵlūn} wə·lō \hlA{yərīḥun}}
		{And there ye shall serve gods, the work of men’s hands, wood and stone, which neither \hlA{see}, nor \hlA{hear}, nor \hlA{eat}, nor \hlA{smell}.}
	\end{example}
\end{frame}



\begin{frame}{\ex\ Inability of idols \hfill (Ps.~115:4–7) [1/2]}
	\begin{example}{Ps.}{115}{4–5}{}{}
		\quoling
		{%
			עֲצַבֵּיהֶם כֶּ֣סֶף וְזָהָ֑ב מַ֝עֲשֵׂ֗ה יְדֵ֣י אָדָֽם׃
			פֶּֽה־לָ֭הֶם וְלֹ֣א \hlA{יְדַבֵּ֑רוּ} עֵינַ֥יִם לָ֝הֶ֗ם וְלֹ֣א \hlA{יִרְאֽוּ}׃
		}
		{%
			Eu delwau hwy [ydynt] o aur, ac arian, [ſef] o waith dynnion.
			Genau [ſydd] iddynt, ac ni \hlA{lefarant}, llygaid [ſydd] ganddynt, ac ni \hlA{welant}.
		}
		{%
			ʿăṣabbēhɛm kɛsɛp̄ wə·zå̄hå̄ḇ maʿăśē yəḏē ʾå̄ḏå̄m |
			pɛ·l·å̄hɛm wə·lō \hlA{yəḏabbērū} ʿēnayim l·å̄hɛm wə·lō \hlA{yirʾū}
		}
		{%
			Their idols \textit{are} silver and gold, the work of men’s hands.
			They have mouths, but they \hlA{speak} not: eyes have they, but they \hlA{see} not:
		}
	\end{example}
\end{frame}



\begin{frame}{\excont\ Inability of idols \hfill (Ps.~115:4–7) [2/2]}
	\begin{example}{Ps.}{115}{6–7}{}{}
		\quoling
		{%
			אָזְנַ֣יִם לָ֭הֶם וְלֹ֣א \hlA{יִשְׁמָ֑עוּ} אַ֥ף לָ֝הֶ֗ם וְלֹ֣א \hlA{יְרִיחֽוּן}׃
			יְדֵיהֶ֤ם ׀ וְלֹ֬א \hlA{יְמִישׁ֗וּן} רַ֭גְלֵיהֶם וְלֹ֣א \hlA{יְהַלֵּ֑כוּ} לֹֽא־\hlA{יֶ֝הְגּ֗וּ} בִּגְרוֹנָֽם׃
		}
		{%
			~[Y mae] cluſtiau iddynt, ac ni \hlA{chlywant}, ffroenau [ſydd] ganddynt, ac ni \hlA{aroglant}.
			Dwylo [ſydd] iddynt, ac ni \hlA{theimlant}: traed [ſy] iddynt, ac ni \hlA{cherddant}: ni \hlA{leiſiant} [ychwaith] ai gwddf.
		}
		{%
			ʾå̄znayim l·å̄hɛm wə·lō \hlA{yišmå̄ʿū} ʾap̄ l·å̄hɛm wə·lō \hlA{yərīḥūn} |
			yəḏēhɛm wə·lō \hlA{yəmīšūn} raḡlēhɛm wə·lō \hlA{yəhallēḵū} lō·\hlA{yɛhgū} bi·ḡrōnå̄m
		}
		{%
			They have ears, but they \hlA{hear} not: noses have they, but they \hlA{smell} not:
			They have hands, but they \hlA{handle} not: feet have they, but they \hlA{walk} not: neither \hlA{speak} they through their throat.
		}
	\end{example}
\end{frame}



\subsubsection{\C{gweled}:\C{edrych} \C{clywed}:\C{gwrando}}

\begin{frame}{\C{gweled}:\C{edrych} and \C{clywed}:\C{gwrando}}
	\begin{center}
		\scalebox{0.75}{
			\begin{tikzpicture}[text width=\widthof{gwrando}, align=center]
				\node at (0, 0) [circle,draw, line width = 2pt] (X) {{{\Large\bh{rå̄ʾå̄}}}};
				\node at (-2, -2) [circle,draw] (a) {\C{gweled}};
				\node at (+2, -2) [circle,draw] (b) {\C{edrych}};
				\draw [->, line width = 1pt] (X) to (a);
				\draw [->, line width = 1pt] (X) to (b);
			\end{tikzpicture}
		}
		\hfill
		\scalebox{0.75}{
			\begin{tikzpicture}[text width=\widthof{gwrando}, align=center]
				\node at (0, 0) [circle,draw, line width = 2pt] (X) {{{\Large\shama}}};
				\node at (-2, -2) [circle,draw] (a) {\C{clywed}};
				\node at (+2, -2) [circle,draw] (b) {\C{gwꝛando}};
				\draw [->, line width = 1pt] (X) to (a);
				\draw [->, line width = 1pt] (X) to (b);
			\end{tikzpicture}
		}
	\end{center}
\end{frame}



%*** \begin{frame}{\hopoint\C{gweled}:\C{edrych} and \C{clywed}:\C{gwrando}}
%*** 	\begin{center}
%*** 		\begin{tabular}{l|cc}
%*** 			& {sensory} & {\ruby{additional}{(non-sensory)} meaning}\\
%*** 			\hline
%*** 			\C{gweled}, \C{clywed}  & + & -\\
%*** 			\C{edrych}, \C{gwꝛando} & ± & +
%*** 		\end{tabular}
%*** 	\end{center}
%*** \end{frame}
