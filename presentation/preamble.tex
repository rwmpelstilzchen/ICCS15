\documentclass[xetex]{beamer}
\usepackage[framesassubsections]{beamerprosper} 

\usepackage{multicol}

\usepackage{setspace, multirow, xltxtra, longtable, ltxtable, amsmath, fancybox, rotating, needspace, multicol}

%\usepackage{beamerthemesplit}


\usepackage{tikz}
\usetikzlibrary{arrows, decorations.pathmorphing, backgrounds, positioning, fit, petri}

\useoutertheme[width=4em, hideothersubsections]{sidebar}
%\setbeamerfont{frametitle}{size=\small}
\usetheme{Berkeley}
%\usecolortheme{beetle}
\usecolortheme{dove}
\usecolortheme{sidebartab}
%\setbeamertemplate{section in sidebar}[default][20]
\setbeamerfont{section in sidebar}{size=\scriptsize}
\setbeamerfont{subsection in sidebar}{size=\tiny}
\setbeamerfont{subsubsection in sidebar}{size=\tiny}
%\definecolor{beetle@other}{RGB}{64,80,127}
%\definecolor{beetle@other}{RGB}{32,96,48}
%\setbeamercolor{alerted text}{fg=red!70!black}
%%%\setbeamercolor{alerted text}{fg=beetle@other!40!red}
%\setbeamercolor{subsection in sidebar shaded}{fg=beetle@other}
\setbeamercolor*{palette sidebar primary}{fg=black}
%\setbeamercolor{background canvas}{bg=white}
\usepackage{textpos} 
\logo{\vspace{0.2cm}\hspace{-0.25cm}\includegraphics[width=2cm]{draig.png}}

\usepackage{graphics}
%\definecolor{colquote}{rgb}{0.8,0.85,1} 
%\definecolor{colvd}{rgb}{0.8,1,0.8} 
%\newcommand{\quo}[1]{{\uwave{#1}}}
\newcommand{\quo}[1]{{\addfontfeature{Color=003388FF}{{#1}}}}
%\newcommand{\m-vd}[1]{{\addfontfeature{Color=880000FF}\textbf{#1}}}
\newcommand{\secu}[1]{{\addfontfeature{Color=883333FF}{#1}}} % secondary
\newcommand{\highlight}[1]{{{\addfontfeature{Color=880000FF}#1}}}
%\newcommand{\highlight}[1]{\textbf{#1}}
\newcommand{\BHhighlight}[1]{{{\addfontfeature{Color=880000FF}#1}}}
\newcommand{\Chl}[1]{\highlight{#1}}
\newcommand{\BHhl}[1]{\BHhighlight{#1}}

\usepackage{ulem}
\normalem
\usepackage[nooverlap]{ruby}
\renewcommand{\rubysize}{0.75}
\newcommand{\gl}[2]{\ruby{#1}{\fontspec{Palatino Linotype}{\addfontfeature{Color=555555FF}\mdseries\upshape#2\,}}}
\renewcommand{\gl}[2]{#1}
\newcommand{\Hebrew}[1]{{\RL{\fontspec[Script=Hebrew]{Guttman Hodes}#1}}}
\newcommand{\gram}[1]{{\fontspec{Palatino Linotype}\textsc{#1}}}


\usepackage{longtable}
\newenvironment{bilingquote}[0]%{
%~\\
%\vskip-0.5em\begin{parcolumns}[sloppyspaces=true, sloppy=true]{2}}{\end{parcolumns}}
{\begin{longtable}{p{0.46\linewidth}|p{0.46\linewidth}}}{\end{longtable}}

\newcommand{\transline}[2]{
%	\begin{savenotes}
%		\colchunk{\C{#1}}
%		\colchunk{#2}
%		\colplacechunks
%	\end{savenotes}
%	\vskip0.20em
{\Cpar{#1}\vspace{0.0em}} &
{{#2}\vspace{0.5em}}\\
}
\newcommand{\cotransline}[2]{\transline{\scriptsize#1}{\scriptsize#2}}


\newcommand{\Cpar}[1]{{\fontspec[Language=Welsh,SmallCapsFont={LMRoman10 Caps}, HyphenChar=None]{Gentium Basic}#1}}
%\renewcommand{\C}[1]{\Cpar{\textit{#1}}}
\renewcommand{\C}[1]{\textit{#1}}
\newcommand{\BHtr}[1]{{\fontspec[HyphenChar=None]{Gentium}\textit{#1}}}

\newcommand{\paradigma}[1]{\begin{tabular}{|c|}#1\end{tabular}}
\newcommand{\gofid}{\C{Gofid}}
\newcommand{\pisty}{\C{Y~Pistyll}}
\newcommand{\marwo}{\C{Marwolaeth~Stori}}
\newcommand{\teyny}{\C{Te~yn~y~Grug}}
\newcommand{\ymwel}{\C{Ymwelydd~i~De}}
\newcommand{\dianc}{\C{Dianc~i~Lundain}}
\newcommand{\dieit}{\C{Dieithrio}}
\newcommand{\nadol}{\C{Nadolig~y~Cerdyn}}
\newcommand{\zero}{{\fontspec{Apple Symbols}∅}}

\newcommand{\shama}{{\fontspec{Gentium}\textit{\mbox{šåmaʿ}}}}
\newcommand{\sshama}{š}
\newcommand{\YHWH}{{\fontspec[HyphenChar=None]{Gentium}\textsc{{yhwh}}}}
\newcommand{\LORD}{{\textsc{\mbox{lord}}}}

\newcommand{\slot}[1]{{{\textsc{#1}}}}
%\newcommand{\patternlimit}{{\fontspec{Apple Symbols}◇}}
%\newcommand{\pattern}[1]{\patternlimit~{#1}~\patternlimit}
\newcommand{\patternlimit}{{\fontspec{Apple Symbols}}}
\newcommand{\pattern}[1]{\patternlimit~{#1}~\patternlimit}
\newcommand{\pA}{A}
\newcommand{\pB}{B}
\newcommand{\pC}{C}
\newcommand{\Aplus}{\mbox{\pA{$+$}}}
\newcommand{\Aminus}{\mbox{\pA{$-$}}}




\newcommand{\llyfr}[1] {%
	\ifthenelse{\equal{#1}{1}}{Genesis}{%
	\ifthenelse{\equal{#1}{2}}{Exodus}{%
	\ifthenelse{\equal{#1}{3}}{Leviticus}{%
	\ifthenelse{\equal{#1}{4}}{Numbers}{%
	\ifthenelse{\equal{#1}{5}}{Deuteronomy}{#1}}}}}%
}
%\newcommand{\vref}[4]{\ref{verse:#1:#2:#3:#4}\marginnote{\setstretch{0.75}\begin{tabular}{c}\llyfr{1}~#2:#3\\עמ'~\pageref{verse:#1:#2:#3:#4}\end{tabular}}}
\newcommand{\vref}[4]{\ref{verse:#1:#2:#3:#4}}
\newcommand{\xvref}[4]{***} % references for yet to be written exx
\newcommand{\secref}[1]{\mbox{\LR{{\fontspec{SBL Hebrew}§}\LR{\ref{#1}}}}}
\newcommand{\sic}{\ {(\textit{sic})}}


\newcounter{exampleno}
\setcounter{exampleno}{0}
\newcommand{\ex}{\stepcounter{exampleno}\textbf{Ex.~\arabic{exampleno}}: }
\newcommand{\exast}{\stepcounter{exampleno}\arabic{exampleno}}
\newcommand{\exgoback}{\addtocounter{exampleno}{-1}}
\newcommand{\exphantom}{\stepcounter{exampleno}}

\renewenvironment{example}[5] % #1=book, #2=chapter, #3=verse, #4=copy (of the same verse; different copies have different hightlighting), #5=shift (of verse number)
	{
	\framesubtitle{\llyfr{#1} #2:#3\ifthenelse{\not\equal{#5}{}}{*}{}}
	}
	{
	}


\newcommand{\BH}[1]{\RL{\fontspec[Script=Hebrew]{SBL Hebrew}#1}}
\newcommand{\quoling}[4]% Cymraeg, Hebrew (Latin transcription), KJV, Hebrew (Tiberean)
{
\begin{bilingquote}
%	\transline{\C{#1}}{\fontspec{Gentium}#2}
%	\transline{\setLR{#3}}{\setRL{\BH{#4}}}
	\transline{{\fontspec[HyphenChar=None]{Junicode}#1}}{\fontspec[HyphenChar=None]{Gentium}#3}\\
	\transline{{\small\fontspec{Gentium}#4}}{\vspace{-0.9em}\begin{RTL}{\BH{\small#2}}\end{RTL}}
\end{bilingquote}
}


\usepackage{fontspec}
\setmainfont{Vesper Pro}
\setsansfont{Vesper Pro}
%%%\setsansfont{Optima} % No Ŵ/ŵ  FIXME: another font
%\setsansfont{Gentium Basic}
%%%\renewcommand{\textsc}[1]{{\fontspec{Palatino Linotype}\scshape#1}}   FIXME: another font
%\renewcommand{\textsc}[1]{{\fontspec{TeX Gyre Pagella}\scshape#1}}
\newcommand{\textscbf}[1]{\textsc{\textbf{#1}}}


\usepackage{bidi}


\setlength\fboxsep{0pt}
\setlength\fboxrule{1.5pt}
